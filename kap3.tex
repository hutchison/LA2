\section{Normalformen von Matrizen}
In dem ganzen Paragraphen sei $\K$ ein beliebiger Körper und $V$ ein endlich-dimensionaler $\K$-VR der Dimension $\dim V=n$ und $\alpha$ ein Endomorphismus von $V$.\par\medskip

Bezeichne End$(V)$ die Menge aller Endomorphismen von $V$.
Dann bildet End$(V)$ mit der Verknüpfung $\oplus$ wegen $(\alpha \oplus \beta) (v) = \alpha(v) + \beta(v) \quad \forall v\in V$ eine kommutative Gruppe.
Zusammen mit der zweiten Verknüpfung $\odot$ wegen $(k \odot \alpha)(v) = k\cdot \alpha(v) \quad \forall k\in \K, \forall v\in V$ sogar einen Vektorraum über dem zugrunde liegendem Körper $\K$.
Bezüglich $\oplus$ und $\circ$, wobei $(\alpha \circ \beta) = \alpha(\beta(v))$ die Hintereinanderausführung bezeichnet, wird $\left(\End(V),\oplus,\circ\right)$ zu einem nichtkommutativem Ring mit Einselement.\par\medskip

Für jedes $f \in \K[x]$, etwa $f(x) = a_n x^n + \ldots + a_1 x + a_0$, ist $f(\alpha) = a_n \alpha^n + \ldots + a_1 \alpha^1 + a_0 \alpha^0$ ein Endomorphismus von $V$, wobei $\alpha^0 = \id_v$.
Wie man leicht sieht ist 
\begin{align*}
    I = \left\{  f\in \K[x] \mid f(\alpha) = 0 \right\}
\end{align*}
ein Ideal von $\K[x]$. $I$ ist aber nicht das Nullideal, denn End$(V)$ ist isomorph zu dem Körper $\K^{n \times n}$ und daher gilt $\dim\End(V) \cong \dim\K^{n \times n} = n^2$.
Die Menge $\left\{ \alpha^0, \alpha, \alpha^2, \ldots, \alpha^{n^2}\right\}$ besitz aber $n^2+1$ Elemente, die damit linear abhängig sind.
Daher existieren $k_0, \ldots k_{n^2} \in \K$, nicht alle$= 0$, mit $\sum\limits_{i=0}^{n^2} k_i \cdot \alpha^i = 0$.
Damit besitzt $I$ mindestens ein Polynom $f \neq 0$.
Nach dem vorherigen Kapitel existiert also ein Polynom $m \neq 0$ mit $I = m \cdot \K[x]$, welches als normiert angesehen werden kann und durch $\alpha$ eindeutig bestimmt ist.

%---------------------------------------------Definition 3.1------------------------------------------------
%---------------------------------------------Minimalpolynom------------------------------------------------

\begin{mydef}\label{minpoly}\textit{Minimalpolynom}

    Das Polynom $m \neq 0$ kleinsten Grades, welches normiert ist und für das $m(\alpha) = 0$ gilt, heißt Minimalpolynom von $\alpha$.

    Sei $A$ eine quadratische Matrix. Dann heißt das kleinste normierte Polynom $m \neq 0$ mit $m(A) = 0$ das Minimalpolynom von $A$.
\end{mydef}

%------------------------------------------------Bemerkung--------------------------------------------------

\textit{Bemerkung:}
\begin{enumerate}
    \item Minimalpolynom und charakteristisches Polynom von $\alpha$ sind zu unterscheiden.
    \item Wenn $A$ eine Matrix bzgl. einer Basis $B$ des Endomorphismus $\alpha$ ist, dann ist das Minimalpolynom von $\alpha$ gleich dem Minimalpolynom von $A$.
        Insbesondere haben ähnliche Matrizen das gleiche Minimalpolynom.
    \item Nach der Definition ist $\deg m \leq n^2$
    \item Es ist $f(\alpha) = 0$ für ein $f \in \K[x]$ genau dann, wenn $m \mid f$.
\end{enumerate}

%------------------------------------------------Bemerkung--------------------------------------------------

\textit{Bemerkung: }\textit{Bezeichnung}

Im ganzen Paragraphen ist $m$ das Minimalpolynom von $\alpha \in \End(V)$.
Um Klammern zu sparen schreiben wir einfach $fv$ für $f(\alpha(v))$.
Des Weiteren sind $\ker (f) = \ker (f(\alpha))$ und $\IM (f)=\IM (f(\alpha))$.

%---------------------------------------------Definition 3.2------------------------------------------------
%------------------------------------------Invarianter Unterraum--------------------------------------------

\begin{mydef}\textit{Invarianter Unterraum}

    Ein Unterraum $U \leq V$ heißt genau dann invariant unter $\alpha$, wenn $\alpha(U)\subseteq U$.
\end{mydef}

%%------------------------------------------------Bemerkung--------------------------------------------------

\begin{mylemma} \label{lemdef3.2}\

    \begin{enumerate}
        \item $\ker \alpha$ und $\IM \alpha$, sowie $\ker f$ und $\IM f$  sind invariante Unterräume für beliebiges $f \in \K[x]$.

            Des Weiteren sind Summen und Schnitte invarianter Unterräume wieder invariant.
        \item \label{lemdef3.2-2} Für ein beliebiges $v \setminus \left\{ 0 \right\} \in V$ erzeugt $\left\{ v,\alpha(v),\alpha^2(v),... \right\}$ einen invarianten Unterraum.

            Solche invarianten Unterräume heißen zyklische Unterräume.
        \item Wenn $U$ ein invarianter Unterraum ist, dann kann man $\alpha$ auf $U$ einschränken und erhält einen Endomorphismus $\alpha_{/_U}$ von 
            $U$. Dieser hat ebenfalls ein Minimalpolynom, etwa $m_U$.

            Da $m(\alpha_{/_U}) = 0$ ist, folgt aus der Bemerkung zu Definition \ref{minpoly} $m_{/_U} \mid m$.
    \end{enumerate}
\end{mylemma}

%-----------------------------------------------Lemma 3.3---------------------------------------------------
%-----------------------------------------------------------------------------------------------------------

\begin{mylemma}\label{m=f*g}\

    Wenn $m = f\cdot g$ mit normierten Polynomen $f, g \in \K[x]$, dann ist $U := \ker g \geq \IM f$ und $m_{/_U} = g$.

    \textit{Beweis:}

    Es ist $(gf)V = mV = 0$, also $\IM f = fV \leq \ker g=U$. Insbesondere ist $m_{/_U} fV = 0$, also $m \mid f \cdot m_{/_U}$ und daher auch $g \mid m_{/_U}$.
    Andererseits ist $g(\alpha_{/_U})=0$ nach Definition von $U$, also $m_{/_U} \mid g$. Da beide Polynome normiert sind, folgt die Gleichheit.
\end{mylemma}

%----------------------------------------------------------------------------------------------------------
%-----------------------------------------------Satz 3.4---------------------------------------------------
%------------------------------------------zyklische Vektorräume-------------------------------------------

\begin{mysatz} \label{zyklischeVR}\textit{Zyklische Vektorräume}

    Wenn $V$ ein zyklischer Raum ist bezüglich des Endomorphismus $\alpha$, d.h. $\exists\ v$ mit $V = \left\langle v, \alpha (v), \alpha^2 (v), \ldots \right\rangle$, dann gilt:
    \begin{enumerate}
        \item \label{zykVR-1} $n = \dim V = \deg m$ und $\left\{ v,\alpha (v), \ldots, \alpha^{n - 1}(v) \right\}$ ist eine Basis von $V$.
        \item Bezüglich der Basis in \ref{zyklischeVR}.\ref{zykVR-1} hat $\alpha$ die Matrix
            \begin{align*}
                A & =
                \begin{pmatrix}
                    0       & \cdots & \cdots   & 0         & -k_0\\
                    1       & \ddots &          & \vdots    & -k_1\\
                    0       & \ddots & \ddots   & \vdots    & \vdots\\
                    \vdots  & \cdots & 1        & 0         & \\
                    0       & \cdots & 0        & 1         & -k_{n-1}
                \end{pmatrix}
                \intertext{wobei}
                m(x) & = x^n + \sum\limits_{j=0}^{n-1} k_j \cdot x^{j}
            \end{align*}
            das Minimalpolynom ist.
        \item $m$ ist zugleich auch das charakteristische Polynom.
        \item \label{zykVR-4} Wenn $m = f \cdot g$ mit normierten Polynomen $f$ und $g$ ist, dann ist $\ker g = \IM f = U$ ein invarianter Unterrum.
            $U$ ist ebenfalls zyklisch und $\dim U = \deg g$.
        \item \label{zykVR-5} Die Abbildung $g \mapsto \ker g$ ist eine bijektive Abbildung der normierten Teiler von $m$ auf die invarianten Unterräume von $V$.
    \end{enumerate}
    \textit{Beweis:}
    \begin{enumerate}
        \item Sei $r \in \N$ so gewählt, dass $\left\{ v,\alpha(v),\ldots,\alpha^{r-1}(v) \right\}$ linear unabhängig, aber $\left\{ v,\alpha(v),\ldots,\alpha^{r-1}(v),\alpha^r (v) \right\}$  linear abhängig ist.
            Offenbar ist dann $\alpha^r (v)$ eine Linearkombination von $\left\{ v,\alpha(v),\ldots,\alpha^{r-1}(v) \right\}$. Also gibt es $k_i \in \K$ mit
            \begin{align*}
                0 & = \alpha^r(v) + \sum\limits_{i = 0}^{r - 1} k_i \cdot \alpha^i (v) = pv
                \intertext{mit}
                p(x) & = x^r + \sum\limits_{i = 0}^{r - 1} k_i \cdot x^i
            \end{align*}
            und nach Wahl von $r$ ist $p$ das normierte Polynom kleinsten Grades, welches $v$ annulliert.
            Es annulliert dann aber auch alle $\alpha^j (v)$, also auch deren Erzeugnis, d.h. ganz $V$.
            Daher ist $p = m$, insbesondere $\deg m = r$.
            Per Induktion sieht man leicht, dass sich mit $\alpha^r (v)$ auch jedes $\alpha^s (v)$ für $s \geq r$ als Linearkombination von $\left\{ v,\alpha(v),\ldots,\alpha^{r-1}(v)\right\}$ schreiben lässt.
            Diese Menge ist also eine Basis von $V$, insbesondere ist $r = n$.

        \item Das Bild eines der Basisvektoren $\alpha^j (v)$ unter $\alpha$ ist $\alpha^{j+1}(v)$, also der nächste Basisvektor außer im Fall $j = n-1$. Dann ist
            \begin{align*}
                \alpha^n (v) = -\sum\limits_{i = 0}^{n - 1} k_i \cdot \alpha^i(v).
            \end{align*}
            Daher hat $A$ die angegebene Form.

        \item Es ist
            \begin{align*}
                xE-A =
                \begin{pmatrix}
                    x & \cdots & \cdots & 0 & k_0\\
                    -1 & \ddots & & \vdots &\\
                    0 & \ddots & \ddots & \vdots & \vdots\\
                    \vdots & \cdots & -1 & x & k_{n-2}\\
                    0 & \cdots & 0 & -1 & x + k_{n-1}
                \end{pmatrix}
            \end{align*}
            Die Untermatrix, welche durch Streichen der ersten Zeile und ersten Spalte entsteht, hat die gleiche Form, also (per Induktion) die Determinante
            \begin{align*}
                x^{n-1} + \sum\limits_{i=1}^{n-1} k_i \cdot x^{i-1}
            \end{align*}
            Durch Entwickeln nach der ersten Zeile erhält man dann daher
            \begin{align*}
                \det(xE - A) = x \left( x^{n-1} + \sum\limits_{i = 1}^{n - 1} k_i \cdot x^{i-1} \right) + (-1)^{n + 1} k_0 (-1)^{n-1} = m(x)
            \end{align*}

        \item In \ref{m=f*g} stehen schon viele der Aussagen. Wir zeigen, dass $\ker g \leq \IM f$.

            Dazu sei $u \in \ker g$. Weil $V$ zyklisch ist, gibt es ein Polynom $h$ mit $u = hv$, also $u \in \IM h$.
            Dann ist $0 = gu = ghv$, also nach \ref{zyklischeVR}.\ref{zykVR-1}, $m = gf$ ein Teiler von $gh$ und daher $f \mid h$.
            Folglich ist $\IM f \geq \IM h$ und insbesondere $u \in \IM f$. Es ist klar, dass $\left\{ fv, \alpha(fv), \ldots \right\}$ ein Erzeugendensystem von $ fV=U$ ist.
            Also ist $U$ zyklisch. Nach \ref{zyklischeVR}.\ref{zykVR-1} ist dann $\dim U = \deg m_{/_U} = \deg g$.

        \item Wenn $g$ und $h$ normierte Teiler von $m$ sind mit $\ker g = \ker h = U$, dann ist $g = m_{/_U} = h$ nach \ref{m=f*g}.
            Also ist die Abbildung $g \mapsto \ker g$ injektiv.

            Sie ist auch surjektiv: dazu sei $U$ ein beliebiger invarianter Unterraum.
            Setzt man $I = \left\{ h \in \K[x] \mid hv \in U \right\}$, dann ist leicht zu kontrollieren, dass $I$ ein Ideal von $K[x]$ ist, welches $m$ enthält.
            Also ist $I = f \cdot K[x]$ für ein geeignetes normiertes $f \in \K[x]$ mit $f \mid m$, etwa $m = f \cdot g$.
            Weil $V$ zyklisch ist, ist $U = Iv = f \cdot \K[x] v = fV = \IM f = \ker g$, wobei die letzte Gleichheit aus \ref{zyklischeVR}.\ref{zykVR-4} folgt.
    \end{enumerate}
\end{mysatz}

%-----------------------------------------------Lemma 3.5---------------------------------------------------
%-----------------------------------------------------------------------------------------------------------

\begin{mylemma}\label{lem3.5}\

    Seien $\lambda_1, \ldots, \lambda_i \in V^{*}$. Dann gilt:
    \begin{enumerate}
        \item $\dim \left( \bigcap\limits_{i=1}^t \ker \lambda_i \right) \geq n-t$
        \item Wenn $\mu \in V^{*}$ linear abhängig von $\left\{ \lambda_1, \ldots, \lambda_i \right\}$ ist, dann
            \begin{align*}
                \ker \mu \cap \bigcap\limits_{i=1}^t \ker \lambda_i = \bigcap\limits_{i=1}^t \ker \lambda_i
            \end{align*}
    \end{enumerate}
    \textit{Beweis:} Übungsaufgabe 
\end{mylemma}

%----------------------------------------------------------------------------------------------------------
%-----------------------------------------------Satz 3.6---------------------------------------------------
%----------------------------------------------------------------------------------------------------------

\begin{mysatz}\label{Existenz/zyklische/invarianteUR}

    Sei $\alpha \in \End(V)$ mit Minimalpolynom $m = p^r$, wobei $p$ irreduzibel ist.
    Dann gilt:
    \begin{enumerate}
        \item Es gibt einen invarianten zyklischen Unterraum $Z \leq V$ mit $m_{/_Z} = m$.
        \item \label{eziUR-2} Es existiert ein invarianter Unterraum $W$ mit $V = Z \oplus W$.
    \end{enumerate}

    \textit{Beweis:}
    \begin{enumerate}
        \item Wäre zu jedem invarianten zyklischen Unterraum $U$ das Minimalpolynom ein echter Teiler von $m$, d.h $m_{/_U}| p^{r-1}$.
            Aber dann wäre $p^{r-1}(\alpha) = 0$ entgegen der Minimalität von $m$.
        \item Nach \ref{zyklischeVR}.\ref{zykVR-4} und \ref{zyklischeVR}.\ref{zykVR-5} enthält $Z$ genau einen kleinsten invarianten Unterraum $\neq 0$, nämlich $S = p^{r-1} Z$.
            Sei $\lambda \in V^{*}$ so gewählt, dass $\lambda_{/_S} \neq 0$.
            Für jedes $i = 0, 1, \ldots$ ist $\lambda_i = \lambda \alpha^{i} \in V^{*}$.

            Sei
            \begin{align*}
                W = \bigcap\limits_{i = 0}^{\infty} \ker \lambda_i
            \end{align*}
            Dann ist $W$ invariant, denn wenn $w \in W$, dann ist $\lambda_i \alpha(w) = \lambda_{i + 1} (w)$ für alle $i$, also $\alpha(w) \in W$.
            Setzt man $t = \deg m$, so ist $\alpha^t$ und jede höhere Potenz von $\alpha$ eine Linearkombination von $\left\{ \alpha^0, \ldots, \alpha^{t - 1} \right\}$,
            folglich sind $\lambda_t, \lambda_{t+1}, \ldots$ Linearkombinationen von $\left\{ \lambda_0, \ldots, \lambda_{t - 1} \right\}$.

            Aus \ref{lem3.5} folgt jetzt, dass
            \begin{align*}
                W = \bigcap\limits_{i = 0}^{t - 1} \ker \lambda_i
            \end{align*}
            und dass $\dim W \geq n-t$.

            Als Schnitt von invarianten Unterräumen ist $Z \cap W$ invariant.
            Daher ist $Z \cap W = 0$, denn sonst wäre $S \leq Z \cap W \leq W \leq \ker \lambda$ im Widerspruch zur Wahl von $\lambda$.
            Weil $Z$ zyklisch mit Minimalpolynom $m$ ist, gilt $\dim Z = \deg m = t$ nach \ref{zyklischeVR}.\ref{zykVR-1}.
            Folglich ist dann
            \begin{align*}
                n = \dim (V) \geq \dim(Z + W) = \dim Z + \dim W \geq t + (n-t) = n
            \end{align*}
            Es folgt die Behauptung.
    \end{enumerate}
\end{mysatz}

%------------------------------------------------Beispiel---------------------------------------------------

\textit{Beispiel:}

Gegeben sei eine Abbildung $\alpha:\R^4 \mapsto \R^4$ und bezeichne $\left\{ b_1, \ldots, b_4 \right\}$ die Standardbasis.

\begin{minipage}{0.4\linewidth}
    \begin{align*}
        \alpha(b_1) & = b_1\\
        \alpha(b_2) & = b_1 + b_2\\
        \alpha(b_3) & = b_2 + b_3\\
        \alpha(b_4) & = b_4 - b_1
    \end{align*}
\end{minipage}
\begin{minipage}{0.5\linewidth}
    \begin{align*}
        \Rightarrow \qquad A =
        \begin{pmatrix}
            1 & 1 & 0 & -1\\
            0 & 1 & 1 & 0\\
            0 & 0 & 1 & 0\\
            0 & 0 & 0 & 1
        \end{pmatrix}
    \end{align*}
\end{minipage}\par\medskip

Dann sind die Elemente der Menge $\left\{ E, A, A^2 \right\}$ linear unabhängig.

Die nächste Potenz von $A$ lässt sich als Linearkombination darstellen
\begin{align*}
    A^3 =
    \begin{pmatrix}
        1 & 3 & 3 & -3\\
        0 & 1 & 3 & 0\\
        0 & 0 & 1 & 0\\
        0 & 0 & 0 & 1
    \end{pmatrix}
    = 3A^2-3A+E
\end{align*}
Damit kann man das Minimalpolynom ablesen: $m(x) = x^3 - 3x^2 + 3x-1 = (x-1)^3$.\par\medskip

Um den zyklischen invarianten Unterraum $Z$ zu bestimmen überlegt man sich zuerst, dass $\dim Z = 3$ sein muss, da $\deg m = 3$ ist.
Dann ist $Z = \left\langle b_3, \alpha(b_3), \alpha^2(b_3) \right\rangle = \langle \underbrace{b_3}_{e_1},\underbrace{b_2+b_3}_{e_2},\underbrace{b_1+2b_2+b_3}_{e_3} \rangle $.

$\alpha^3(b_3)$ ist linear abhängig von $\left\{ b_3,\alpha(b_3),\alpha^2(b_3) \right\}$:
\begin{align*}
    \alpha_{/_Z}(e_3) & = \alpha(b_1 + 2b_1 + b_3)= b_1 + 2(b_1 + b_2)+(b_2 + b_3)\\
    & = b_3 - 3\cdot(b_2 + b_3) + 3\cdot(b_1 + 2b_1 + b_3)\\
    & = e_1 - 3\cdot e_2 + 3\cdot e_3
\end{align*}
Damit ist
$A' =
\begin{pmatrix}
    0 & 0 & 1\\
    1 & 0 & -3\\
    0 & 1 & 3
\end{pmatrix} $ und $m_{/_Z}(x) = (x-1)^3 = m(x)$  wie im Satz behauptet.\par \medskip
Um den invarianten Unterraum $W$ aus Satz \ref{Existenz/zyklische/invarianteUR}.\ref{eziUR-2} zu bestimmen, muss man den Beweis vollführen.
Der kleinste invariante Unterraum $S$, der in $Z$ enthalten ist, errechnet sich durch $S = p^{r-1}Z$, also in diesem Beispiel
\begin{align*}
    (\alpha-1)^2 (e_1) & = \alpha^2 (e_1) - 2\alpha (e_1) + e_1 = b_1\\
    (\alpha-1)^2 (e_2) & = \alpha^2 (e_2) - 2\alpha (e_2) + e_2 = b_1\\
    (\alpha-1)^2 (e_3) & = \alpha^2 (e_3) - 2\alpha (e_3) + e_3 = b_1\\
    \Rightarrow S & = \left\langle b_1 \right\rangle
\end{align*}
\begin{center}
    \begin{tabular}{cccc}
        $\lambda_0 : V \mapsto \K$ & $\lambda_1 = \lambda \alpha$ & $\lambda_2 = \lambda \alpha^2$ & $\lambda_3 = \lambda \alpha^3$ \\
        $b_1 \mapsto 1$ & $b_1 \mapsto 1$ & $b_1 \mapsto 1$ & $b_1 \mapsto 1 $ \\
        $b_2 \mapsto 0$ & $b_2 \mapsto 1$ & $b_2 \mapsto 2$ & $b_2 \mapsto 3 $ \\
        $b_3 \mapsto 0$ & $b_3 \mapsto 0$ & $b_3 \mapsto 1$ & $b_3 \mapsto 3 $ \\
        $b_4 \mapsto 0$ & $b_4 \mapsto -1$ & $b_4 \mapsto -2$ & $b_4\mapsto -3$ \\
    \end{tabular}
\end{center}
Und damit erhält man dann
\begin{align*}
    \ker \lambda_1 & = \left\langle b_3, b_1 + b_4, b_2 + b_4 \right\rangle \\
    \ker \lambda_2 & = \left\langle b_1 - b_3, b_2 + b_4, 2b_1 - b_2 \right\rangle\\
    \ker \lambda_3 & = \left\langle b_2 + b_4, b_3 + b_4, 3b_1 - b_2 \right\rangle
\end{align*}
Daher folgt $W = \bigcap\limits_{i = 0}^3 \ker \lambda_i = \left\langle b_2 + b_4 \right\rangle$.

Zur Probe kann man noch leicht nachprüfen, dass $V = Z \oplus W$ gilt.

%----------------------------------------------------------------------------------------------------------
%-----------------------------------------------Satz 3.7---------------------------------------------------
%----------------------------------------------------------------------------------------------------------

\begin{mysatz}
    Wenn $m = m_1 \cdot m_2 \cdot \ldots \cdot m_r$ eine Faktorisierung des Minimalpolynoms des Endomorphismus $\alpha$ in paarweise teilerfremde Faktoren ist,
    die alle normiert sind, und $U_i = \ker m_i$ für $i = 1, \ldots, r$ dann gilt:
    \begin{enumerate}
        \item Jedes $U_i$ ist invariant.
        \item $m_i$ ist das Minimalpolynom von $\alpha_{/_U}$.
        \item $V = U_1 \oplus \ldots \oplus U_r$.
    \end{enumerate}

    \textit{Beweis:}
    \begin{enumerate}
        \item Das steht schon in der Bemerkung zu \ref{lemdef3.2}.\ref{lemdef3.2-2}.
        \item Das steht schon in \ref{m=f*g}.
        \item Induktion über $r$. Der Fall $r=1$ ist trivial, denn $\ker m =V$.

            $r=2$ Nach \ref{lem2.13} existieren Polynome $f$ und $g$ mit $f m_1 + g m_2 = 1$.

            Für jedes $v \in V$ gilt daher $v = 1 v = g m_2 v + f m_1 v \in U_1+U_2$, denn $g m_2\in \IM m_2 \leq \ker m_1 = U_1$ nach \ref{m=f*g} und ebenso $fm_1v\in U_2$.

            Wenn $v \in U_1 \cap U_2$, dann ist $v = 1v = g m_2v + f m_1 v = 0$ und damit $V = U_1 \oplus U_2$ gezeigt.

            Für $r>2$ setze $\tilde{m}_1 = m_1 \cdot m_2 \cdot \ldots \cdot m_{r-1}$ und $\tilde{U}_1 = \ker \tilde{m}_1$.

            Verwende den Fall $r = 2$ für $m = \tilde{m}_1 \cdot m_{r}$ sowie Induktion für $\alpha_{/_{\tilde{U}_1}}$.
    \end{enumerate}
\end{mysatz}

%    %----------------------------------------------------------------------------------------------------------
%    %-----------------------------------------------Satz 3.8---------------------------------------------------
%    %---------------------------------------kanonisch rationale Form-------------------------------------------
%
%    \begin{satz}\label{kanonischrationaleForm} \qquad  \par
%        Sei $\alpha$ ein Endomorphismus von $V$, dann existiert eine Basis von $V$ bzgl. der die Abbildungsmatrix die Form
%        \begin{align*}
%            A=\mata A_1&0&\cdots&0 \\ 0&\ddots&&\vdots \\ \vdots&&\ddots&0\\0&\cdots&0&A_t\mate\qquad\mbox{mit }
%            A_i=\mata 0&\cdots&\cdots&0&-k_0^{(i)}\\1&\ddots&&\vdots&\\0&\ddots&\ddots&\vdots&\vdots\\ \vdots&\cdots&1&0&\\0&\cdots&0&1&-k_{n_i-1}^{(i)} \mate
%            \in\K^{n_i\times n_i},
%        \end{align*}
%        wobei
%        \begin{align*}
%            x^{n_i} + \sum_{j=0}^{n_i-1} k_j^{(i)}\cdot x^j = p_i(x)^{s_i}
%        \end{align*}
%        mit $s_i\in \N$ und $p_i\in \K[x]$ irreduzibel ist. Dabei gilt: Wenn $d_i=\deg p_i$, dann ist $n_i=d_i\cdot s_i$ und $\sum_{i=1}^{t}=\dim V$.
%    \end{satz}
%
%    \begin{bew}
%        Faktorisiere das Minimalpolynom in Potenzen von normierten irreduziblen Faktoren, etwa
%        \begin{align*}
%            m=\prod_{j\in I} p_j^{e_j}.
%        \end{align*}
%        Mit $U_j=\ker p_j^{e_j}$ ist dann $V=U_1\oplus \ldots \oplus U_r$, wobei $p_j^{e_j}$ das Minimalpolynom von $\alpha_{|_{U_j}}$ ist. Nach 
%        \ref{Existenz/zyklische/invarianteUR} (und trivialer Induktion über die Dimension) zerfällt jedes $U_j$ in die direkte Summe von zyklischen 
%        Unterräumen. Daher findet man für jeden zyklischen Raum $Z_j$ nach \ref{zyklischeVR}$.1$ eine Basis. Dann ist die Vereinigung der Basen all dieser 
%        zyklischen Unterräume eine Basis von $V$, bzgl. der die Abbildungsmatrix die behauptete Gestalt hat.
%    \end{bew}
%
%    %---------------------------------------------Definition 3.9------------------------------------------------
%    %-----------------------------------------kanonisch rationale Form------------------------------------------
%
%    \begin{mydef} \qquad \par
%        Eine Matrix der Gestalt aus Satz \ref{kanonischrationaleForm} heißt Matrix in kanonisch rationaler Form von $A$.
%    \end{mydef}
%
%    %-----------------------------------------------Lemma 3.9---------------------------------------------------
%    %-----------------------------------------------------------------------------------------------------------
%
%    \begin{lem}\label{lem3.10} \qquad \par
%        Sei A in kanonisch rationaler Form. Dann gelten:
%        \begin{itemize}
%            \item [1.] Für jedes $f\in \K[x]$ ist $f(A)=\mata f(A_1) & & \\ & \ddots & \\ & & f(A_t)\mate$.
%            \item [2.] Wenn $m_i$ das Minimalpolynom von $A_i$ ist, dann ist $m$ das kleinste gemeinsame Vielfacher der $m_i$.
%        \end{itemize}
%    \end{lem}
%
%    \begin{bew}
%        1. Diese Aussage ist trivial.\par\medskip
%        2. Nach 1. ist $f(A)=0$ genau dann, wenn $f(A_i)=$ für jedes $i$ gilt. Nach der Bemerkung zu Definition \ref{minpoly} ist das äquivalent zu der 
%        Aussage, dass $m_i|f$ für jedes $i$. Daraus folgt die Behauptung.
%    \end{bew}
%
%    %----------------------------------------------------------------------------------------------------------
%    %-----------------------------------------------Satz 3.11--------------------------------------------------
%    %--------------------------------------------Cayley-Hamilton-----------------------------------------------
%
%    \begin{satz}\label{CayleyHamilton} \textit{Satz von Cayley-Hamilton} \par
%        \begin{itemize}
%            \item [1.] Das Minimalpolynom ist ein Teiler des charakteristischen Polynoms.
%            \item [2.] $\alpha$ ist Nullstelle des charakteristischen Polynoms.
%            \item [3.] Jeder irreduzible Teiler des charakteristischen Polynom ist auch Teiler des Minimalpolynoms.
%        \end{itemize}
%    \end{satz}
%
%    \begin{bew}
%        Sei A eine Matrix zu $\alpha$ in kanonischer rationaler Form und $m_i$ das Minimalpolynom von $A_i$ wie oben. Dann ist $m$ das kleinste gemeinsame 
%        Vielfache der $m_i$ nach \ref{lem3.10}$.2$. Für jedes $i$ ist $m_i$ zugleich das charakteristische Polynom von $A_i$ nach \ref{zyklischeVR}$.3$. 
%        Daher ist das charakteristische Polynom von $A$ gleich dem Produkt der $m_i$. Daraus folgen alle Behauptungen.
%    \end{bew}
%
%
%    \newpage
%
%
%
%    %----------------------------------------------------------------------------------------------------------
%    %-----------------------------------------------Satz 3.12--------------------------------------------------
%    %----------------------------------------------------------------------------------------------------------
%
%    \begin{satz}\label{anzahlbloeckeKRF} \qquad \par
%        Sei $B$ die Matrix eines Endomorphismus $\alpha$ die ähnlich ist zu einer Matrix der Form 
%        \begin{align*}
%            A=\mata A_1 & & \\ & \ddots & \\ & & A_t\mate.
%        \end{align*} 
%        Dann ist $A$ durch $B$ bis auf die Reihenfolge der $A_i$ eindeutig bestimmt. Sei $p$ ein irreduzibles Polynom und $0<k\in \N$, dann lässt sich die 
%        Anzahl  $z(p,k)$ der Blöcke $A_i$ mit $m_i=p^k$ aus der Formel 
%        \begin{align*}
%            z\cdot \deg p = \RG p^{k-1}(B) + \RG p^{k+1}(B) - 2\RG p^k(B)
%        \end{align*}
%        bestimmen.
%    \end{satz}
%
%    \begin{bew}
%        Die erste Aussage ist nur eine Umformulierung von \ref{kanonischrationaleForm}. Die behauptete Eindeutigkeit von $A$ folgt aus der Formel für $z$. 
%        Es genügt also, diese zu beweisen. Ähnliche Matrizen haben den gleichen Rang. Daher reicht es aus, den Beweis für $A$ zu zeigen. Nach Lemma 
%        \ref{lem3.10} ist für ein $f\in\K[x]$ stets $\RG f(A) = \sum_{i=1}^t \RG f(A_i)$ und daher reicht es aus nur einen Block $A_i$ zu betrachten und 
%        \begin{align*}
%            r_i = \RG p^{k-1}(A_i) + \RG p^{k+1}(A_i)-2\RG p^k(A_i) = \begin{cases} \deg p & m_i=p^k\\0 & \mbox{sonst} \end{cases}
%            \end{align*}
%            zu zeigen.\par
%            Sei also $m_i=q^s$ mit einem irreduziblen Polynom $q$. Wenn $p\neq q$, dann sind $p^k$ und $m_i$ teilerfremd. Nach Lemma \ref{lem2.13} existieren 
%            $f,g\in\K[x]$ mit $1=f\cdot p^k+g\cdot m_i$. Durch Einsetzen von $A_i$ folgt
%            \begin{align*}
%                E=f(A_i)\cdot p^k(A_i) + g(A_i)\cdot m_i(A_i) = f(A_i)\cdot p^k(A_i),
%            \end{align*}
%            denn $m_i(A)=0$. Hierbei ist $E$ die Einheitsmatrix passender Größe. Insbesondere ist $p^k(A_i)$ invertierbar, hat also vollen Rang. Mit derselben 
%            Begründung gilt dies auch für $p^{k-1}(A_i)$ und $p^{k+1}(A_i)$. Da die drei Matrizen den gleichen Rang haben, ist $r_i=0$.\par
%            Sei also $p = q$. Wenn $s<k$, dann ist $s\leq k-1$ und daher $p^{k-1}(A_i)=0$ und erst recht $p^k(A_i)=p^{k+1}(A_i)=0$. Auch in diesem Fall haben 
%            also alle beteiligten Matrizen den Rang (diesmal $0$), und daher ist $r_i=0$.\par
%            Wenn $s=k$, dann ist wieder $p^k(A_i)=p^{k+1}(A_i)=0$. Dagegen ist $\RG p^{k-1}(A_i)=\deg p$ nach \ref{zyklischeVR}. In diesem Fall ist also 
%            $r_i=\deg p$. \par
%            Mit \ref{zyklischeVR} wird auch der letzte Fall diskutiert. Wenn $s>k$, etwa $s=k+t$ dan ist $m_i=p^s=p^k\cdot p^t$ eine Faktorisierung. Demnach ist 
%            also $\RG p^k(A_i)=\deg p^t = t\dot \deg p$. Genauo ist $\RG p^{k+1}(A_i)=\deg p^{t+1} = (t+1)\dot \deg p$ und 
%            $\RG p^{k-1}(A_i)=\deg p^{t-1} = (t-1)\dot \deg p$. Daraus ergibt sich wieder $r_i=0$.
%        \end{bew}
%
%
%
%        \newpage
%
%
%
%        %------------------------------------------------Beispiel---------------------------------------------------
%
%        \begin{bsp} \qquad \par
%            Sei
%            \begin{align*}
%                B=\mata -3&-1&4&-3&-1\\1&1&-1&1&0\\-1&0&2&0&0\\4&1&-4&5&1\\-2&0&2&-2&1 \mate \in \IR^{5 \times 5} 
%            \end{align*}
%            diese Matrix bzgl. der Standardvektoren. Dann ist $\mbox{charpol }B=(x-1)^4\cdot (x-2)$. Dieses besitzt also die beiden irreduziblen Faktoren $p_1(x)=x-1$ und $p_2(x)=x-2$. Dann ist \par\medskip
%            \begin{tabular}{cc}
%                \begin{minipage}{5cm}
%                    \begin{eqnarray*}
%                        \RG (B-E) & = & 3 \\ \RG (B-E)^2 &=&2\\ \RG (B-E)^3 &=& 1 \\ \RG (B-E)^4 &=& 1
%                    \end{eqnarray*}
%                \end{minipage}
%                &
%                \begin{minipage}{5cm}
%                    \begin{tabular}{c|c|c}
%                        $k$  &  $\RG p_1^k(B)$  &  $z\cdot\deg p_1$  \\ \hline
%                        $0$  &       $5$        &         $-$                            \\ 
%                        $1$  &       $3$        &         $1$                            \\ 
%                        $2$  &       $2$        &         $0$                            \\ 
%                        $3$  &       $1$        &         $1$                            \\ 
%                        $4$  &       $1$        &         $0$                            \\ 
%                    \end{tabular}
%                \end{minipage}
%            \end{tabular} \par \medskip 
%
%            Die Einträge in der letzten Spalte wurden mit der Formel aus Satz \ref{anzahlbloeckeKRF} errechnet. Die Ränge von $p_1^k(B)$ wurden so lange 
%            bestimmt, bis sich der Rang für ein bestimmtes $k$ nicht mehr ändert. 
%
%
%            Aus der Tabelle entnimmt man, dass die kanonisch rationale Form der Matrix $B$ einen $1$-er Block zu $m_{A_1}=x-1$ und einen $3$-er Block zu 
%            $m_{A_2}=(x-1)^3=x^3-3x^2+3x-1$ besitzt. Zusätzlich kommt dann ein $1$-er Block zu $m_{A_3}=x-2$ dazu. Daher hat die kanonisch rationale Form der 
%            Matrix $B$ die folgende Gestalt.
%            \begin{align*}
%                A=\mata 1&&&& \\ &0&0&1&\\ &1&0&-3& \\ &0&1&3& \\ &&&&2 \mate
%            \end{align*}
%            Ist man zusätzlich an der Transformationmatrix interessiert ($A$ und $B$ sind ähnlich, d.h. $A=T^{-1}BT$), so muss man die verallgemeinerten 
%            Eigenräume zu den Blöcken betrachten. Es ist (zum $3$-er Block)
%            \begin{align*}
%                \ker (B-E)^2 = \left\langle \mata0\\0\\1\\1\\0\mate,\mata1\\0\\1\\0\\0\mate,\mata0\\1\\0\\0\\-1\mate \right\rangle \qquad \mbox{und} \qquad
%                \ker (B-E)^3 = \left\langle \mata0\\0\\0\\0\\1\mate,\mata0\\0\\1\\1\\0\mate,\mata1\\0\\1\\0\\ 0\mate,\mata0\\1\\0\\0\\0\mate\right\rangle
%            \end{align*}
%            Nun wählt man ein Element $b_2$ aus dem Kern von $(B-E)^3$, welches nicht im Kern von $(B-E)^2$ liegt, also $b_2=(0,0,0,0,1)^T$. Dann erzeugt man 
%            ausgehend von diesem Element den Invarianten Unterraum, also
%            \begin{align*}
%                b_2=\mata0\\0\\0\\0\\1\mate \qquad \alpha(b_2) =b_3= \mata-1\\0\\0\\1\\1\mate \qquad \alpha^2(v)=b_4=\mata-1\\0\\1\\2\\1\mate
%            \end{align*}
%            Zu den beiden $1$-er Blöcken und den dazugehörigen Eigenwerten müssen die entsprechenden Eigenvektoren berechnet werden, also
%            \begin{align*}
%                b_1=\mata1\\0\\1\\0\\0\mate\mbox{ ist EV zum EW }\lambda_1=1 \mbox{ und } b_5=\mata0\\1\\2\\3\\-2\mate \mbox{ ist EV zum EW }\lambda_2=2
%            \end{align*}
%            Damit hat man die Transformationsmatrix $T=(b_1,b_2,b_3,b_4,b_5)$ gefunden, mit der man zur Probe $TA = TB$ durchrechnen kann.
%
%        \end{bsp}
%
%        %------------------------------------------------Beispiel---------------------------------------------------
%
%        \begin{bsp}
%            Sei
%            \begin{align*}
%                B=\mata3&-1&0&0&-2\\1&3&1&0&0\\0&1&3&0&2\\1&0&1&3&1\\0&0&0&0&3\mate
%            \end{align*}
%            wieder bzgl. der Standardbasis. Dann ist $p(x)=(x-3)^5$ das charakteristische Polynom von $B$.\par \medskip
%            \begin{tabular}{cc}
%                \begin{minipage}{5cm}
%                    \begin{eqnarray*}
%                        \RG (B-3E) &=& 3\\ \RG (B-3E)^2=1 \\ \RG (B-3E)^3 &=&0
%                    \end{eqnarray*}             
%                \end{minipage}
%                &
%                \begin{minipage}{5cm}
%                    \begin{tabular}{c|c|c}
%                        $k$   &   $\RG p^k(B)$   &   $z\cdot \deg p$ \\ \hline
%                        $0$   &       $5$        &       $-$                             \\
%                        $1$   &       $3$        &       $0$                             \\
%                        $2$   &       $1$        &       $1$                             \\
%                        $3$   &       $0$        &       $1$                             \\
%                    \end{tabular}
%                \end{minipage}
%            \end{tabular} \par \medskip
%            Aus der Tabelle erkennt man, dass $m(x)=m_1(x)=(x-3)^3$ das Minimalpolynom der Matrix $B$ ist. Also muss es mindestens einen $3$-er Block in der 
%            kanonisch rationalen Form der Matrix $B$ geben. Zusätzlich existiert dann noch ein $2$-er Block zu dessen Minimalpolynom $m_2(x)=(x-3)^2$. Die 
%            kanonisch rationale Form der Matrix hat also die folgende Gestalt.
%            \begin{align*}
%                A=\mata0&0&27&&\\1&0&-27&&\\0&1&9&&\\&&&0&-9\\&&&1&6\mate
%            \end{align*}
%            Sei nun noch die Transformationsmatrix gesucht, dann geht man wieder blockweise vor. Für den Block der Größe $3$ sucht man sich ein Element 
%            $b_1\in\ker(B-3E)^3\backslash\ker(B-3E)^2$ und bildet den dazugehörigen zyklischen Unterraum. Zu dem Block der Größe $2$ sucht man sich ein Element 
%            $b_4\in\ker(B-3E)^2\backslash\ker(B-3E)$ und bildet auch hier den dazugehörigen zyklischen Unterraum. Es ist
%            \begin{eqnarray*}
%                \ker(B-3E)^3&=& \IR^5\qquad\ker(B-3E)^2=\left\langle\mata0\\1\\0\\0\\0\mate,\mata0\\0\\0\\1\\0\mate,\mata0\\0\\0\\0\\1\mate,\mata1\\0\\-1\\0\\0\mate\right\rangle \\
%                \ker(B-3E)  &=& \left\langle \mata1\\0\\-1\\0\\0\mate,\mata0\\0\\0\\1\\0\mate \right\rangle
%            \end{eqnarray*}
%            Daher ist (lineare Unabhängigkeit bei der Auswahl beachten)
%            \begin{eqnarray*}
%                b_1 &=& \mata1\\0\\0\\0\\0\mate \qquad \alpha(b_1)=b_2=\mata3\\1\\0\\1\\0\mate \qquad \alpha^2(b_1)=b_3=\mata8\\6\\1\\6\\0\mate \\
%                b_4 &=& \mata0\\0\\0\\0\\1\mate \qquad \alpha(b_4)=b_5=\mata-2\\0\\2\\1\\3\mate
%            \end{eqnarray*}
%            Und man erhält somit schließlich $T=(b_1,b_2,b_3,b_4,b_5)$ die Transformationsmatrix.
%        \end{bsp}
%
%        %------------------------------------------------Bemerkung--------------------------------------------------
%
%        \begin{bem}\qquad \par
%            Sei $\alpha$ ein Endomorphismus von $V$ mit Minimalpolynom $m$ über $\K$ und $\K$ algebraisch abgeschlossen, d.h. alle Polynome aus $\K[x]$ 
%            zerfallen vollständig in Linearfaktoren.
%        \end{bem}
%
%        %----------------------------------------------------------------------------------------------------------
%        %-----------------------------------------------Satz 3.13--------------------------------------------------
%        %----------------------------------------------------------------------------------------------------------
%
%        \begin{satz}\label{jordanscheNormalform} \textit{Jordansche Normalform}\par
%            Bezüglich einer geeigneten Basis von $V$ hat $A$ die Gestalt
%            \begin{align*}
%                A=\mata A_1 & &0\\ & \ddots & \\ 0& & A_t \mate \qquad \mbox{ mit } A_i=\mata \lambda_i&&&0\\1&\ddots&&\\&\ddots&\ddots&\\0&&1&\lambda_i\mate.
%            \end{align*}
%            Dabei heißt $A$ die Jordansche Normalform und die $A_i$ werden als Jordankästchen bezeichnet. $A$ ist bis auf die Reihenfolge der $A_i$ eindeutig 
%            bestimmt. 
%        \end{satz}
%
%        \begin{bew}
%            Existenz einer geeigneten Basis: Nach dem Beweis von \ref{kanonischrationaleForm} existiert eine Zerlegung \par $V=Z_1\oplus\cdots\oplus Z_t$ mit 
%            zyklischen invarianten Unterräumen, deren Minimalpolynome $m_i$ Potenzen von irreduziblen Teilern (nach \ref{CayleyHamilton}) des charakteristischen 
%            Polynoms von $\alpha$ sind. Nun muss für jedes $Z_i$ eine Basis gefunden werden, sodass die Matrix von $\alpha_{|_{Z_i}}$ die Gestalt eines 
%            Jordan-Kästchens hat. Daher darf man annehmen, dass $V=Z_1$ zyklisch ist und dass $m=(x-\lambda)^n$ für geeignetes $\lambda \in \K$ gilt.
%            Daher gibt es nach \ref{zyklischeVR} eine Basis $v,\alpha(v),\ldots,\alpha^{n-1}(v)$.\par
%            Da $(\alpha-\lambda)^n$ das normierte Polynom kleinsten Grades ist mit $(\alpha-\lambda)^nv = 0$ ist, sind die Vektoren 
%            $b_1=v, b_2=(\alpha-\lambda)(v),\ldots,b_n=(\alpha-\lambda)^{n-1}(v)$ linear unabhängig, bilden also eine Basis. Aus $(\alpha-\lambda)(b_k)=b_{k+1}$ 
%            (für $k<n$) bzw. $(\alpha-\lambda)(b_n)=(\alpha-\lambda)^n(v) = 0$ folgt $\alpha(b_k) = b_{k+1} + \lambda b_k$ (für $k<n$) bzw. 
%            $\alpha(b_n)= \lambda b_n$. Bezüglich dieser Basis hat $\alpha$ also die angegebene Form. \par
%            Die Eindeutigkeit folgt aus der folgenden Formel:
%            \begin{align*}
%                z= \RG (\alpha-\lambda)^{k-1} + \RG (\alpha-\lambda)^{k+1} - 2\cdot \RG (\alpha-\lambda)^{k}
%            \end{align*}
%            denn das ist genau die Formel aus \ref{kanonischrationaleForm} mit $\deg p=1$.
%        \end{bew}
%
%        %------------------------------------------------Bemerkung--------------------------------------------------
%
%        \begin{bem}\qquad \par
%            \begin{itemize}
%                \item [1.]Um die Jordansche Normalform zu berechnen, müssen zu jedem Eigenwert $\lambda$ ide Ränge von $(\alpha-\lambda)^s$ bestimmt werden.
%                \item [2.]Eine Basis zu jedem Block der Größe $k$ erhält man durch
%                    \begin{align*}
%                        v\in \ker(\alpha-\lambda)^k\backslash \ker(\alpha-\lambda)^{k-1} \\ 
%                        \Rightarrow \left\{ v,(\alpha-\lambda)v,\ldots,(\alpha-\lambda)^{k-1},\ldots\right\}
%                    \end{align*}
%                    Insbesondere ist $\ker(\alpha-\lambda)\leq \ker(\alpha-\lambda)^2\leq \ldots \leq \ker(\alpha-\lambda)^k=\ker(\alpha-\lambda)^{k+1}$
%            \end{itemize}
%        \end{bem}
%
%        %--------------------------------------------Definition 3.14-----------------------------------------------
%        %-----------------------------------------------nilpotent--------------------------------------------------
%
%        \begin{mydef}\textit{nilpotent} \par
%            $\alpha$ heißt genau dann nilpotent, wenn $\exists k\in \N$ mit $\alpha^k=0$
%        \end{mydef}
%
%        %-----------------------------------------------Bemerkung--------------------------------------------------
%
%        \begin{bem}\qquad \par
%            \begin{itemize}
%                \item [1.]Der Enomorphismus $\alpha$ ist genau dann nilpotent, wenn $x^n$ das charakteristische Polynom von $\alpha$ ist.
%                \item [2.]Das Minimalpolynom entspricht dann dem charakteristischen Polynom.
%            \end{itemize}
%        \end{bem}
%
%        \newpage
