\section{Eigenwerte und Eigenvektoren}

Sei $V$ ein $n$-dimensionaler Vektorraum über dem Körper $\K$.

%---------------------------------------------Definition 1.3------------------------------------------------
%----------------------------------------Eigenwert, Eigenvektor---------------------------------------------
\begin{mydef} \textit{Eigenwert und Eigenvektor}

    Sei $\alpha$ ein Endomorphismus des $\K$-VR V und $\lambda\in \K$. $\lambda$ heißt genau dann Eigenwert von $\alpha$, wenn ein Vektor $v\in 
    V\backslash\lbrace0\rbrace$ existiert mit
    \begin{align*}
        \alpha(v) = \lambda v
    \end{align*}
    Der Vektor $v$ heißt dann Eigenvektor zum Eigenwert $\lambda$.
\end{mydef}

%-----------------------------------------------Beispiel---------------------------------------------------
\textit{Beispiel:}

Sei $V=\R^2$ und der Endomorphismus $\alpha:(x,y)\mapsto (x+y,x+y)$.
\begin{align*}
    \Rightarrow \quad & \alpha(x,y)=(x+y,x+y) = \lambda(x,y)=(\lambda x,\lambda y)\\
    & \begin{array}{cccrccc}
        \lambda=0 & \Rightarrow & y= & -x & \Rightarrow & v= &\langle \begin{pmatrix}1\\-1\end{pmatrix} \rangle \\
        \lambda=2 & \Rightarrow & y= &  x & \Rightarrow & v= &\langle \begin{pmatrix}1\\ 1\end{pmatrix} \rangle
    \end{array}
\end{align*}
Bezüglich der Standardbasis des $\R^2$ hat $\alpha$ die Abbildungsmatrix $A=\begin{pmatrix}1&1\\1&1\end{pmatrix}$, bzgl. der Basis $\langle(1,1),(1,-1)\rangle$ die
Abbildungsmatrix $D=\begin{pmatrix}2&0\\0&0\end{pmatrix}$. Dann nennt man $A$ ähnlich zu $D$, denn $A'=T^{-1}AT$, wobei $T$ die Matrix des Basiswechsels ist.\\

Bonusfrage: wie sieht $A^{n}$ aus?

Dazu betrachte man $A$ bzgl. der Eigenvektorbasis $\left\langle (1,1),(1,-1) \right\rangle$:
\begin{align*}
    A' & =
    \begin{pmatrix}
        2 & 0\\
        0 & 0
    \end{pmatrix}\\
    A' & = T^{-1} A T\\
    \langle (1,0),(0,1) \rangle & \overset{T}{\to} \langle (1,1),(1,-1) \rangle\\
    T & = 
    \begin{pmatrix}
        1 & 1\\
        1 & -1
    \end{pmatrix}
    \intertext{Und nun wird die $n$-te Potenz berechnet:}
    A & = T A' T^{-1}\\
    \uline{A^n} & = (T A' T^{-1})(T A' T^{-1}) \ldots (T A' T^{-1}) = \uuline{T {A'}^n T^{-1}}\\
    & = -\frac{1}{2}
    \begin{pmatrix}
        1 & 1\\
        1 & -1
    \end{pmatrix}
    \begin{pmatrix}
        2^n & 0\\
        0 & 0
    \end{pmatrix}
    \begin{pmatrix}
        -1 & -1\\
        -1 & 1
    \end{pmatrix}
    =
    -\frac{1}{2}
    \begin{pmatrix}
        2^n & 0\\
        2^n & 0
    \end{pmatrix}
    \begin{pmatrix}
        -1 & -1\\
        -1 & 1
    \end{pmatrix}\\
    & =
    -\frac{1}{2}
    \begin{pmatrix}
        -2^n & -2^n\\
        -2^n & -2^n
    \end{pmatrix}
    =
    \uuline{
    \begin{pmatrix}
        2^{n-1} & 2^{n-1}\\
        2^{n-1} & 2^{n-1}
    \end{pmatrix}
    }
\end{align*}

\newpage

%----------------------------------------------------------------------------------------------------------
%-----------------------------------------------Satz 1.2---------------------------------------------------
%----------------------------------------------------------------------------------------------------------
\begin{mysatz}\ \\
    Sei $\alpha$ ein Endomorphismus von $V$, dann sind folgende Aussagen äqiuvalent
    \begin{enumerate}
        \item $\lambda$ ist Eigenwert von $\alpha$
        \item $\det(\lambda \id_V-\alpha)=0$
        \item rg$(\lambda \id_V-\alpha)<n$
        \item $\ker(\lambda \id_V-\alpha) \neq \left\{ 0 \right\}$
    \end{enumerate}
\end{mysatz}

%---------------------------------------------Definition 1.3------------------------------------------------
%----------------------------------------charakteristisches Polynom-----------------------------------------
\begin{mydef}\label{charpol} \textit{charakteristisches Polynom}
    \begin{enumerate}
        \item Sei $A$ eine $n\times n$-Matrix. Das charakteristische Polynom von $A$ (charpol $A$) ist das Polynom
            \begin{align*}
                \det(\lambda\cdot E-A)\in \K[\lambda].
            \end{align*}
        \item Sei $\alpha$ ein Endomorphismus von $V$. Das charakteristische Polynom von $\alpha$ ist charpol $\alpha=\charpol A$, wobei $A$ die Darstellungsmatrix von $\alpha$ bzgl. einer Basis von $V$ ist.
    \end{enumerate}
\end{mydef}

%-----------------------------------------------Bemerkung---------------------------------------------------
\textit{Bemerkung:}
\begin{enumerate}
    \item Die Eigenwerte von $\alpha$ sind die Nullstellen des charakteristischen Polynoms.
    \item Ähnliche Matrizen haben dasselbe charpol
        \begin{eqnarray*}
            \charpol A' &=& \det(\lambda E-T^{-1}AT) = \det(T^{-1}\lambda ET-T^{-1}AT)=\det(T^{-1}(\lambda E-A)T)\\
            &=& \det(T^{-1}) \cdot \det(\lambda E-A) \cdot \det(T) = \det(\lambda E-A)\\
            &=& \charpol A
        \end{eqnarray*}
\end{enumerate}

\textit{Beispiel:} $V = \R^2$
\begin{align*}
    A & =
    \begin{pmatrix}
        1 & -1\\
        1 & 1
    \end{pmatrix}\\
    \charpol A & = \det
    \begin{pmatrix}
        \lambda - 1 & -1\\
        -1 & \lambda - 1
    \end{pmatrix}
    = \lambda^2 -2 \lambda + 2\\
    \lambda_{1,2} & = 1 \pm \sqrt{1-2} \notin \R\\
    & \Rightarrow \text{keine Eigenwerte}
\end{align*}

\textit{Anmerkung:}

Für die Fälle $n=2$ und $n=3$ existieren Formeln für das charakteristische Polynom:
\begin{align*}
    p(\lambda) & = \lambda^2 - \left( \Sp A \right) \lambda  + \det A & (n=2)\\
    p(\lambda) & = -\lambda^3 + \left( \Sp A \right) \lambda^2 - k_1 + \det A & (n=3)\\
    k_1 \text{ ist für } A & = 
    \begin{pmatrix}
        a & b & c\\
        d & e & f\\
        g & h & j
    \end{pmatrix}
    \text{ definiert als }
    k_1 = 
    \begin{vmatrix}
        a & b\\
        d & e
    \end{vmatrix}
    +
    \begin{vmatrix}
        a & c\\
        g & j
    \end{vmatrix}
    +
    \begin{vmatrix}
        e & f\\
        h & j
    \end{vmatrix}
\end{align*}

\newpage

\textit{Anmerkung:}
\textsc{Horner}-Schema\footnote{William George Horner, * 1786 in Bristol; $\dagger$ 22. September 1837 in Bath, englischer Mathematiker}
\begin{itemize}
    \item[Vor.:] Polynom $f(x) = a_n x^n + a_{n-1} x^{n-1} + \ldots + a_1 x + a_0$
    \item[Bsp.:] $f(x) = 2x^4 + x^3 -3x -34$
\end{itemize}
\begin{itemize}
    \item \textit{Berechne} $f(-1)$!

        Für das \textsc{Horner}-Schema benötigt man eine kleine Tabelle mit 3 Zeilen und $n+2$ Spalten, wobei $n$ der Grad des Polynoms ist.
        \begin{enumerate}
            \item In die Kopfzeile schreibt man alle Koeffizienten des Polynoms; in das 1. Feld der 2. Zeile den zu berechnenden Wert:

                \hspace{3cm}
                    \begin{tabular}{c|ccccc}
                        & $2$   & $1$   & $0$   & $-3$  & $-34$ \\\hline
                        $-1$    &       &       &       &       &       \\\hline
                        \phantom{$-1$}
                    \end{tabular}
            \item Das 2. Feld der 2. Zeile bleibt leer. Darunter jedoch (2. Feld, 3. Zeile) schreibt man den Wert, der in der 1. Zeile steht (also $2$).

                \hspace{3cm}
                    \begin{tabular}{c|ccccc}
                        & $2$   & $1$   & $0$   & $-3$  & $-34$ \\\hline
                        $-1$    &       &       &       &       &       \\\hline
                        & $2$
                    \end{tabular}
                \hspace{2cm}
                $\downarrow$
            \item Nun multipliziert man den eben eingetragenen Wert mit der Zahl ganz links (also dem zu berechnenden Argument) und trägt ihn das Feld rechts darüber ein.

                \hspace{3cm}
                    \begin{tabular}{c|ccccc}
                        & $2$   & $1$   & $0$   & $-3$  & $-34$ \\\hline
                        $-1$    &       & $-2$  &       &       &       \\\hline
                        & $2$
                    \end{tabular}
                \hspace{1.3cm}
                $\cdot (-1) \quad \nearrow$
            \item Jetzt addiert man die Werte der 3. Spalte und trägt das Ergebnis in das unterste Feld ein.

                \hspace{3cm}
                    \begin{tabular}{c|ccccc}
                        & $2$   & $1$   & $0$   & $-3$  & $-34$ \\\hline
                        $-1$    &       & $-2$  &       &       &       \\\hline
                        & $2$   & $-1$
                    \end{tabular}
                \hspace{1.5cm}
                $+ \quad \downarrow$
            \item Dieses Prinzip führt man bis zum Ende durch (mit dem Wert ganz links multiplizieren; in das Feld rechts darüber eintragen; summieren).
                \begin{center}
                    \begin{tabular}{c|ccccc}
                        & $2$   & $1$   & $0$   & $-3$  & $-34$ \\\hline
                        $-1$    &       & $-2$  & $1$   &       &       \\\hline
                        & $2$   & $-1$
                    \end{tabular}
                    $\Rightarrow \ldots \Rightarrow$
                    \begin{tabular}{c|ccccc}
                        & $2$   & $1$   & $0$   & $-3$  & $-34$ \\\hline
                        $-1$    &       & $-2$  & $1$   & $-1$  & $4$   \\\hline
                        & $2$   & $-1$  & $1$   & $-4$  & \framebox{$-30$}
                    \end{tabular}
                \end{center}
            \item[$\Rightarrow$] Der eingerahmte Wert ist der gesuchte Funktionswert.
        \end{enumerate}
    \item \textit{Polynomdivision}

        Angenommen man hat durch Ausprobieren herausgefunden, dass $f(2)=0$ gilt und möchte jetzt das Polynom zerlegen.
        \begin{enumerate}
            \item Dazu schaut man sich das ausgefüllte \textsc{Horner}-Schema an:
                \begin{center}
                    \begin{tabular}{c|ccccc}
                        & $2$   & $1$   & $0$   & $-3$  & $-34$ \\\hline
                        $2$     &       & $4 $  & $10$  & $20$  & $34$   \\\hline
                        & $2$   & $5$   & $10$  & $17$  & $0$
                    \end{tabular}
                \end{center}
            \item Die Einträge der letzten Zeile bilden die Koeffizienten des Restpolynoms.
            \item[$\Rightarrow$] Insgesamt gilt also:
                \begin{align*}
                    f(x) = (x-2)(2x^3+5x^2+10x+17)
                \end{align*}
        \end{enumerate}
\end{itemize}


%---------------------------------------------Definition 1.4------------------------------------------------
%------------------------------------------------Eigenraum--------------------------------------------------
\begin{mydef}\label{eigenraum} \textit{Eigenraum}

    Sei $\alpha:V\to V$ ein Endomorphismus und $\lambda \in \K$ ein Eigenwert von $\alpha$, dann heißt
    \begin{align*}
        E(\lambda) = \left\{ v\in V \mid \alpha(v)=\lambda v \right\}
    \end{align*}
    der Eigenraum zum Eigenwert $\lambda$.
\end{mydef}

%-----------------------------------------------Bemerkung---------------------------------------------------
\textit{Bemerkung:}
\begin{enumerate}
    \item $E(\lambda)$ ist die Menge aller EV von $\lambda$ und $0$
    \item $E(\lambda) = \ker(\lambda\id_V-\alpha) )$
    \item $E(\lambda)\leq V$
\end{enumerate}

%---------------------------------------------Definition 1.5------------------------------------------------
%------------------------------------Diagonalmatrix, diagonalisierbar---------------------------------------
\begin{mydef} \textit{Diagonalmatrix, diagonalisierbar}

    Eine $n\times n$-Matrix $A$ heißt Diagonalmatrix, wenn $a_{ij}=0$ für $i\neq j; \ i,j=1,\ldots,n$ gilt.

    Man nennt die Matrix $A$ genau dann diagonalisierbar, wenn $A$ zu einer Diagonalmatrix ähnlich ist.

    Ein Endomorphismus $\alpha$ heißt genau dann diagonalisierbar, wenn eine Basis $B$ von $V$ existiert, bzgl. der die Abbildungsmatrix eine Diagonalmatrix ist.
\end{mydef}

%-----------------------------------------------Bemerkung---------------------------------------------------
\textit{Bemerkung:}

Der Endomorphismus $\alpha$ ist genau dann diagonalisierbar, wenn es eine Basis von $V$ gibt, welche aus Eigenvektoren von $\alpha$ besteht. Die verschiedenen Einträge auf der Diagonalen sind dann genau di Eigenwerte von $\alpha$.\\

%-----------------------------------------------Beispiel---------------------------------------------------
\textit{Beispiel:}

Gegeben sei $V=\R^3$ und ein Endomorphismus in $V$, welcher bzgl. der Standardbasis die folgende Abbildungsmatrix besitzt:
\begin{align*}
    A =
    \begin{pmatrix}
        4 & 0 & 2\\
        -6 & 1 & -4\\
        -6 & 0 & -3
    \end{pmatrix}
\end{align*}
Nach \ref{charpol} ist dann $\charpol A = \det(\lambda E-A)=(\lambda-1)^2\cdot \lambda$. Demnach sind nach \ref{eigenraum}
\begin{eqnarray*}
    E(\la) & = & \ker(\la E-A) = \ker
    \begin{pmatrix}
        -3 & 0 & -2\\
        6 & 0 & 4\\
        6 & 0 & 4
    \end{pmatrix}
    = \langle
    \begin{pmatrix}
        0\\1\\0
    \end{pmatrix}
    ,
    \begin{pmatrix}
        -2\\0\\3
    \end{pmatrix}
    \rangle\\
    E(\lb) & = & \ker (\lb E-A) = \ker
    \begin{pmatrix}
        -4 & 0 & -2\\
        6 & -1 & 4\\
        6 & 0 & 3
    \end{pmatrix}
    = \langle
    \begin{pmatrix}
        1\\-2\\-2
    \end{pmatrix}
    \rangle
\end{eqnarray*}
Die Eigenräume zu den Eigenwerten $\la=1$ und $\lb=0$.

Demnach ist die Matrix $A$ ähnlich zu der Matrix
\begin{align*}
    A' =
    \begin{pmatrix}
        1 & 0 & 0\\
        0 & 1 & 0\\
        0 & 0 & 0
    \end{pmatrix}
    \qquad \mbox{ mit } \qquad 
    A' = T^{-1} \cdot A \cdot T \qquad \mbox{ und } \qquad
    T =
    \begin{pmatrix}
        0 & -2 & 1\\
        1 & 0 & -2\\
        0 & 3 & -2
    \end{pmatrix}
\end{align*}
In der Transformationsmatrix $T$ stehen spaltenweise die Eigenvektoren.\\

%-----------------------------------------------Bemerkung---------------------------------------------------
\textit{Bemerkung:}

Die Summe der Eigenwerte einer Matrix (mit der Vielfachheit genommen) entspricht der Spur der Matrix. Sei $A\in \R^{n \times n}$ ähnlich zu der 
Matrix $D=T^{-1}AT$ dann ist $D$ eine Diagonalmatrix und es gilt
\begin{align*}
    Sp D = Sp(T^{-1}(AT))=Sp(T^{-1}TA)=Sp(A)
\end{align*}
%\end{bem}
%-----------------------------------------------Lemma 1.6---------------------------------------------------
%-----------------------------------------------------------------------------------------------------------
\begin{mylemma}\label{lemdiag}\ \\
    Seien $v_1,\ldots,v_s$ Eigenvektoren zu verschiedenen Eigenwerten $\lambda_1,\ldots,\lambda_s$ eines Endomorphimus $\alpha$ von $V$. Dann ist
    \begin{align*}
        \sum_{i=1}^s v_i \neq 0
    \end{align*}

    \textit{Beweis:}
    Induktion über $s$.
    
    Klar für $s=1$, da per Definition $0$ kein Eigenvektor ist.
    
    Sei $s>1$ und $w_i=(\lambda_s-\lambda_i)\cdot v_i$ für $i=1,\ldots,s-1$. Dann ist $w_i$ ebenfalls ein Eigenvektor zum Eigenwert $\lambda_i$, denn
    \begin{align*}
        \alpha(w_i)=\alpha((\lambda_s-\lambda_i)\cdot v_i)=(\lambda_s-\lambda_i)\cdot \alpha(v_i)= \lambda_i\cdot (\lambda_s-\lambda_i)\cdot v_i
    \end{align*}
    Nun folgt für den Induktionsschritt
    \begin{align*}
        (\lambda_s\id_V - \alpha)\cdot \sum_{i=1}^s v_i = \sum_{i=1}^s (\lambda_s\id_v - \alpha)\cdot v_i
        =\sum_{i=1}^{s-1} (\lambda_s-\lambda_i)\cdot v_i = \sum_{i=1}^{s-1} w_i \neq 0
    \end{align*}
\end{mylemma}

%----------------------------------------------------------------------------------------------------------
%-----------------------------------------------Satz 1.7---------------------------------------------------
%----------------------------------------------------------------------------------------------------------
\begin{mysatz}\ \\
    Sei $\alpha$ ein Endomorphismus von $V$ und $\lambda_1,\ldots,\lambda_s$ verschiedene Eigenwerte von $\alpha$ in $\K$. Für $j=1,\ldots,s$ sei $d_j=\dim E(\lambda_j)$. Dann gilt:
    \begin{enumerate}
        \item $\sum\limits_{j=1}^s d_j \leq n$
        \item $\sum\limits_{j=1}^s d_j = n \Leftrightarrow \alpha$ ist diagonalisierbar.
    \end{enumerate}

    \textit{Beweis:}
    \begin{enumerate}
        \item Sei für $j=1,\ldots,s: \left\{ b_{ji} \mid i=1,\ldots,d_j \right\}$ eine Basis von $E(\lambda_j)$. Dann ist die Vereinigung aller Basen $\left\{ b_{ji}|=1,\ldots,s; i=1,\ldots,d_j \right\}$ linear unabhängig, denn wenn
            \begin{align*}
                0 = \sum_{j,i} k_{ji}b_{ji} = \sum_{j=1}^s v_j, \quad \mbox{wobei} \quad v_j = \sum_j k_{ji} b_{ji}
            \end{align*}
            und (für festes $j$) nicht alle $k_{ji}=0$, dann ist $v_j \neq 0$, also ein Eigenvektor zu $\lambda_j$, was aber \ref{lemdiag} widerspricht. Dann folgt auch schon die Behauptung, denn
            \begin{align*}
                n & \geq \left|\left\{ b_{ji} \mid j = 1,\ldots,s;\ i=1,\ldots,d_j \right\} \right| = \sum_{j=1}^s d_j
            \end{align*}
        \item In diesem Fall existiert eine Basis $\left\{ b_{ji} \mid =1,\ldots,s;\ i=1,\ldots,d_j \right\}$, die nur Eigenvektoren enthält. Damit ist $\alpha$ diagonalisierbar. Sei umgekehrt $\alpha$ diagonalisierbar. Dann ist $\left\{ b_{ji} \mid i=1,\ldots,d_j \right\}$ eine Basis von $E(\lambda_j)$. Damit folgt die Behauptung.
    \end{enumerate}
\end{mysatz}

\newpage

%-----------------------------------------------Bemerkung---------------------------------------------------
\textit{Bemerkung:}

Sind die Eigenwerte eines Endomorphismus $\alpha$ alle verschieden, dann ist $\alpha$ diagonalisierbar.
\begin{center}
Die Umkehrung gilt im Allgemeinen nicht!
\end{center}
$\alpha$ ist diagonalisierbar, wenn das charpol vollständig in Linearfaktoren zerfällt und $\dim E(\lambda_i)=$ Vielfachheit der Nullstelle 
$\lambda_i$ in charpol $\alpha$ ist.

Falls $\alpha$ diagonalisierbar ist, so ist $V=\sum\limits_{i=1}^s E(\lambda_i) \quad \Rightarrow \quad E\left( \lambda_i \right) \cap E \left( \lambda_j \right) = \left\{ 0 \right\} \quad i \neq j$

%---------------------------------------------Definition 1.8------------------------------------------------
%----------------------------------------direkte Summe, Projektion------------------------------------------
\begin{mydef} \textit{Direkte Summe, Projektion}

    Sei $V$ ein Vektorraum über $\K$ und $U_1,\ldots,U_s$ Unterräume von $V$. Man nennt $V$ die direkte Summe der $U_i$ genau dann, wenn jeder Vektor $v$ aus $V$ sich eindeutig als Summe
    \begin{align*}
        v=\sum_{i=1}^s u_i \qquad \mbox{ mit } u_i\in U_i
    \end{align*}
    darstellen lässt (Bezeichnung: $V = U_1 \oplus \ldots \oplus U_s$).

    Wenn $V = \sum\limits_{i=1}^s \oplus U_i$, dann sei $\pi_i:V\rightarrow U_i$ definiert durch
    \begin{align*}
        \pi_i \left( \sum_{j=1}^s u_j \right) = u_i
    \end{align*}
    Man nennt $\pi_i$ die Projektion von $V$ auf $U_i$.
\end{mydef}

%-----------------------------------------------Bemerkung---------------------------------------------------
\textit{Bemerkung:}
\begin{enumerate}
    \item Der Durchschnitt zweier Unterräume einer direkten Summe ist stets trivial:
        \begin{align*}
            U_i \cap U_j = \left\{  0 \right\} \mbox{ für } i \neq j
        \end{align*}
    \item $\dim V=\sum\limits_{i=1}^s\dim U_i$
    \item Eine Summe aus Unterräumen heiß \textit{direkt}, wenn sich die Unterräume paarweise trivial schneiden.
    \item Die Summe von Eigenräumen zu verschiedenen Eigenwerten ist direkt.
    \item $\alpha$ ist genau dann diagonalisierbar, wenn $V$ die direkte Summe der Eigenräume $E(\lambda_i)$ zu den verschiedenen Eigenwerten $\lambda_i$ ist.
    \item $\pi_i$ ist ein Epimorphismus (surjektive lineare Abbildung).
\end{enumerate}

%----------------------------------------------------------------------------------------------------------
%-----------------------------------------------Satz 1.9---------------------------------------------------
%----------------------------------------------------------------------------------------------------------
\begin{mysatz}\ \\
    Seien $\alpha$ und $\beta$ diagonalisierbare Endomorphismen, welche miteinander kommutieren (d.h. $\alpha\beta = \beta\alpha$). Dann sind $\alpha$ und $\beta$ gemeinsam diagonalisierbar, d.h. es gibt eine Basis aus Eigenvektoren von $\alpha$ und $\beta$.

    \textit{Beweis:}
    $\alpha$ ist diagonalisierbar $\Rightarrow \lambda_1,\ldots,\lambda_s$ seien verschiedene Eigenwerte und $U_1,\ldots,U_s$ die dazugehörigen Eigenräume mit $V = U_1 \oplus \ldots \oplus U_s$.
    Sei $u_i\in U_i$, dann ist auch $\beta(u_i)=u_i'\in U_i$, weil
    \begin{align*}
        \alpha(u_i')= \alpha(\beta(u_i)) = \beta\alpha(u_i) = \beta \lambda_i u_i = \lambda_i u_i'
    \end{align*}
    Da $\beta$ diagonalisierbar ist, existiert eine Basis $\left\{ b_1,\ldots,b_n \right\}$ von $V$ aus Eigenvektoren von $\beta$, d.h. $\beta(b_j)=\mu_j\cdot b_j$. Jedes $b_j$ lässt sich eindeutig schreiben als $b_j=\sum\limits_{i=1}^s u_{ij}$ mit $u_{ij}\in U_i.$ Dann ist
    \begin{align*}
        \sum_{i=1}^s \beta(u_{ij}) = \beta(b_j)=\mu_j\cdot b_j = \sum_{i=1}^s\mu_j u_{ij}
    \end{align*}
    Da $u_{ij}' = \beta(u_{ij}) \in U_i$ für jeden $i$ und $\mu_ju_{ij} \in U_i$, folgt die Eindeutigkeit $\beta(u_{ij}) = \mu_j u_{ij}$. Demnach ist, wenn $u_{ij}\neq0$ gilt $u_{ij}$ ein Eigenwert zu $\alpha$ mit $\alpha(u_{ij}) = \lambda_i \cdot u_{ij}$, aber auch ein Eigenwert zu $\beta$ mit $\beta(u_{ij})=\mu_j\cdot u_{ij}$ und daher ist die Menge $\left\{ u_{ij} \mid i = 1,\ldots,s;\ j = 1,\ldots,n \right\}$ ein Erzeugendensystem, aus dem man eine Basis wählen kann.
\end{mysatz}

%-----------------------------------------------Beispiel---------------------------------------------------
\textit{Beispiel:}

Gegeben Sei $V=\R^2$ und die Endomorphismen $\alpha,\beta$, die bzgl. der Standardbasis die folgenen Abbildungsmatrizen besitzen.
\begin{align*}
    \alpha:A=\begin{pmatrix}1&0\\1&2\end{pmatrix} \qquad \mbox{und} \qquad \beta:B=\begin{pmatrix}3&0\\-4&-1\end{pmatrix} \qquad \mbox{mit} \qquad AB=BA=\begin{pmatrix}3&0\\-5&-2\end{pmatrix}
\end{align*}
Dann ist 
\begin{align*}
    E_{\alpha}(1) = \langle \begin{pmatrix}1\\-1\end{pmatrix} \rangle = E_{\beta}(3) \mbox{ und }	E_{\alpha}(2) = \langle \begin{pmatrix}0\\ 1\end{pmatrix} \rangle = E_{\beta}(-1)
\end{align*}
