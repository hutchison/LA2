\section{Orthogonale Räume und orthogonale Abbildungen}

%---------------------------------------Definition 4.1--------------------------------------------
%----------------------------------------Bilinearform---------------------------------------------

\begin{mydef} \label{Bilinearform} \textit{Bilinearform}

    Sei $V$ ein $\K$-VR, dann heißt die Abbildung $(\cdot , \cdot ):V \times V \mapsto \R$ Bilinearform genau dann, wenn gilt
    \begin{enumerate}
        \item $(v, w_1 + w_2) = (v, w_1) + (v, w_2) \ \forall v, w_1, w_2 \in V$

            $(v_1+v_2,w)=(v_1,w) + (v_2,w) \ \forall v_1,v_2,w \in V$
        \item $(k v,w) = k(v,w) = (v, k w) \ \forall v, w \in V, \forall k \in \K$

            Ist zusätzlich noch
        \item $(v,w) = (w,v) \ \forall v ,w \in V$
    \end{enumerate}
    erfüllt, dann heißt die Bilinearform symmetrisch.
\end{mydef}

%----------------------------------------Beispiel-------------------------------------------------
\textit{Beispiel:}

Sei $V = \K^n$
\begin{itemize}
    \item $(v,w) \mapsto v^T \cdot w \in \K$
\end{itemize}
Sei $V = \R^3$
\begin{itemize}
    \item $(v,w) \mapsto v_1 w_1 + v_2 w_2 - v_3 w_3$
\end{itemize}
Sei $V = \R[x]$
\begin{itemize}
    \item $(p,q) \mapsto \int\limits_{-1}^1 p(x) \cdot q(x) \ dx$
\end{itemize}
Sei $V=\F_2[x]$
\begin{itemize}
    \item $(p,q) \mapsto \sum\limits_{i \in I} a_i \cdot b_i$, wobei $p = \sum\limits_{i = 0}^r a_i \cdot x^i$ und $p = \sum\limits_{i = 0}^s b_i \cdot x^i$
\end{itemize}

%----------------------------------------Bemerkung-------------------------------------------------

\begin{mydef}\label{gramscheMatrix}

    Sei $\dim V = n$ und $\{ b_1, \ldots, b_n \}$ eine Basis von $V$ und $(\cdot ,\cdot )$ eine Bilinearform auf $V$.
    Desweiteren sei $(b_i, b_j)$ für alle $i, j \in \{1, \ldots, n \}$ gegeben.
    Für beliebige $v, w \in V$ ist dann
    \begin{align*}
        (v,w)   & = \left( \sum\limits_{i = 1}^n v_i b_i, \sum\limits_{j = 1}^n w_j b_j \right) = \sum\limits_{i = 1}^n \left(v_i b_i ,\sum\limits_{j = 1}^n w_j b_j \right) \\
        & = \sum\limits_{i, j} v_i(b_i, b_j) w_j
    \end{align*}
    Man nennt die Matrix
    \begin{align*}
        G =
        \begin{pmatrix}
            (b_1,b_1) & \cdots & (b_1,b_n) \\
            \vdots & & \vdots \\
            (b_n,b_1) & \cdots & (b_n,b_n)
        \end{pmatrix}
    \end{align*}
    die Gramsche Matrix der Bilineaform $(\cdot , \cdot )$ bzgl. der Basis B.
\end{mydef}
\textit{Bemerkung:}
\begin{enumerate}
    \item \label{gramscheMatrix-1} Offensichtlich ist
        \begin{align*}
            (v,w) & = v^T \cdot G \cdot w = (v_1, \ldots, v_n) \cdot G \cdot
            \begin{pmatrix}
                w_1\\ \vdots \\ w_n
            \end{pmatrix}
        \end{align*}
    \item \label{gramscheMatrix-2} Die Bilinearform $(\ \cdot \ , \ \cdot \ )$ ist genau dann symmetrisch, wenn $G$ symmetrisch ist.
    \item \label{gramscheMatrix-3} Sei $A$ eine beliebige Matrix aus $\K^{n \times n}$, dann ist durch $(v,w) := v^T \cdot A \cdot w$ eine Bilinearform auf $V = \K^n$ definiert.
\end{enumerate}

%---------------------------------------Definition 4.2--------------------------------------------
%-----------------------------------orthogonal, ausgeartet----------------------------------------

\begin{mydef} \label{orthogonal,ausgeartet} \textit{orthogonal, ausgeartet}

    Sei $(\cdot , \cdot )$ eine Bilinearform auf $V$.
    Die Vektoren $v,w\in V$ heißen orthogonal zueinander, wenn 
    \begin{align*}
        (v,w) & = 0
    \end{align*}
    gilt.
    Eine Bilinearform heißt genau dann nicht ausgeartet, wenn der Nullvektor der einzige Vektor ist, derzu allen Vektoren orthogonal ist.
    \begin{align*}
        (v,w) & = 0 \ \forall w \Rightarrow v = 0.
    \end{align*}
\end{mydef}

%------------------------------------------Lemma 4.3----------------------------------------------
%--------------------------------------TODO!!!----------------------------------------------------

\begin{mylemma}

    Sei $(\cdot , \cdot )$ eine Bilinearform mit der Gramschen Matrix $G$ bzgl. einer Basis $\{ b_1, \ldots, b_n \}$ von V.
    Dann ist $(\cdot , \cdot)$ genau dann nicht ausgeartet, wenn die Gramsche Matrix regulär ist.\par\medskip

    \textit{Beweis:}
    \begin{itemize}
        \item[,,$\Leftarrow$'']
        \item[,,$\Rightarrow$'']
    \end{itemize}
\end{mylemma}

%---------------------------------------Definition 4.4--------------------------------------------
%-----------------------------------orthogonales Komplement---------------------------------------

\begin{mydef} \label{orthogonalesKomplemen} \textit{orthogonales Komplement}

    Sei $X$ eine beliebige Teilmenge eines VR $V$, auf der die Bilinearform $(\cdot , \cdot)$ definiert ist.
    Man nennt die Menge
    \begin{align*}
        X^{\perp} & = \left\{ v \in V \mid (v,x) = 0 \quad \forall x \in X \right\}
    \end{align*}
    das orthogonale Komplement zu $X$. 
    Analog ist $^{\perp}X$ definiert.
\end{mydef}

%----------------------------------------------------------------------------------------------------------
%-----------------------------------------------Satz 4.5---------------------------------------------------
%----------------------------------------------------------------------------------------------------------

\begin{mysatz}\label{eigenschaftenorthKomplement}

    Sei $V$ ein VR mit der Bilinearform $(\cdot , \cdot)$ und $X \subseteq V$.
    Dann gilt
    \begin{enumerate}
        \item $X^{\perp}$ ist ein Unterraum von $V$.
        \item $X \subseteq Y$, dann ist $Y^{\perp} \subseteq X^{\perp}$
        \item $X^{\perp} = \left\langle X \right\rangle^{\perp}$
        \item \label{EigOrthKompl4} $X \subseteq X^{\perp\perp}$
    \end{enumerate}
\end{mysatz}
\textit{Beweis:} Übungsaufgabe

%------------------------------------------Lemma 4.6----------------------------------------------
%---------------------------------------Dimensionsformel------------------------------------------

\begin{mylemma} \label{dimensionsformelorthKomplement} \textit{Dimensionsformel}

    Sei $V$ ein $n-$dimensionaler VR, $(\cdot , \cdot )$ eine nicht ausgeartete Bilinearform auf $V$, so gilt für alle Unterräume $U$ von $V$
    \begin{align*}
        \dim U^{\perp} & = n - \dim U
    \end{align*}
    \textit{Beweis:}

    Sei $\{ b_1, \ldots, b_m, \ldots, b_n \}$ eine Basis von $V$ und $\{ b_1, \ldots, b_m \}$ eine Basis von $U$.
    Sei $x \in U^{\perp}$.
    Dann gilt für alle $j = 1, \ldots, m$, dass $(x, b_j) = 0$.
    Nun lässt sich $x$ als Linearkombination der Basiselemente von $V$ darstellen, etwa $\displaystyle{x=\sum_{i=1}^n k_i\cdot b_i}$.

    Dann folgt
    \begin{align*}
        0 & = (x, b_j) = \sum\limits_{i = 1}^n k_i \cdot (b_i, b_j) \ \forall j = 1, \ldots, m
    \end{align*}
    Das ist ein homogenes lineares Gleichungssystem mit $m$ Gleichungen und $n$ Unbekannten ($k_1, \ldots, k_n$).
    In der Koeffizientenmatrix $G'$ stehen die ersten $m$ Zeilen der Gramschen Matrix $G$.
    Dann hat $G'$ den Rang $n$, weil $G$ regulär ist.
    Also existiert eine $n-m$ parametrige Lösung des Gleichungssystem und daher ist $\dim U^{\perp} = \dim V - \dim U = n-m$.
\end{mylemma}

%----------------------------------------Bemerkung-------------------------------------------------
\textit{Bemerkung:}

Sei $\dim V = n$ und die Bilinearform $(\cdot , \cdot)$ nicht ausgeartet.
Wenn $U = U^{\perp \perp}$ (möglich nach \ref{eigenschaftenorthKomplement}.\ref{EigOrthKompl4}) dann ist $\dim U = \dim U^{\perp\perp}$

%---------------------------------------Definition 4.7--------------------------------------------
%---------------------------------------Skalarprodukte--------------------------------------------

\begin{mydef}\label{skalarprodukte} \textit{Skalarprodukt}

    Eine Bilinearform $\left\langle \cdot , \cdot \right\rangle$ auf dem Vektorraum $V$ heißt genau dann Skalaprodukt,
    wenn sie symmetrisch und positiv definit ist, d.h.
    \begin{align*}
        \forall v\in V : \ \left\langle v, v \right\rangle \geq 0 \text{ und } \left\langle v, v \right\rangle = 0 \Leftrightarrow v = 0
    \end{align*}
\end{mydef}

%----------------------------------------Bemerkung-------------------------------------------------
\textit{Bemerkung:}

Sei auf $V$ ein Skalarprodukt $\left\langle \cdot , \cdot \right\rangle$  definiert.
Dann gilt:
\begin{enumerate}
    \item $\left\langle \cdot ,  \cdot \right\rangle$ ist nicht ausgeartet.
    \item Sei $U$ ein beliebiger Unterraum von $V$. Dann ist $U \cap U^{\perp} = \{ 0 \} \Rightarrow V = U \oplus U^{\perp}$
\end{enumerate}

%----------------------------------------Beispiel-------------------------------------------------

\textit{Beispiel:}

Sei $V = \R^3$. Als Basis wählen wir die Standardbasis. Sei nun $\left\langle \cdot , \cdot \right\rangle$ definiert durch
\begin{align*}
    \left\langle
    \begin{pmatrix}
        x_1\\y_1\\z_1
    \end{pmatrix},
    \begin{pmatrix}
        x_2\\y_2\\z_2
    \end{pmatrix}
    \right\rangle = 3 x_1 x_2 + 2 y_1 y_2 + 2 z_1 z_2 - x_1 y_2 - y_1 z_2 - z_1 y_2 - x_2 y_1 - y_2 z_1 - x_1 z_2
\end{align*}
Nach \ref{gramscheMatrix} hat die Gramsche Matrix der Bilinearform die Gestalt:
\begin{align*}
    G =
    \begin{pmatrix}
        3 & -1 & -1\\
        -1 & 2 & -1\\
        -1 & -1 & 2
    \end{pmatrix}
\end{align*}
Da $G$ symmetrisch ist, ist auch die Bilinearform symmetrisch (vgl. \ref{gramscheMatrix}.\ref{gramscheMatrix-2}).

$(\cdot , \cdot)$ ist nicht ausgeartet, da $G$ regulär ist.

Bleibt also nur noch die positive Definitheit zu überprüfen.
Dazu betrachtet man $(x, x)$ mit $x = (x, y, z)^T$, was nach \ref{gramscheMatrix}.\ref{gramscheMatrix-1} gegeben ist durch
\begin{align*}
    \left\langle
    \begin{pmatrix}
        x\\y\\z
    \end{pmatrix},
    \begin{pmatrix}
        x\\y\\z
    \end{pmatrix}
    \right\rangle & = (x,y,z) \cdot G \cdot
    \begin{pmatrix}
        x\\y\\z
    \end{pmatrix} = 3x^2+2y^2+2z^2-2xy-2xz-2yz \\
    & = (x - y)^2 + (x - z)^2 + (y - z)^2 + x^2 \geq 0
\end{align*}
Und $\left\langle x, x \right\rangle = 0$ genau dann wenn $x = 0$, was man leicht überprüft.
Daher ist $\left\langle \cdot , \cdot \right\rangle$ ein Skalarprodukt.\\

%----------------------------------------Beispiel-------------------------------------------------
\textit{Beispiel:}

Sei $V = \mathcal{C}[a,b]$, der Raum der stetigen Funktionen auf dem Intervall $[a,b] \subset \R$.
Auf diesem Raum sei $\left\langle \cdot , \cdot \right\rangle$ definiert durch
\begin{align*}
    \left\langle f,g \right\rangle & := \int\limits_a^b f(x) \cdot g(x) \ dx
\end{align*}
Die Eigenschaften des Integrals übertragen sich auf die Eigenschaften von $\left\langle \cdot ,  \cdot \right\rangle$.
Damit ist $\left\langle \cdot , \cdot \right\rangle$ zumindestens schon mal eine symmetrische Bilineaform.
Für die positive Definitheit muss das Integral so gewählt werden, dass $\left\langle f, f \right\rangle > 0$ ist.
Wenn $\left\langle f, f \right\rangle = 0$, so folgt aufgrund der Stetigkeit von $f$, dass die Funktion $f = 0$ ist.

%---------------------------------------Definition 4.9--------------------------------------------
%--------------------------------------------Norm-------------------------------------------------

\begin{mydef} \label{norm} \textit{Norm}

    Sei $V$ ein VR über $\R$. Dann heißt eine Abbildung $\| \cdot \|: V \mapsto \R^{+}$ Norm, falls folgende Eigenschaften erfüllt sind:
    \begin{enumerate}
        \item $\| v \| \geq 0 \quad \Rightarrow \quad \| v \| = 0 \Leftrightarrow v = 0 \qquad \forall v \in V$
        \item $\| \lambda v\| = |\lambda| \cdot \| v \| \qquad \forall \lambda \in \R, \ \forall v \in V$
        \item $\| v + w \| \leq \| v \| + \| w \| \qquad \forall v,w\in V$.
    \end{enumerate}
\end{mydef}

%----------------------------------------Beispiel-------------------------------------------------

\textit{Beispiel:}
\begin{enumerate}
    \item   $V = \R^n$ mit $\| v \|_{\infty} =\max\limits_{i} | x_i |$, wobei $v = (x_1, \ldots, x_n)^T$
        $V = \R^n$ mit $\| v \|_{1} = | x_1| + \ldots + | x_n | = \sum\limits_{i = 1}^n | x_i |$
    \item $V = \R^{n \times n}$ mit $\| A \| = \max\limits_{v \in V \setminus\{ 0 \}} \frac{\| Av \|}{\| v \|} = \max\limits_{\| v\| = 1} \| Av \|$
\end{enumerate}

%----------------------------------------------------------------------------------------------------------
%-----------------------------------------------Satz 4.10--------------------------------------------------
%--------------------------------------------euklidische Norm----------------------------------------------

\begin{mysatz}\label{euklidischeNorm} \textit{Euklidische Norm}

    Durch $\| v \| = \sqrt{\left\langle v, v \right\rangle}$ wird in einem euklidischen Vektorraum $V$ (reellwertiger Vektorraum mit Skalaprodukt) eine Norm definiert.

    \textit{Beweis:}

    Die ersten beiden Normeigenschaften folgen unmittelbar. Die Dreiecksungleichung folgt mit Hilfe des nächsten Satzes.
\end{mysatz}

%----------------------------------------------------------------------------------------------------------
%-----------------------------------------------Satz 4.11--------------------------------------------------
%--------------------------------------Cauchy-Schwarzsche-Ungleichung--------------------------------------

\begin{mysatz} \label{CSU} \textit{Cauchy-Schwarzsche-Ungleichung}

    Sei $V$ ein euklidischer Vektorraum. Dann gilt
    \begin{align*}
        | \left\langle v,w \right\rangle | & \leq \| v \| \cdot \| w \|
        \intertext{oder in äquvalenter Weise}
        \left\langle v,w \right\rangle^2 & \leq \left\langle v,v \right\rangle \cdot \left\langle w,w \right\rangle \qquad \forall\ v,w\in V
    \end{align*}

    \textit{Beweis:}

    Wenn $w=0$ oder $v=0$, dann ist die Aussage trivial.

    Daher sei $v,w \neq 0$. Dann gilt für jedes $k \in \R$:
    \begin{align*}
        0 & \leq \left\langle v - kw, v - kw \right\rangle = \left\langle v,v \right\rangle + k^2 \left\langle w,w \right\rangle - 2 k \left\langle v,w \right\rangle
    \end{align*}
    Also insbesondere auch für $k = \frac{\left\langle v,w \right\rangle}{\left\langle w,w \right\rangle}$. Dann ist nämlich
    \begin{align*}
        0 & \leq \left\langle v,v \right\rangle + \frac{\left\langle v,w \right\rangle^2}{\left\langle w,w \right\rangle^2} \left\langle w,w \right\rangle - 2 \frac{\left\langle v,w \right\rangle}{\left\langle w,w \right\rangle} \left\langle v,w \right\rangle = \left\langle v,v \right\rangle - \frac{\left\langle v,w \right\rangle^2}{\left\langle w,w \right\rangle}
    \end{align*}
    und der Satz ist bewiesen.
\end{mysatz}

%----------------------------------------Bemerkung-------------------------------------------------

\textit{Bemerkung:}

Wegen \ref{CSU} gilt
\begin{align*}
    | \left\langle v,w \right\rangle | & \leq  \| v \| \cdot \| w \|
    \intertext{Umstellen liefert:}
    -1 & \leq \frac{ \left\langle v,w \right\rangle}{\| v \| \cdot \| w \|}  \leq  1
    \intertext{Daher existiert im Fall $v, w \neq 0$ genau ein $\varphi \in [0,\pi]$ mit}
    \cos(\varphi) & = \frac{ \left\langle v,w \right\rangle}{\| v \| \cdot \| w \|}
\end{align*}
Dies entspricht der Definition des Winkels zwischen $2$ Vektoren.\\

%----------------------------------------Beispiel-------------------------------------------------

\textit{Beispiel:}

Im $\R^2,\R^3$ mit dem Standardskalarprodukt entspricht $\| v \|$ der Länge und $\varphi$ dem Winkel zwischen den Vektoren $v,w$ im elementargeometrischen Sinn.

\begin{center}
    \begin{tikzpicture}
        \draw[>=latex,->] (0,0) -- (10,0);% Vektor v
        \draw (7,0) node[anchor=north] {$v$};
        \draw[>=latex,->] (0,0) -- (4,3);% Vektor w
        \draw (2,1.5) node[anchor=north west] {$w$};
        \draw (4,3) -- (6.4,4.8);% Strecke q
        \draw (6.4,4.8) -- (10,0);
        \draw (4,0) -- (4,3);
        \draw (4,0.3) arc (90:180:0.3);
        \draw (3.9,0.1) node {$\cdot$};
        \draw (6.16,4.62) arc(210:305:0.3);
        \draw (6.4,4.6) node {$\cdot$};
        \draw (1.8,0) arc (0:37:1.8);% Winkel alpha
        \draw (1.1,0.3) node {$\sphericalangle (v,w)$};
        \draw [gray,decorate,decoration={brace,amplitude=5pt},yshift=-3pt] (4,0)  -- (0,0) node [black,midway,below=4pt] {$p$};
        \draw [gray,decorate,decoration={brace,amplitude=5pt},yshift=2pt,xshift=-2pt] (0,0)  -- (6.4,4.8) node [black,midway,anchor=south east,yshift=1pt,xshift=-1pt] {$q$};
    \end{tikzpicture}
\end{center}
\begin{align*}
    \left.
    \begin{array}{c}
        \displaystyle{ \cos \sphericalangle (v,w) = \frac{p}{\| w \|} } \\
        \displaystyle{ \cos \sphericalangle (v,w) = \frac{q}{\| v \|} }
    \end{array}
    \right\}
    & \Rightarrow \left\langle v,w \right\rangle = q \cdot \| w \| = p \cdot \| v \| \qquad \text{ und damit }\\
    \cos \sphericalangle (v,w) & = \frac{\left\langle v,w \right\rangle}{\| v \| \cdot \| w \|}
\end{align*}
Das Skalarprodukt von $v$ und $w$ ist die Länge von $v$ mal der Länge der Projektion von $w$ auf $v$.
Desweiteren sind zwei Vektoren orthogonal zueinander, wenn das Skalarprodukt verschwindet.
Dies gilt für $\varphi = \pi/2$, aber auch, wenn einer der beiden Vektoren der Nullvektor ist.

%---------------------------------------Definition 4.12-------------------------------------------
%------------------------------------orthogonal,orthonormal---------------------------------------

\begin{mydef}\label{orthogonal,orthonormal}\textit{orthogonal, orthonormal}

    Eine Menge $\{ v_1, \ldots, v_n \}$ von Vektoren in einem euklidischen Vektorraum $V$ heißt genau dann orthogonal,
    wenn $\left\langle v_i,v_j \right\rangle = 0$ für alle $i = j \in \{ 1, \ldots, n \}$ und orthonormal, wenn $\left\langle v_i, v_j \right\rangle = \delta_{ij}$ ist.
\end{mydef}

%------------------------------------------Lemma 4.13---------------------------------------------
%-------------------------------------------------------------------------------------------------

\begin{mylemma}
    Eine orthogonale Menge, die den Nullvektor nicht enthält, ist linear unabhängig.

    \textit{Beweis:}

    Sei $\{ v_1, \ldots, v_n \}$ die orthogonale Menge.
    Die Vektoren $v_1, \ldots, v_n$ sind genau dann linear unabhängig, wenn aus $\sum\limits_{i = 1}^n k_i \cdot v_i = 0$ folgt, dass die Koeffizienten $k_i$ alle $0$ sind.
    Da die Vektoren orthogonal zueinander sind, folgt
    \begin{align*}
        0 & = \left\langle \sum\limits_{i = 1}^n k_i \cdot v_i , v_j \right\rangle = k_j \cdot \left\langle \underbrace{v_j, v_j}_{ \neq 0} \right\rangle
    \end{align*}
    woraus folgt, dass der Koeffizienten $k_j=0$ ist.
\end{mylemma}

%----------------------------------------Bemerkung-------------------------------------------------

\textit{Bemerkung:}

Orthonormale Mengen sind stets linear unabhängig, solange der Nullvektor nicht in der Menge enthalten sind.

%----------------------------------------------------------------------------------------------------------
%-----------------------------------------------Satz 4.14--------------------------------------------------
%--------------------------------------Orthonormalisierungsverfahren---------------------------------------

\begin{mysatz} \textit{Orthonormalisierungsverfahren nach 
    \textsc{Gram}\footnote{J\o rgen Pedersen Gram (* 27. Juni 1850 in Nustrup; $\dagger$ 29. April 1916 in Kopenhagen), dänischer Mathematiker}
    und
    \textsc{Schmidt}\footnote{Erhard Schmidt (* 13. Januar 1876 in Dorpat (heutiges Tartu, Estland); $\dagger$ 6. Dezember 1959 in Berlin), dt. Mathematiker}}

    Jeder endlich dimensionale euklidische Vektorraum besitzt eine Orthonormalbasis, kurz ON-Basis.

    \textit{Beweis:} Induktion über die Dimension $n$.

    \begin{itemize}
        \item $n = 1$: Sei $V=\left\langle b \right\rangle$. Dann ist $\{ e=\frac{b}{\| b \|}\}$ eine ON-Basis.\par \medskip
        \item $n - 1 \rightarrow n$: Sei dazu $\{ e_1, \ldots, e_{n-1} \}$ eine ON-Basis von $U \subset V$.
            Dann kann man mit einem geeigenten $b_n\in V$ die Menge $\{ e_1, \ldots, e_{n-1}, b_n \}$ zu einer Basis von $V$ ergänzen.
            Setze nun
            \begin{align*}
                e_n' & = \left( - \sum\limits_{i = 1}^{n-1} \frac{ \left\langle e_i, b_n \right\rangle }{ \left\langle e_i, e_i \right\rangle } \cdot e_i \right) + b_n
            \end{align*}
            Dann folgt für alle $k = 1, \ldots, n-1$
            \begin{align*}
                \left\langle e_n', e_k \right\rangle & = - \sum_{i = 1}^{n-1} \frac{ \left\langle e_i, b_n \right\rangle }{ \left\langle e_i,e_i \right\rangle } \cdot 
                \left\langle e_i,e_k \right\rangle + \left\langle b_n,e_k \right\rangle \\
                & = - \frac{ \left\langle e_k,b_n \right\rangle }{ \left\langle e_k,e_k \right\rangle } \cdot \left\langle e_k,e_k \right\rangle + \left\langle b_n,e_k \right\rangle = 0
            \end{align*}
            Damit hat man ein Element gefunden, welches senkrecht auf den anderen Basisvektoren steht. Normieren von $e_n'$ liefert dann die ON-Basis von $V$.
    \end{itemize}
\end{mysatz}

%----------------------------------------Beispiel-------------------------------------------------

\textit{Beispiel:}

Sei
\begin{align*}
    \{ b_1,b_2,b_3 \} =
    \left\{ \begin{pmatrix}
        1\\-1\\0\\1
    \end{pmatrix}
    ,
    \begin{pmatrix}
        1\\0\\1\\0
    \end{pmatrix}
    ,
    \begin{pmatrix}
        1\\-1\\0\\0
    \end{pmatrix}
    \right\}
\end{align*}
eine Basis von $U \subset \R^4$.
Dann ist $e_1' = b_1$ und $e_2' = \lambda e_1' + b_2$, daher folgt
\begin{align*}
    0 & = \left\langle e_2', e_1' \right\rangle = \lambda \left\langle e_1', e_1' \right\rangle + \left\langle b_2, e_1' \right\rangle \\
    \lambda & = - \frac{ \left\langle b_2, e_1' \right\rangle}{ \left\langle e_1', e_1' \right\rangle} = - \frac{1}{3}
\end{align*}
Also ist
\begin{align*}
    e_2' & = - \frac{1}{3} \cdot
    \begin{pmatrix}
        1\\-1\\0\\1
    \end{pmatrix}
    +
    \begin{pmatrix}
        1\\0\\1\\0
    \end{pmatrix}
    = \frac{1}{3} \cdot
    \begin{pmatrix}
        2\\1\\3\\-1
    \end{pmatrix}
\end{align*}
Für den dritten Basisvektor muss gelten gilt $e_3' = \lambda_1 e_1 + \lambda_2 e_2 + b_3$.
Daher folgt
\begin{align*}
    0 & = \left\langle e_3', e_1' \right\rangle = \lambda_1 \left\langle e_1', e_1' \right\rangle + \lambda_2 \left\langle e_2', e_1' \right\rangle + \left\langle b_3, e_1' \right\rangle\\
    \lambda_1 & = - \frac{ \left\langle b_3, e_1'\right\rangle }{ \left\langle e_1', e_1' \right\rangle } = - \frac{2}{3}
\end{align*}
Dann muss $\lambda_2 = - \frac{1}{5}$ sein, sodass
\begin{align*}
    e_3' = - \frac{2}{3} \cdot
    \begin{pmatrix}
        1\\-1\\0\\1
    \end{pmatrix}
    - \frac{1}{15} \cdot
    \begin{pmatrix}
        2\\1\\3\\-1
    \end{pmatrix}
    +
    \begin{pmatrix}
        1\\-1\\0\\0
    \end{pmatrix}
    = \frac{1}{5}
    \begin{pmatrix}
        1\\-2\\-1\\-3
    \end{pmatrix}
\end{align*}
Normieren von der $e_i'$ liefert dann die ON-Basis
\begin{align*}
    \left\{ \frac{1}{\sqrt{3}} \cdot
    \begin{pmatrix}
        1\\-1\\0\\1
    \end{pmatrix},
    \frac{1}{\sqrt{15}} \cdot
    \begin{pmatrix}
        2\\1\\3\\-1
    \end{pmatrix},
    \frac{1}{\sqrt{15}} \cdot
    \begin{pmatrix}
        1\\-2\\-1\\3
    \end{pmatrix}
    \right\}
\end{align*}

%%%%%%%%%%%%%%%%%%%%%%%%%%%%%%%%%%%%%%%%%%%%%%%%%%%%%%%%%%%%%%%%%
%  TODO Beispiel nochma durchrechnen, irgendwas stimmt da nicht %
%%%%%%%%%%%%%%%%%%%%%%%%%%%%%%%%%%%%%%%%%%%%%%%%%%%%%%%%%%%%%%%%%

%----------------------------------------Bemerkung-------------------------------------------------

\textit{Bemerkung:}

Sei $\{ e_1, \ldots, e_n \}$ eine ON-Basis von $V$ und $v \in V$, dann sind die Koordinaten $k_i$ aus $v = \sum k_i e_i$ gegeben durch
\begin{align*}
    k_i = \left\langle e_i,v \right\rangle
\end{align*}

%%%%% TODO hier fehlt noch einiges %%%%%

%---------------------------------------Definition 4.15-------------------------------------------
%-----------------------------------isometrische Abbildungen--------------------------------------

\begin{mydef}\textit{Isometrische Abbildungen}

    Eine lineare Abbildung $f$ eines euklidischen VR $V$ in einen euklidischen Vektorraum $W$ heißt genau dann isometrisch, wenn
    \begin{align*}
        \left\langle v,w \right\rangle_V = \left\langle f(v),f(w) \right\rangle_W \quad \forall\ v,w \in W
    \end{align*}
\end{mydef}

%----------------------------------------Bemerkung-------------------------------------------------

\textit{Bemerkung:}

Seien $V$ und $W$ euklidische Vektorräume.
Die lineare Abbildung $f:V\rightarrow W$ ist genau dann eine isometrische Abbildung, wenn sie invariant bzgl. der Normen der jeweiligen Vektorräume ist.
\begin{align*}
    \| f(v) \|_W = \| v \|_V \quad \forall\ v\in V
\end{align*}

%------------------------------------------Lemma 4.16---------------------------------------------
%-------------------------------------------------------------------------------------------------
\begin{mylemma}
    Isometrische Abbildungen sind injektiv.\medskip

    \textit{Beweis:}
    Sei $v \in \ker f$. Dann ist $fv = 0$ und daher
    \begin{align*}
        \left\langle fv,fv \right\rangle = 0 \Leftrightarrow \left\langle v,v \right\rangle = 0 \Leftrightarrow v = 0
    \end{align*}
    Damit liegt nur der Nullvektor im Kern der Abbildung und das Lemma ist bewiesen.
\end{mylemma}

%------------------------------------------Lemma 4.17---------------------------------------------
%-------------------------------------------------------------------------------------------------

\begin{mylemma}
    Sei $\{ b_1, \ldots, b_n \}$ eine ON-Basis von $V$.

    $f$ ist genau dann isometrisch, wenn $\{ f(b_1), \ldots, f(b_n) \}$ eine ON-Basis ist.\medskip

    \textit{Beweis:}
    \begin{itemize}
        \item[,,$\Rightarrow$''] Wenn $f$ isometrisch ist, dann folgt $\left\langle f(b_i),f(b_j) \right\rangle = \left\langle b_i, b_j \right\rangle = \delta_{ij}$.
        \item[,,$\Leftarrow$''] Zu zeigen ist, dass $f$ isometrisch ist, wenn $\{ f(b_1), \ldots, f(b_n) \}$ eine ON-Basis ist.
            Seien dazu $v_1 = \sum k_i \cdot b_i$ und $v_2 = \sum l_i \cdot b_i$.
            Dann ist
            \begin{align*}
                \left\langle v_1, v_2 \right\rangle = \left\langle \sum\limits_{i = 1}^n k_i f(b_i), \sum\limits_{i = 1}^n l_i f(b_i) \right\rangle = \sum k_i \cdot l_i = \left\langle v_1, v_2 \right\rangle
            \end{align*}
    \end{itemize}
\end{mylemma}

Im folgenden sei stets $V = W$. Dann wird $f$ zu einem isometrischen Endomorphismus (auch Automorphismus genannt).
Diese Abbildungen bekommen einen bestimmten Namen. Sie heißen orthogonale Abbildungen.\medskip

Die Menge aller orthogonalen Abbildungen ist eine Gruppe
\begin{align*}
    \mathcal{O}(n) \leq GL(n,\R) = \{ \text{reguläre Matrizen vom Typ $n\times n$} \}
\end{align*}

%----------------------------------------Bemerkung-------------------------------------------------

\textit{Bemerkung:}\medskip

Eine quadratische Matrix heißt genau dann orthogonal, wenn die Spaltenvektoren ein ON-System bilden.

%----------------------------------------------------------------------------------------------------------
%-----------------------------------------------Satz 4.18--------------------------------------------------
%----------------------------------------------------------------------------------------------------------

\begin{mysatz}\label{orthogonaleMatrix}
    Für Matrizen $A\in \R^{n \times n}$ sin folgende Aussagen äquivalent
    \begin{enumerate}
        \item $A$ ist orthogonal.
        \item Die Abbildung $f_A : \R^n \mapsto \R^n$ mit $f_a(v) = A \cdot v$ ist isometrisch.
        \item \label{oM-3} $A^T \cdot A = A \cdot A^T = E$.
        \item Wenn $A$ regulär, dann ist $A^{-1} = A^T$.
        \item Die Zeilenvektoren bilden ein ON-System.
    \end{enumerate}

    \textit{Beweis:}\medskip

    $1.\Leftrightarrow 2.$
\end{mysatz}

%----------------------------------------Bemerkung-------------------------------------------------

\textit{Bemerkung:}
\begin{enumerate}
    \item Sei $A$ eine orthogonale Matrix. Dann ist $\det A = \pm 1$, denn
        \begin{align*}
            1 = \det E = \det A^T \cdot A = \det A^T \cdot \det A = (\det A)^2.
        \end{align*}
    \item Zwei Matrizen $A, B \in \R^{n \times n}$ heißen orthogonal ähnlich, wenn eine orthogonale Matrx $T$ existiert mit $T^{-1} A T = B$.
\end{enumerate}


%-----------------------------------------Beispiel-------------------------------------------------

\textit{Beispiel:}\medskip

\textit{Beschreibung von orthogonalen Abbildungen in $\R^2$}

Sei $V = \R^2$ mit dem Standardskalarprodukt.
Gesucht ist eine Matrix $A \in \R^{2\times 2}$ die orthogonal ist. Nach \ref{orthogonaleMatrix}.\ref{oM-3} muss für $A$ gelten:
\begin{align*}
    \begin{pmatrix}
        a_{11} & a_{12}\\
        a_{21} & a_{22}
    \end{pmatrix}
    \cdot
    \begin{pmatrix}
        a_{11} & a_{21}\\
        a_{12} & a_{22}
    \end{pmatrix}
    =
    \begin{pmatrix}
        1 & 0\\
        0 & 1
    \end{pmatrix}
\end{align*}
Daher muss $a_{11}^2 + a_{12}^2 = 1 = a_{22}^2 + a_{21}^2$ und $a_{11} a_{21} + a_{12} a_{22} = 0$ gelten.
Gleichzeitig müssen die Spalten der Matrix aber auch ein ON-System bilden.
Daher gilt noch $a_{11}^2 + a_{21}^2 = 1$.
Für diese Gleichungen existiert ein $\varphi \in [0 , 2 \pi]$ mit $a_{11} = \cos(\varphi), a_{21} = \sin(\varphi)$.\medskip

Dann sind
\begin{align*}
    D_{\varphi} =
    \begin{pmatrix}
        \cos(\varphi) & - \sin(\varphi)\\
        \sin(\varphi) & \cos(\varphi)
    \end{pmatrix}
    \intertext{oder}
    S_{\varphi} =
    \begin{pmatrix}
        \cos(\varphi) & \sin(\varphi)\\
        \sin(\varphi) & - \cos(\varphi)
    \end{pmatrix}
\end{align*}
die beiden orthogonalen Matrizen in $\R^2$.
Dabei hat $D_{\varphi}$ die Determinante $1$ und stellt eine Drehung um den Winkel $\varphi$ dar.
Die Eigenwerte von $D_{\varphi}$ sind $\lambda_{1,2} = \cos(\varphi) \pm \sqrt{\cos(\varphi)^2 - 1}$.
Diese sind demnach nur dann reell, wenn $\varphi$ den Wert $0$ oder $\pi$ annimmt.\medskip

Dagegen hat $S_{\varphi}$ die Determinante $-1$ und das charakteristische Polynom:
$\charpol S_{\varphi} = \lambda^2 - \cos(\varphi)^2 - \sin(\varphi)^2 = \lambda^2 - 1$ mit den Eigenwerten $\pm 1$.

$S_{\varphi}$ stellt eine Spiegelung der Richtung des Eigenvektors dar.\\


%----------------------------------------Bemerkung-------------------------------------------------

\textit{Bemerkung:}\medskip

Sei $f$ eine orthogonale Abbildung. Dann nehmen die Eigenwerte von $f$, falls diese existieren, den Wert $\pm 1$ an, denn
\begin{align*}
    \| v \| = \| f(v) \| = \| \lambda \cdot v\| = | \lambda | \cdot \| v \|
\end{align*}

%----------------------------------------------------------------------------------------------------------
%-----------------------------------------------Satz 4.19--------------------------------------------------
%----------------------------------------------------------------------------------------------------------

\begin{mysatz}
    Sei $f$ eine orthogonale Abbildung des $n$-dim. Vektorraums $V$ in sich.
    Dann existiert eine ON-Basis von $V$, bzgl der $f$ die Darstellungsmatrix
    \begin{align*}
        \begin{pmatrix}
            1 & 0 & & & \cdots &\cdots & & & 0 \\
            0 & \ddots & & & & & & & \\
            & & 1 & & & & & & \\
            & & & -1 & & & & & \vdots \\
            \vdots & & & & \ddots & & &  &\vdots \\
            \vdots & & & & & -1 & & & \\
            & & & & & & A_1 & & \\
            & & & & & & & \ddots & 0 \\
            0 & & & \cdots & \cdots & & & 0 & A_1
        \end{pmatrix}
    \end{align*}
    besitzt.
    \textit{Beweis:} \medskip

    Induktion über $n$
\end{mysatz}
