\documentclass[%
a4paper,
%empty,         % keine Seitenzahlen
%a5paper,       % alle weiteren Papierformat einstellbar
10pt,           % Schriftgröße (12pt, 11pt (Standard))
%leqno,         % Nummerierung von Gleichungen links
%fleqn,         % Ausgabe von Gleichungen linksbündig
%twoside,
]
{scrartcl}

%% Deutsche Anpassungen 
\usepackage[ngerman]{babel}
\usepackage[T1]{fontenc}
\usepackage[sc]{mathpazo}
\linespread{1.05}
\usepackage[utf8]{inputenc}

%obligatorischer Mathekram:
\usepackage{amssymb,amstext,dsfont,trsym,pifont}
\usepackage[sumlimits]{amsmath}
\usepackage{eulervm}

%nützlicher Mathekram:
\newcommand{\R}{\mathbb{R}}
\newcommand{\C}{\mathbb{C}}
\newcommand{\K}{\mathbb{K}}
\newcommand{\Z}{\mathbb{Z}}
\newcommand{\N}{\mathbb{N}}
\newcommand{\F}{\mathbb{F}}
\newcommand{\off}{\text{off}\,}
\newcommand{\rg}{\text{rg}\,}
\newcommand{\Sp}{\text{Sp}\,}
\newcommand{\charpol}{\text{charpol}\,}
\newcommand{\id}{\mbox{ id}}
\newcommand{\TV}{\text{TV}}
\newcommand{\End}{\text{End}}
\newcommand{\IM}{\text{Im}\,}
\newcommand{\ora}{\overrightarrow}
%Borners Kommandos:
\newcommand{\la}{\lambda_1}
\newcommand{\lb}{\lambda_2}

%nützliche Extras:
\usepackage{array,
hhline,
longtable,
tabularx,
enumerate,
enumitem,
hyperref,
color,
xcolor,
setspace,
booktabs,
%cite,
caption,
%lineno,
%lastpage,
algorithmic,
algorithm,
ulem,
stmaryrd,
}

%schreibt "Algorithmus" statt "Algorithm":
%\floatname{algorithm}{Algorithmus}

%\usepackage{arydshln}

\usepackage[amsmath,thmmarks,thref]{ntheorem}

% meine Theoremdefinitionen:
% +------------------------+

% Definitionen:
\theoremstyle{plain}
\theoremheaderfont{\bfseries}
\theorembodyfont{}
\theoremseparator{\ }
%\theoremprework{\hfill \rule{0.5\textwidth}{1pt} \hspace*{0.25\textwidth} }
%\theorempostwork{\rule}
%\theoremindent2ex
\newtheorem{mydef}{Definition}[section]

%Sätze:
\theoremstyle{plain}
\theoremheaderfont{\bfseries}
\theorembodyfont{}
\theoremsymbol{$\Box$}
\newtheorem{mysatz}[mydef]{Satz}

%Lemmata:
\theoremstyle{plain}
\theoremheaderfont{\bfseries}
\theorembodyfont{}
\theoremsymbol{$\Box$}
\newtheorem{mylemma}[mydef]{Lemma}

%Bemerkungen:
\theoremstyle{plain}
\theoremheaderfont{\itshape}
\theorembodyfont{}
\theoremsymbol{}
\newtheorem{mybem}[mydef]{Bemerkung}

%Beispiele:
\theoremstyle{plain}
\theoremheaderfont{\itshape}
\theorembodyfont{}
\theoremsymbol{}
\newtheorem*{mybsp}[mydef]{Beispiel}

% Randverwaltung (entweder geometry oder =fullpage=)
%\usepackage[left=1cm,right=1cm,top=1cm,bottom=1cm,includeheadfoot]{geometry}
\usepackage[%cm,
%headings,
]{fullpage}

% die fancy-Header:
%\usepackage{fancyhdr}
%\pagestyle{fancy}
%\fancyhf{}
%
%\fancyhead{}
%\fancyfoot{}
%\fancyhead[L]{}
%\fancyhead[C]{}
%\fancyhead[R]{\nouppercase \leftmark}
%\fancyfoot[L]{}
%\fancyfoot[C]{\thepage}
%\fancyfoot[OR]{\thepage}
%\fancyfoot[LE]{\thepage}
%Linie oben/unten
%\renewcommand{\headrulewidth}{0.0pt}
%\renewcommand{\footrulewidth}{0.0pt}

%kein Einrücken der Paragraphen
\parindent 0pt

%% Packages für Grafiken & Abbildungen
%\usepackage{graphicx}
%\usepackage{subfig}    %%Teilabbildungen in einer Abbildung
\usepackage{tikz}      %%TeX ist kein Zeichenprogramm
\usetikzlibrary{arrows,decorations.pathreplacing}
%\usepackage[all]{xy}
%\usepackage{pst-all}

\makeindex

\begin{document}

\title{
\includegraphics[scale=0.3]{bilder/uni-logo.png}\\
\textsc{Mitschriften Lineare Algebra II}
\\ Universität Rostock}

\author{Dr. K. \textsc{Mahrhold}\\ Sommersemester 2010}

\begin{titlepage}
\maketitle
\end{titlepage}

\tableofcontents

\section{Eigenwerte und Eigenvektoren}

Sei $V$ ein $n$-dimensionaler Vektorraum über dem Körper $\K$.

%---------------------------------------------Definition 1.3------------------------------------------------
%----------------------------------------Eigenwert, Eigenvektor---------------------------------------------
\begin{mydef} \textit{Eigenwert und Eigenvektor}

    Sei $\alpha$ ein Endomorphismus des $\K$-VR V und $\lambda\in \K$. $\lambda$ heißt genau dann Eigenwert von $\alpha$, wenn ein Vektor $v\in 
    V\backslash\lbrace0\rbrace$ existiert mit
    \begin{align*}
        \alpha(v) = \lambda v
    \end{align*}
    Der Vektor $v$ heißt dann Eigenvektor zum Eigenwert $\lambda$.
\end{mydef}

%-----------------------------------------------Beispiel---------------------------------------------------
\textit{Beispiel:}

Sei $V=\R^2$ und der Endomorphismus $\alpha:(x,y)\mapsto (x+y,x+y)$.
\begin{align*}
    \Rightarrow \quad & \alpha(x,y)=(x+y,x+y) = \lambda(x,y)=(\lambda x,\lambda y)\\
    & \begin{array}{cccrccc}
        \lambda=0 & \Rightarrow & y= & -x & \Rightarrow & v= &\langle \begin{pmatrix}1\\-1\end{pmatrix} \rangle \\
        \lambda=2 & \Rightarrow & y= &  x & \Rightarrow & v= &\langle \begin{pmatrix}1\\ 1\end{pmatrix} \rangle
    \end{array}
\end{align*}
Bezüglich der Standardbasis des $\R^2$ hat $\alpha$ die Abbildungsmatrix $A=\begin{pmatrix}1&1\\1&1\end{pmatrix}$, bzgl. der Basis $\langle(1,1),(1,-1)\rangle$ die
Abbildungsmatrix $D=\begin{pmatrix}2&0\\0&0\end{pmatrix}$. Dann nennt man $A$ ähnlich zu $D$, denn $A'=T^{-1}AT$, wobei $T$ die Matrix des Basiswechsels ist.\\

Bonusfrage: wie sieht $A^{n}$ aus?

Dazu betrachte man $A$ bzgl. der Eigenvektorbasis $\left\langle (1,1),(1,-1) \right\rangle$:
\begin{align*}
    A' & =
    \begin{pmatrix}
        2 & 0\\
        0 & 0
    \end{pmatrix}\\
    A' & = T^{-1} A T\\
    \langle (1,0),(0,1) \rangle & \overset{T}{\to} \langle (1,1),(1,-1) \rangle\\
    T & = 
    \begin{pmatrix}
        1 & 1\\
        1 & -1
    \end{pmatrix}
    \intertext{Und nun wird die $n$-te Potenz berechnet:}
    A & = T A' T^{-1}\\
    \uline{A^n} & = (T A' T^{-1})(T A' T^{-1}) \ldots (T A' T^{-1}) = \uuline{T {A'}^n T^{-1}}\\
    & = -\frac{1}{2}
    \begin{pmatrix}
        1 & 1\\
        1 & -1
    \end{pmatrix}
    \begin{pmatrix}
        2^n & 0\\
        0 & 0
    \end{pmatrix}
    \begin{pmatrix}
        -1 & -1\\
        -1 & 1
    \end{pmatrix}
    =
    -\frac{1}{2}
    \begin{pmatrix}
        2^n & 0\\
        2^n & 0
    \end{pmatrix}
    \begin{pmatrix}
        -1 & -1\\
        -1 & 1
    \end{pmatrix}\\
    & =
    -\frac{1}{2}
    \begin{pmatrix}
        -2^n & -2^n\\
        -2^n & -2^n
    \end{pmatrix}
    =
    \uuline{
    \begin{pmatrix}
        2^{n-1} & 2^{n-1}\\
        2^{n-1} & 2^{n-1}
    \end{pmatrix}
    }
\end{align*}

%\newpage

%----------------------------------------------------------------------------------------------------------
%-----------------------------------------------Satz 1.2---------------------------------------------------
%----------------------------------------------------------------------------------------------------------
\begin{mysatz}\ \\
    Sei $\alpha$ ein Endomorphismus von $V$, dann sind folgende Aussagen äqiuvalent
    \begin{enumerate}
        \item $\lambda$ ist Eigenwert von $\alpha$
        \item $\det(\lambda \id_V-\alpha)=0$
        \item rg$(\lambda \id_V-\alpha)<n$
        \item $\ker(\lambda \id_V-\alpha) \neq \left\{ 0 \right\}$
    \end{enumerate}
\end{mysatz}

%---------------------------------------------Definition 1.3------------------------------------------------
%----------------------------------------charakteristisches Polynom-----------------------------------------
\begin{mydef}\label{charpol} \textit{charakteristisches Polynom}
    \begin{enumerate}
        \item Sei $A$ eine $n\times n$-Matrix. Das charakteristische Polynom von $A$ (charpol $A$) ist das Polynom
            \begin{align*}
                \det(\lambda\cdot E-A)\in \K[\lambda].
            \end{align*}
        \item Sei $\alpha$ ein Endomorphismus von $V$. Das charakteristische Polynom von $\alpha$ ist charpol $\alpha=\charpol A$, wobei $A$ die Darstellungsmatrix von $\alpha$ bzgl. einer Basis von $V$ ist.
    \end{enumerate}
\end{mydef}

%-----------------------------------------------Bemerkung---------------------------------------------------
\textit{Bemerkung:}
\begin{enumerate}
    \item Die Eigenwerte von $\alpha$ sind die Nullstellen des charakteristischen Polynoms.
    \item Ähnliche Matrizen haben dasselbe charpol
        \begin{eqnarray*}
            \charpol A' &=& \det(\lambda E-T^{-1}AT) = \det(T^{-1}\lambda ET-T^{-1}AT)=\det(T^{-1}(\lambda E-A)T)\\
            &=& \det(T^{-1}) \cdot \det(\lambda E-A) \cdot \det(T) = \det(\lambda E-A)\\
            &=& \charpol A
        \end{eqnarray*}
\end{enumerate}

\textit{Beispiel:} $V = \R^2$
\begin{align*}
    A & =
    \begin{pmatrix}
        1 & -1\\
        1 & 1
    \end{pmatrix}\\
    \charpol A & = \det
    \begin{pmatrix}
        \lambda - 1 & -1\\
        -1 & \lambda - 1
    \end{pmatrix}
    = \lambda^2 -2 \lambda + 2\\
    \lambda_{1,2} & = 1 \pm \sqrt{1-2} \notin \R\\
    & \Rightarrow \text{keine Eigenwerte}
\end{align*}

\textit{Anmerkung:}

Für die Fälle $n=2$ und $n=3$ existieren Formeln für das charakteristische Polynom:
\begin{align*}
    p(\lambda) & = \lambda^2 - \left( \Sp A \right) \lambda  + \det A & (n=2)\\
    p(\lambda) & = -\lambda^3 + \left( \Sp A \right) \lambda^2 - k_1 + \det A & (n=3)\\
    k_1 \text{ ist für } A & = 
    \begin{pmatrix}
        a & b & c\\
        d & e & f\\
        g & h & j
    \end{pmatrix}
    \text{ definiert als }
    k_1 = 
    \begin{vmatrix}
        a & b\\
        d & e
    \end{vmatrix}
    +
    \begin{vmatrix}
        a & c\\
        g & j
    \end{vmatrix}
    +
    \begin{vmatrix}
        e & f\\
        h & j
    \end{vmatrix}
\end{align*}

%\newpage

\textit{Anmerkung:}
\textsc{Horner}-Schema\footnote{William George Horner, * 1786 in Bristol; $\dagger$ 22. September 1837 in Bath, englischer Mathematiker}
\begin{itemize}
    \item[Vor.:] Polynom $f(x) = a_n x^n + a_{n-1} x^{n-1} + \ldots + a_1 x + a_0$
    \item[Bsp.:] $f(x) = 2x^4 + x^3 -3x -34$
\end{itemize}
\begin{itemize}
    \item \textit{Berechne} $f(-1)$!

        Für das \textsc{Horner}-Schema benötigt man eine kleine Tabelle mit 3 Zeilen und $n+2$ Spalten, wobei $n$ der Grad des Polynoms ist.
        \begin{enumerate}
            \item In die Kopfzeile schreibt man alle Koeffizienten des Polynoms; in das 1. Feld der 2. Zeile den zu berechnenden Wert:

                \hspace{3cm}
                    \begin{tabular}{c|ccccc}
                        & $2$   & $1$   & $0$   & $-3$  & $-34$ \\\hline
                        $-1$    &       &       &       &       &       \\\hline
                        \phantom{$-1$}
                    \end{tabular}
            \item Das 2. Feld der 2. Zeile bleibt leer. Darunter jedoch (2. Feld, 3. Zeile) schreibt man den Wert, der in der 1. Zeile steht (also $2$).

                \hspace{3cm}
                    \begin{tabular}{c|ccccc}
                        & $2$   & $1$   & $0$   & $-3$  & $-34$ \\\hline
                        $-1$    &       &       &       &       &       \\\hline
                        & $2$
                    \end{tabular}
                \hspace{2cm}
                $\downarrow$
            \item Nun multipliziert man den eben eingetragenen Wert mit der Zahl ganz links (also dem zu berechnenden Argument) und trägt ihn das Feld rechts darüber ein.

                \hspace{3cm}
                    \begin{tabular}{c|ccccc}
                        & $2$   & $1$   & $0$   & $-3$  & $-34$ \\\hline
                        $-1$    &       & $-2$  &       &       &       \\\hline
                        & $2$
                    \end{tabular}
                \hspace{1.3cm}
                $\cdot (-1) \quad \nearrow$
            \item Jetzt addiert man die Werte der 3. Spalte und trägt das Ergebnis in das unterste Feld ein.

                \hspace{3cm}
                    \begin{tabular}{c|ccccc}
                        & $2$   & $1$   & $0$   & $-3$  & $-34$ \\\hline
                        $-1$    &       & $-2$  &       &       &       \\\hline
                        & $2$   & $-1$
                    \end{tabular}
                \hspace{1.5cm}
                $+ \quad \downarrow$
            \item Dieses Prinzip führt man bis zum Ende durch (mit dem Wert ganz links multiplizieren; in das Feld rechts darüber eintragen; summieren).
                \begin{center}
                    \begin{tabular}{c|ccccc}
                        & $2$   & $1$   & $0$   & $-3$  & $-34$ \\\hline
                        $-1$    &       & $-2$  & $1$   &       &       \\\hline
                        & $2$   & $-1$
                    \end{tabular}
                    $\Rightarrow \ldots \Rightarrow$
                    \begin{tabular}{c|ccccc}
                        & $2$   & $1$   & $0$   & $-3$  & $-34$ \\\hline
                        $-1$    &       & $-2$  & $1$   & $-1$  & $4$   \\\hline
                        & $2$   & $-1$  & $1$   & $-4$  & \framebox{$-30$}
                    \end{tabular}
                \end{center}
            \item[$\Rightarrow$] Der eingerahmte Wert ist der gesuchte Funktionswert.
        \end{enumerate}
    \item \textit{Polynomdivision}

        Angenommen man hat durch Ausprobieren herausgefunden, dass $f(2)=0$ gilt und möchte jetzt das Polynom zerlegen.
        \begin{enumerate}
            \item Dazu schaut man sich das ausgefüllte \textsc{Horner}-Schema an:
                \begin{center}
                    \begin{tabular}{c|ccccc}
                        & $2$   & $1$   & $0$   & $-3$  & $-34$ \\\hline
                        $2$     &       & $4 $  & $10$  & $20$  & $34$   \\\hline
                        & $2$   & $5$   & $10$  & $17$  & $0$
                    \end{tabular}
                \end{center}
            \item Die Einträge der letzten Zeile bilden die Koeffizienten des Restpolynoms.
            \item[$\Rightarrow$] Insgesamt gilt also:
                \begin{align*}
                    f(x) = (x-2)(2x^3+5x^2+10x+17)
                \end{align*}
        \end{enumerate}
\end{itemize}


%---------------------------------------------Definition 1.4------------------------------------------------
%------------------------------------------------Eigenraum--------------------------------------------------
\begin{mydef}\label{eigenraum} \textit{Eigenraum}

    Sei $\alpha:V\to V$ ein Endomorphismus und $\lambda \in \K$ ein Eigenwert von $\alpha$, dann heißt
    \begin{align*}
        E(\lambda) = \left\{ v\in V \mid \alpha(v)=\lambda v \right\}
    \end{align*}
    der Eigenraum zum Eigenwert $\lambda$.
\end{mydef}

%-----------------------------------------------Bemerkung---------------------------------------------------
\textit{Bemerkung:}
\begin{enumerate}
    \item $E(\lambda)$ ist die Menge aller EV von $\lambda$ und $0$
    \item $E(\lambda) = \ker(\lambda\id_V-\alpha) )$
    \item $E(\lambda)\leq V$
\end{enumerate}

%---------------------------------------------Definition 1.5------------------------------------------------
%------------------------------------Diagonalmatrix, diagonalisierbar---------------------------------------
\begin{mydef} \textit{Diagonalmatrix, diagonalisierbar}

    Eine $n\times n$-Matrix $A$ heißt Diagonalmatrix, wenn $a_{ij}=0$ für $i\neq j; \ i,j=1,\ldots,n$ gilt.

    Man nennt die Matrix $A$ genau dann diagonalisierbar, wenn $A$ zu einer Diagonalmatrix ähnlich ist.

    Ein Endomorphismus $\alpha$ heißt genau dann diagonalisierbar, wenn eine Basis $B$ von $V$ existiert, bzgl. der die Abbildungsmatrix eine Diagonalmatrix ist.
\end{mydef}

%-----------------------------------------------Bemerkung---------------------------------------------------
\textit{Bemerkung:}

Der Endomorphismus $\alpha$ ist genau dann diagonalisierbar, wenn es eine Basis von $V$ gibt, welche aus Eigenvektoren von $\alpha$ besteht. Die verschiedenen Einträge auf der Diagonalen sind dann genau di Eigenwerte von $\alpha$.\\

%-----------------------------------------------Beispiel---------------------------------------------------
\textit{Beispiel:}

Gegeben sei $V=\R^3$ und ein Endomorphismus in $V$, welcher bzgl. der Standardbasis die folgende Abbildungsmatrix besitzt:
\begin{align*}
    A =
    \begin{pmatrix}
        4 & 0 & 2\\
        -6 & 1 & -4\\
        -6 & 0 & -3
    \end{pmatrix}
\end{align*}
Nach \ref{charpol} ist dann $\charpol A = \det(\lambda E-A)=(\lambda-1)^2\cdot \lambda$. Demnach sind nach \ref{eigenraum}
\begin{eqnarray*}
    E(\la) & = & \ker(\la E-A) = \ker
    \begin{pmatrix}
        -3 & 0 & -2\\
        6 & 0 & 4\\
        6 & 0 & 4
    \end{pmatrix}
    = \langle
    \begin{pmatrix}
        0\\1\\0
    \end{pmatrix}
    ,
    \begin{pmatrix}
        -2\\0\\3
    \end{pmatrix}
    \rangle\\
    E(\lb) & = & \ker (\lb E-A) = \ker
    \begin{pmatrix}
        -4 & 0 & -2\\
        6 & -1 & 4\\
        6 & 0 & 3
    \end{pmatrix}
    = \langle
    \begin{pmatrix}
        1\\-2\\-2
    \end{pmatrix}
    \rangle
\end{eqnarray*}
Die Eigenräume zu den Eigenwerten $\la=1$ und $\lb=0$.

Demnach ist die Matrix $A$ ähnlich zu der Matrix
\begin{align*}
    A' =
    \begin{pmatrix}
        1 & 0 & 0\\
        0 & 1 & 0\\
        0 & 0 & 0
    \end{pmatrix}
    \qquad \mbox{ mit } \qquad 
    A' = T^{-1} \cdot A \cdot T \qquad \mbox{ und } \qquad
    T =
    \begin{pmatrix}
        0 & -2 & 1\\
        1 & 0 & -2\\
        0 & 3 & -2
    \end{pmatrix}
\end{align*}
In der Transformationsmatrix $T$ stehen spaltenweise die Eigenvektoren.\\

%-----------------------------------------------Bemerkung---------------------------------------------------
\textit{Bemerkung:}

Die Summe der Eigenwerte einer Matrix (mit der Vielfachheit genommen) entspricht der Spur der Matrix. Sei $A\in \R^{n \times n}$ ähnlich zu der 
Matrix $D=T^{-1}AT$ dann ist $D$ eine Diagonalmatrix und es gilt
\begin{align*}
    Sp D = Sp(T^{-1}(AT))=Sp(T^{-1}TA)=Sp(A)
\end{align*}
%\end{bem}
%-----------------------------------------------Lemma 1.6---------------------------------------------------
%-----------------------------------------------------------------------------------------------------------
\begin{mylemma}\label{lemdiag}\ \\
    Seien $v_1,\ldots,v_s$ Eigenvektoren zu verschiedenen Eigenwerten $\lambda_1,\ldots,\lambda_s$ eines Endomorphimus $\alpha$ von $V$. Dann ist
    \begin{align*}
        \sum_{i=1}^s v_i \neq 0
    \end{align*}

    \textit{Beweis:}
    Induktion über $s$.
    
    Klar für $s=1$, da per Definition $0$ kein Eigenvektor ist.
    
    Sei $s>1$ und $w_i=(\lambda_s-\lambda_i)\cdot v_i$ für $i=1,\ldots,s-1$. Dann ist $w_i$ ebenfalls ein Eigenvektor zum Eigenwert $\lambda_i$, denn
    \begin{align*}
        \alpha(w_i)=\alpha((\lambda_s-\lambda_i)\cdot v_i)=(\lambda_s-\lambda_i)\cdot \alpha(v_i)= \lambda_i\cdot (\lambda_s-\lambda_i)\cdot v_i
    \end{align*}
    Nun folgt für den Induktionsschritt
    \begin{align*}
        (\lambda_s\id_V - \alpha)\cdot \sum_{i=1}^s v_i = \sum_{i=1}^s (\lambda_s\id_v - \alpha)\cdot v_i
        =\sum_{i=1}^{s-1} (\lambda_s-\lambda_i)\cdot v_i = \sum_{i=1}^{s-1} w_i \neq 0
    \end{align*}
\end{mylemma}

%----------------------------------------------------------------------------------------------------------
%-----------------------------------------------Satz 1.7---------------------------------------------------
%----------------------------------------------------------------------------------------------------------
\begin{mysatz}\ \\
    Sei $\alpha$ ein Endomorphismus von $V$ und $\lambda_1,\ldots,\lambda_s$ verschiedene Eigenwerte von $\alpha$ in $\K$. Für $j=1,\ldots,s$ sei $d_j=\dim E(\lambda_j)$. Dann gilt:
    \begin{enumerate}
        \item $\sum\limits_{j=1}^s d_j \leq n$
        \item $\sum\limits_{j=1}^s d_j = n \Leftrightarrow \alpha$ ist diagonalisierbar.
    \end{enumerate}

    \textit{Beweis:}
    \begin{enumerate}
        \item Sei für $j=1,\ldots,s: \left\{ b_{ji} \mid i=1,\ldots,d_j \right\}$ eine Basis von $E(\lambda_j)$. Dann ist die Vereinigung aller Basen $\left\{ b_{ji}|=1,\ldots,s; i=1,\ldots,d_j \right\}$ linear unabhängig, denn wenn
            \begin{align*}
                0 = \sum_{j,i} k_{ji}b_{ji} = \sum_{j=1}^s v_j, \quad \mbox{wobei} \quad v_j = \sum_j k_{ji} b_{ji}
            \end{align*}
            und (für festes $j$) nicht alle $k_{ji}=0$, dann ist $v_j \neq 0$, also ein Eigenvektor zu $\lambda_j$, was aber \ref{lemdiag} widerspricht. Dann folgt auch schon die Behauptung, denn
            \begin{align*}
                n & \geq \left|\left\{ b_{ji} \mid j = 1,\ldots,s;\ i=1,\ldots,d_j \right\} \right| = \sum_{j=1}^s d_j
            \end{align*}
        \item In diesem Fall existiert eine Basis $\left\{ b_{ji} \mid =1,\ldots,s;\ i=1,\ldots,d_j \right\}$, die nur Eigenvektoren enthält. Damit ist $\alpha$ diagonalisierbar. Sei umgekehrt $\alpha$ diagonalisierbar. Dann ist $\left\{ b_{ji} \mid i=1,\ldots,d_j \right\}$ eine Basis von $E(\lambda_j)$. Damit folgt die Behauptung.
    \end{enumerate}
\end{mysatz}

%\newpage

%-----------------------------------------------Bemerkung---------------------------------------------------
\textit{Bemerkung:}

Sind die Eigenwerte eines Endomorphismus $\alpha$ alle verschieden, dann ist $\alpha$ diagonalisierbar.
\begin{center}
Die Umkehrung gilt im Allgemeinen nicht!
\end{center}
$\alpha$ ist diagonalisierbar, wenn das charpol vollständig in Linearfaktoren zerfällt und $\dim E(\lambda_i)=$ Vielfachheit der Nullstelle 
$\lambda_i$ in charpol $\alpha$ ist.

Falls $\alpha$ diagonalisierbar ist, so ist $V=\sum\limits_{i=1}^s E(\lambda_i) \quad \Rightarrow \quad E\left( \lambda_i \right) \cap E \left( \lambda_j \right) = \left\{ 0 \right\} \quad i \neq j$

%---------------------------------------------Definition 1.8------------------------------------------------
%----------------------------------------direkte Summe, Projektion------------------------------------------
\begin{mydef} \textit{Direkte Summe, Projektion}

    Sei $V$ ein Vektorraum über $\K$ und $U_1,\ldots,U_s$ Unterräume von $V$. Man nennt $V$ die direkte Summe der $U_i$ genau dann, wenn jeder Vektor $v$ aus $V$ sich eindeutig als Summe
    \begin{align*}
        v=\sum_{i=1}^s u_i \qquad \mbox{ mit } u_i\in U_i
    \end{align*}
    darstellen lässt (Bezeichnung: $V = U_1 \oplus \ldots \oplus U_s$).

    Wenn $V = \sum\limits_{i=1}^s \oplus U_i$, dann sei $\pi_i:V\rightarrow U_i$ definiert durch
    \begin{align*}
        \pi_i \left( \sum_{j=1}^s u_j \right) = u_i
    \end{align*}
    Man nennt $\pi_i$ die Projektion von $V$ auf $U_i$.
\end{mydef}

%-----------------------------------------------Bemerkung---------------------------------------------------
\textit{Bemerkung:}
\begin{enumerate}
    \item Der Durchschnitt zweier Unterräume einer direkten Summe ist stets trivial:
        \begin{align*}
            U_i \cap U_j = \left\{  0 \right\} \mbox{ für } i \neq j
        \end{align*}
    \item $\dim V=\sum\limits_{i=1}^s\dim U_i$
    \item Eine Summe aus Unterräumen heiß \textit{direkt}, wenn sich die Unterräume paarweise trivial schneiden.
    \item Die Summe von Eigenräumen zu verschiedenen Eigenwerten ist direkt.
    \item $\alpha$ ist genau dann diagonalisierbar, wenn $V$ die direkte Summe der Eigenräume $E(\lambda_i)$ zu den verschiedenen Eigenwerten $\lambda_i$ ist.
    \item $\pi_i$ ist ein Epimorphismus (surjektive lineare Abbildung).
\end{enumerate}

%----------------------------------------------------------------------------------------------------------
%-----------------------------------------------Satz 1.9---------------------------------------------------
%----------------------------------------------------------------------------------------------------------
\begin{mysatz}\ \\
    Seien $\alpha$ und $\beta$ diagonalisierbare Endomorphismen, welche miteinander kommutieren (d.h. $\alpha\beta = \beta\alpha$). Dann sind $\alpha$ und $\beta$ gemeinsam diagonalisierbar, d.h. es gibt eine Basis aus Eigenvektoren von $\alpha$ und $\beta$.

    \textit{Beweis:}
    $\alpha$ ist diagonalisierbar $\Rightarrow \lambda_1,\ldots,\lambda_s$ seien verschiedene Eigenwerte und $U_1,\ldots,U_s$ die dazugehörigen Eigenräume mit $V = U_1 \oplus \ldots \oplus U_s$.
    Sei $u_i\in U_i$, dann ist auch $\beta(u_i)=u_i'\in U_i$, weil
    \begin{align*}
        \alpha(u_i')= \alpha(\beta(u_i)) = \beta\alpha(u_i) = \beta \lambda_i u_i = \lambda_i u_i'
    \end{align*}
    Da $\beta$ diagonalisierbar ist, existiert eine Basis $\left\{ b_1,\ldots,b_n \right\}$ von $V$ aus Eigenvektoren von $\beta$, d.h. $\beta(b_j)=\mu_j\cdot b_j$. Jedes $b_j$ lässt sich eindeutig schreiben als $b_j=\sum\limits_{i=1}^s u_{ij}$ mit $u_{ij}\in U_i.$ Dann ist
    \begin{align*}
        \sum_{i=1}^s \beta(u_{ij}) = \beta(b_j)=\mu_j\cdot b_j = \sum_{i=1}^s\mu_j u_{ij}
    \end{align*}
    Da $u_{ij}' = \beta(u_{ij}) \in U_i$ für jeden $i$ und $\mu_ju_{ij} \in U_i$, folgt die Eindeutigkeit $\beta(u_{ij}) = \mu_j u_{ij}$. Demnach ist, wenn $u_{ij}\neq0$ gilt $u_{ij}$ ein Eigenwert zu $\alpha$ mit $\alpha(u_{ij}) = \lambda_i \cdot u_{ij}$, aber auch ein Eigenwert zu $\beta$ mit $\beta(u_{ij})=\mu_j\cdot u_{ij}$ und daher ist die Menge $\left\{ u_{ij} \mid i = 1,\ldots,s;\ j = 1,\ldots,n \right\}$ ein Erzeugendensystem, aus dem man eine Basis wählen kann.
\end{mysatz}

%-----------------------------------------------Beispiel---------------------------------------------------
\textit{Beispiel:}

Gegeben Sei $V=\R^2$ und die Endomorphismen $\alpha,\beta$, die bzgl. der Standardbasis die folgenen Abbildungsmatrizen besitzen.
\begin{align*}
    \alpha:A=\begin{pmatrix}1&0\\1&2\end{pmatrix} \qquad \mbox{und} \qquad \beta:B=\begin{pmatrix}3&0\\-4&-1\end{pmatrix} \qquad \mbox{mit} \qquad AB=BA=\begin{pmatrix}3&0\\-5&-2\end{pmatrix}
\end{align*}
Dann ist 
\begin{align*}
    E_{\alpha}(1) = \langle \begin{pmatrix}1\\-1\end{pmatrix} \rangle = E_{\beta}(3) \mbox{ und }	E_{\alpha}(2) = \langle \begin{pmatrix}0\\ 1\end{pmatrix} \rangle = E_{\beta}(-1)
\end{align*}


\section{Hauptidealringe}

Ein Menge $R$ mit zwei Verknüpfungen $(+)$ und $(\cdot)$ heißt Ring, falls folgende Axiome gelten:
\begin{enumerate}
    \item $(R,+)$ ist eine kommutative Gruppe
    \item $(R,\cdot)$ ist assoziativ, d.h. $(a b) c = a (b c)$ für alle $a, b, c \in R$
    \item $(a + b) c = a c + a b$ und $c (a + b) = c a + c b$ für alle $a, b, c \in R$. Diese Eigenschaft nennt man Distributivität.
\end{enumerate}
Gilt zu den genannten Axiomen zusätzlich noch
\begin{enumerate}
    \item $a b = b a$ für alle $a, b \in R$, so heißt $R$ kommutativer Ring.
    \item $\exists e \in R$ mit $e a = a e = a$ für alle $a \in R$, so heißt $R$ Ring mit Einselement.
\end{enumerate}

%---------------------------------------------Definition 2.1------------------------------------------------
%--------------------------------------Linksideal, Rechtsideal,Ideal----------------------------------------
\begin{mydef} \textit{Linksideal, Rechtsideal, Ideal}\medskip

    Sei $R$ ein Ring und $I$ eine Untergruppe von $(R,+)$. Man nennt $I$ Linksideal genau dann, wenn
    \begin{align*}
        rx \in I \quad \forall r \in R \ \forall x \in I
    \end{align*}
    und Rechtsideal genau dann, wenn
    \begin{align*}
        xr \in I \quad \forall r \in R \ \forall x \in I
    \end{align*}
    und Ideal, falls $I$ sowohl Rechts-, als auch Linksideal ist. In diesem Fall schreibt man $I \lhd R$.
\end{mydef}

%-----------------------------------------------Bemerkung---------------------------------------------------
\textit{Bemerkung:}
\begin{enumerate}
    \item $\{ 0 \}$ und $R$ sind Ideale in jedem Ring.
    \item Ist $R$ kommutativ, dann fallen die Begriffe Linksideal, Rechtsideal und Ideal zusammen.
    \item Bsp. $R = \Z$ mit dem Ideal $I_m = \left\{ k\cdot m \mid k\in \Z, m\in \N \right\}$
    \item Bsp. $R = \R[x]$ mit dem Ideal $I = \{ p(x) \mid p(-1) = p(1) = 0 \} = \{ p(x) = (x-1)(x+1) \cdot q(x) \mid q(x) \in \R[x] \}$
    \item Der Durchschnitt von Linksidealen ist wieder ein Linksideal.
        $x \in I_1 \cap I_2 \Rightarrow rx \in I_1 \wedge rx \in I_2 \Rightarrow rx \in I_1\cap I_2$
    \item Die Summe von Linksidealen ist wieder ein Linksideal.
        $I_1 + I_2 = \{ x_1 + x_2 \mid x_1 \in I_1 \wedge x_2 \in I_2\}$
    \item Wenn $R$ ein Ring mit Einselement ist und $I$ ein zweiseitiges Ideal, das $1$ enthält, dann ist $I=R$.
\end{enumerate}

%---------------------------------------------Definition 2.2------------------------------------------------
%--------------------------------------Hauptideal, Hauptidealring-------------------------------------------
\begin{mydef}\textit{Hauptideal, Hauptidealring}\medskip

    Sei $R$ ein kommutativer Ring und $I$ ein Ideal von $R$. Man nennt $I$ Hauptideal, wenn ein $a \in R$ existiert mit
    \begin{align*}
        I = a \cdot R = \{ a\cdot r \mid r\in R \} = (a)
    \end{align*}
    Sei $R$ ein kommutativer, nullteilerfreier Ring mit $1$. Dann heißt $R$ Hauptidealring, wenn jedes Ideal ein Hauptideal ist.
\end{mydef}

%-----------------------------------------------Lemma 2.3---------------------------------------------------
%-----------------------------------------Z ist Hauptidealring----------------------------------------------

\begin{mylemma} \textit{$\Z$ ist Hauptidealring}\medskip

    Falls $I=\{ 0 \}$ so ist die Behauptung trivial. Nehmen wir an $I \neq \{ 0 \}$ und $0 \neq x \in I$. Dann ist auch $-x \in I$. Also ist entweder $x$ oder $-x$ positiv und daher aus $\N$. Daher ist $I \cap \N \neq \emptyset$.
    Sei $m$ die kleinste pos. Zahl in $I$. Dann ist $m \Z \subseteq I$, weil $m \in I \Rightarrow k \cdot m \in I$. Umgekehrt ist $I \subseteq m\Z$, denn sei $a \in I \Rightarrow a = q \cdot m + r$ mit $0 \leq r<m$. Dann ist $a-q\cdot m=r$. Da $a-q\cdot m \in I$ ist, muss auch $r \in I$ sein. Da aber $a$ die kleinste positive Zahl aus $i$ ist, folgt, dass $r=0$ ist. Damit bleibt $a=q\cdot m = (m)$.
\end{mylemma}

%-----------------------------------------------Definition 2.4-----------------------------------------------
%--------------------------------------------Grad eines Polynoms---------------------------------------------

\begin{mydef}\textit{Grad eines Polynoms}\medskip

    Sei $\K$ ein Körper und $p \in \K[x]$ ein Polynom mit $p = a_n x^n + a_{n-1} x^{n-1} + \ldots + a_1 x + a_0$ und $a_n \neq 0$, dann heißt $n$ der Grad des Polynoms $(\deg p)$. Das Nullpolynom hat den Grad $-\infty$
\end{mydef}

%-----------------------------------------------Lemma 2.5---------------------------------------------------
%----------------------------------------------Gradformeln--------------------------------------------------
\begin{mylemma}\label{gradformeln}\textit{Gradformeln}
    \begin{itemize}
        \item $\deg p \cdot q = \deg p + \deg q$
        \item $\deg(p + q) = \max \left\{ \deg p, \deg q \right\}$
    \end{itemize}
\end{mylemma}

%-----------------------------------------------Lemma 2.6---------------------------------------------------
%--------------------------------------------Division mit Rest----------------------------------------------
\begin{mylemma}\label{divRestK[x]}\textit{Division mit Rest}\medskip

    Seien $p , q \in \K[x]$, dann existieren eindeutig bestimmte Polynome $t, r \in \K[x]$ mit $p = t \cdot q + r$, wobei $\deg r < \deg q$ und $q\neq0$.

    \begin{itemize}
        \item Existenz: Induktion über $\deg p$.

            Induktionsanfang:
            \begin{align*}
                \deg p = 0
                \begin{cases}
                    \deg q>0  & \rightarrow p=0\cdot q+p \\
                    \deg q=0  & \rightarrow p=k\cdot q+0
                \end{cases}
            \end{align*}
            Induktionsschritt: $n - 1 \Rightarrow n$
            \begin{enumerate}
                \item Fall $\deg p < \deg q \Rightarrow p = 0 \cdot q + p$
                \item Fall $\deg p \geq \deg q$:

                    $p = a_n x^n + \ldots + a_1 x + a_0$ und $b_m x^m + \ldots + b_1 x + b_0$ mit $m \leq n$.

                    Sei $t_1 = a_n b_m^{-1} x^{n-m} \in \K[x]$. Dann ergibt $t_1 \cdot q = a_n x^n + s(x)$ mit $\deg s<n$.

                    Setzt man $p_1 = p - t_1 \cdot q$ mit $\deg p_1 < \deg p$.

                    Per Induktion gibt es dann $t_2, r \in \K[x]$ mit $p_1 = t_2 \cdot q + r$.

                    Also ist $p = p_1 + t_1 \cdot q = t_2 \cdot q + r + t_1 \cdot q = (t_1 + t_2) \cdot q + r$.
            \end{enumerate}

        \item Eindeutigkeit

            Sei $p = t \cdot q + r$, sowie $p = t' \cdot q + r'$ gegeben.

            Dann ist o.B.d.A $\deg r \geq \deg r'$ sowie $\deg(t' - t) \geq 0$. Dann folgt aufgrund von $r - r' = (t' - t) \cdot q$ und mit \ref{gradformeln}, dass
            \begin{align*}
                \deg r \geq \deg(r - r') = \deg ((t' - t) \cdot q) = \deg(t' - t) + \deg q
            \end{align*}
            Dann wäre $\deg q \leq \deg r$. Widerspruch.
            Demnach muss $\deg(t' - t)= -\infty \Rightarrow t' - t \mbox{ ist nach }$ \ref{gradformeln} das Nullpolynom.
            Somit folgt $t' = t$ und damit schließlich $r' = r$.
    \end{itemize}
\end{mylemma}

%----------------------------------------------------------------------------------------------------------
%-----------------------------------------------Satz 2.7---------------------------------------------------
%------------------------------------------K[x] ist Hauptidealring-----------------------------------------

\begin{mysatz}\textit{$\K[x]$ ist  Hauptidealring}\medskip

    $\K[x]$ ist kommutativ mit Eins und Nullteilerfremd.

    Letzteres gilt, weil aus $0 \neq p, q \in \K[x]$ folgt, dass $\deg (p \cdot q) = \deg p + \deg q \geq 0$ und daher $pq \neq 0$.

    Sei $I \neq \{ 0 \}$, für diesen Fall ist die Aussage klar, ein Ideal von $\K[x]$.
    Dann enthält $I$ mindestens ein Polynom $p \neq 0$.
    Sei $q \in \K[x]$ ein Polynom mit kleinstem in $I$ auftretenden Grad.
    Dann ist klar, dass $q \cdot \K[x] \subseteq I$, weil $q \in I$.

    Umgekehrt sei $p \in I$, d.h. $p = t \cdot q + r$ mit geeignetem $t, r \in \K[x]$.
    Diese existieren nach \ref{divRestK[x]} mit $\deg r < \deg q$.
    Umstellen liefert $I \ni p - tq = r$.
    Damit liegt auch $r \in I$ und wegen $\deg r < \deg q$ bleibt für $r$ nur das Nullpolynom übrig.
    Es folgt, dass $p = tq \in q \cdot \K[x]$ und daher ist $I = q \cdot \K[x]$ ein Hauptideal.
\end{mysatz}

%-----------------------------------------------Definition 2.8-----------------------------------------------
%------------------------------------------------------------------------------------------------------------

\begin{mydef}\textit{Einheit, Inverse, Vielfaches, kongruent, irreduzibel, teilerfremd, Primideal, Primelement, maximales Ideal}\medskip

    Sei $R$ ein kommutativer Ring mit Einselement.
    \begin{enumerate}
        \item $u \in R$ heißt Einheit, falls ein $v \in R$ existiert mit $u \cdot v = 1$ ($v$ ist dann eindeutig bestimmt und heißt das Inverse von $u$).
        \item Seien $a, b \in R$. Man sagt $a$ teilt $b$ in Zeichen $a \mid b$ oder $b$ ist ein Vielfaches von $a$ genau dann, wenn ein $c \in R$ existiert mit $a \cdot c = b$.
        \item Seien $a, b, c \in R$. $a$ heißt kongruent zu $b$ modulo $c$ in Zeichen $a \equiv b \mbox{ mod }c$ genau dann, wenn $c \mid b-a$.
        \item Wenn $p$ keine Einheit ist, aber aus $p = a \cdot b$ stets folgt, dass $a$ oder $b$ Einheit ist, dann heißt $p$ irreduzibel.
        \item Seien $a_1, \ldots ,a_n \in R$. Man nenn diese Elemente teilerfremd, falls aus $u \mid a_i \ \forall i \in I$ folgt, dass $u$ eine Einheit ist.
        \item Ein Ideal heißt Primideal, falls aus $a \cdot b \in P$ folgt, dass $a \in P$ oder $b \in P$ ist.
        \item Ein Element $0 \neq p \in R$ heißt genau dann Primelement, wenn $pR$ ein Primideal ist.
        \item Ein Ideal $M$ heißt maximal, falls $\{ 0 \}\subseteq M \subsetneqq R$ und es kein $I$ existiert mit $M \subsetneqq I \subsetneqq R$.
    \end{enumerate}
\end{mydef}

%-----------------------------------------------Bemerkung 2.9------------------------------------------------

\begin{mylemma}\ \label{lem2.9}

    \begin{enumerate}
        \item In einem Körper sind alle Elemente außer $0$ Einheiten. Primelemente existieren nicht.
        \item $a \mid b$ genau dann, wenn $aR \supseteq bR$

            \textit{Beweis:}
            \begin{itemize}
                \item $\exists c$ mit $b = c \cdot a \Rightarrow b \in aR$
                \item $b \in aR \Rightarrow b = a \cdot r \Rightarrow a \mid b$
            \end{itemize}
        \item Primelemente sind stets irreduzibel.\label{lem2.9.3}

            \textit{Beweis:}

            Sei $p$ Primelement mit $p = a \cdot b \Rightarrow P = pR$ ist Primideal und $ab \in P.$

            O.B.d.A.: $b \in P \Rightarrow b = p \cdot c$.
            Dann ist $p = a \cdot p \cdot c \Rightarrow (1 - ac)p = 0$ und somit $a \cdot c = 1 \Rightarrow a$ ist Einheit und damit $p$ irreduzibel.
        \item $p$ Primelement $\Leftrightarrow P = p \cdot R$ ist Primideal.\label{lem2.9.4}

            \textit{Beweis:}
            \begin{itemize}
                \item $ab \in P \Rightarrow a \in P \vee b \in P$
                \item $ab = r \cdot p \Rightarrow p \mid a \vee p\mid b$
            \end{itemize}
        \item Der Ring $R = \{a + bi \mid a, b\in \Z \}$ wird als Ganze Gaußsche Zahlenebene bezeichnet.
    \end{enumerate}
\end{mylemma}

%----------------------------------------------------------------------------------------------------------
%-----------------------------------------------Satz 2.10--------------------------------------------------
%----------------------------------------------------------------------------------------------------------

\begin{mysatz}\label{zornschelemma}\ \medskip

    Sei $R$ ein Hauptidealring und $0 \neq p \in R$.
    \begin{enumerate}
        \item Jede nichtleere Menge von Idealen enhält ein maximales Element. Insbesondere ist jedes Ideal $I \neq R$ in einem maximalen Ideal enthalten. \label{zli1}
        \item Die folgenden Aussagen sind äquivalent:
            \begin{enumerate}
                \item $p$ ist Primelement
                \item $p$ ist irreduzibel
                \item $Rp$ ist maximales Ideal
                \item $Rp$ ist Primideal
            \end{enumerate}
    \end{enumerate}
    Für den vollständigen Beweis von \ref{zornschelemma}.\ref{zli1} benötigt man das \textsc{Zorn}sche\footnote{Max August Zorn, * 6. Juni 1906 in Krefeld; $\dagger$ 9. März 1993 in Bloomington, Indiana, USA, US-amerikanischer Professor der Mathematik deutscher Abstammung} Lemma.
    Daher beschränken wir uns beim Beweis auf die beiden Spezialfälle $R = \Z$ und $R = \K[x]$.
    \begin{itemize}
        \item Sei $\mathcal{M}$ eine nichtleere Menge von Idealen, die nicht nur das Nullideal enthält. Für diesen Fall wäre nichts zu zeigen.
            Da jedes Ideal $\{0\} \neq I\in \mathcal{M}$ von einem Element $a$ erzeugt wird, also $I=(a)$,
            wählt man unter allen Idealen in $\mathcal{M}$ ein Ideal so, dass $0 < |a|$ für $I = \Z$ oder $0 < \deg(a)$ für $I = \K[x]$ minimal ist.
            Dieses Ideal ist dann ein maximales Ideal, denn für alle anderen Ideale gilt, dass das erzegende Element ein Vielfaches von $a$ ist.
            Die zweite Aussage folgt aus der ersten: sei $I \neq R$ ein Ideal; ein maximales Element in der (nicht-leeren) Menge 
            $\mathcal{M} = \{ A \mid I \leq A \lhd R, A \neq R\}$ ist dann ein maximales Ideal von $R$, welches $I$ enthält.
        \item $(1) \Rightarrow (2)$ steht schon in \ref{lem2.9}.\ref{lem2.9.4}
        \item $(2)\Rightarrow(3)$: Sei $p$ irreduzibel (d.h. $p$ keine Einheit) $\Rightarrow Rp \neq R$. Sei $Rp \lneq q A$ für ein Ideal $A$.
            Zu zeigen ist dann, dass $A = R$ gilt. Da $A$ ein Hauptideal ist, folgt $A = Ra$ für ein $a$. Wegen $p \in A$ gibt es ein $b$ mit $p = ab$.
            Die Irreduzibilität von $p$ erzwingt, dass $a$ oder $b$ eine Einheit ist. Wäre $b$ eine Einheit, dann $a = pb^{-1} \in pR$, also $A = Ra \leq Rp$, Widerspruch.
            Also ist $a$ eine Einheit und daher $A = Ra = R$.
        \item $(3)\Rightarrow(4)$: Sei $ab \in Rp$. Angenommen, weder $a$ noch $b$ liegen in $Rp$.
            Dann sind die Ideale $Ra + Rp$ und $Rb + Rp$ beide echt größer als $Rp$, also gleich $R$ wegen der Maximalität von $Rp$.
            Daher gibt es $r,s,t,u \in R$ mit $ra+sp = 1 = tb+up$. Dann folgt $1 = (ra + sp)(tb + up) = rtab + p(rua + stb + sup)\in Rp$, also $Rp = R$, Widerspruch.
        \item $(4)\Rightarrow(1)$: per Definition.
    \end{itemize}
\end{mysatz}

%-----------------------------------------------Bemerkung---------------------------------------------------

\textit{Bemerkung:}

Außer den von den Primelementen erzeugten Idealen gibt es noch ein weiteres Primideal, nämlich $\{0\}$. Ist dieses Ideal maximal, dann ist $R$ ein Körper.

%----------------------------------------------------------------------------------------------------------
%-----------------------------------------------Satz 2.11--------------------------------------------------
%-----------------------------------------Primfaktorzerlegung in HR---------------------------------------

\begin{mysatz} \textit{Primfaktorzerlegung in Hauptidealringen} \label{pfz-in-hir}

    Sei $R$ ein Hauptidealring und $\{Rp_i \mid i \in I\}$ die Menge der maximalen Ideale $\neq \{ 0 \}$.
    Dann hat jedes Element $0 \neq a \in R$ (bis auf Reihenfolge) eine eindeutige Zerlegung
    \begin{align*}
        a = u \cdot \prod_{i \in I} p_i^{n_i}
    \end{align*}
    mit $n_i \in \N$, fast allen $n_i = 0$ und der Einheit $u$.

    \textit{Beweis:}
    \begin{itemize}
        \item Existenz: Es sei $\mathcal{M} = \left\{Rb \mid a = b \cdot \prod\limits_{i \in I}p_i^{n_i}\right\}$. (Diese Menge existiert immer, trivialerweise für $b = a$ und alle $n_i = 0$.)
            Also gibt es ein maximales Element $Rc$ in $\mathcal{M}$.

            Wenn $Rc \neq R$, dann liegt $Rc$ in einem maximalen Ideal $M$ von $R$. Es ist $m \neq \{ 0 \}$, denn sonst ist $Rc = \{ 0 \}$, also $c = 0, a = 0$, Widerspruch.

            Also ist $M = Rp_j$ für ein $j \in I$. Dann ist $Rc \subseteq Rp_j$, also $c = dp_j$ mit geeignetem $d$.

            Wegen
            \begin{align*}
                a = c \prod\limits_{i\in I}p_i^{n_i} = dp_j\prod\limits_{i \in I}p_i^{n_i}
            \end{align*}
            folgt $Rd\in \mathcal{M}$. Es ist aber $c=p_jd\in Rd$, also $Rc \subseteq Rd$ und daher $Rc=Rd$ wegen der Maximalität von $Rc$ in $\mathcal{M}$. Es 
            folgt, dass $p_j$ ein Einheit ist, d.h $Rp_j=R$. Aber $Rp_j$ ist maximales Ideal, Widerspruch.
            Also ist $Rc=R$, d.h. $c$ ist ein Einheit und $a=c\prod_{i\in I}p_i^{n_i}$\par \medskip
        \item Eindeutigkeit: Angenommen, es ist
            \begin{align*}
                u \prod\limits_{i \in I}p_i^{a_i} = v \prod\limits_{i \in I}p_i^{b_i}
            \end{align*}
            mit $a_1 > a_2$. Dann ist
            \begin{align*}
                u p_1^{a_1 - b_1} \prod\limits_{i \neq 1}p_i^{a_i}= v \prod_{i \neq 1}p_i^{b_i} \in Rp_1
            \end{align*}
            Da $Rp_1$ ein Primideal ist, muss mindestens einer der Faktoren von $v \prod\limits_{i \neq 1}p_i^{b_i}$ in $Rp_1$ liegen. $v$ tut's nicht, weil $v$ eine Einheit ist. 
            Es ist aber auch $p_j \notin Rp_1$, denn sonst ist $Rp_j \subseteq Rp_1$, also $Rp_j = Rp_1$, wegen der Maximalität von $Rp_j$ und $Rp_1$, Widerspruch.
            Also ist $a_i = b_i$ für alle $i$ und dann auch $u = v$.
    \end{itemize}
\end{mysatz}

%-----------------------------------------------Lemma 2.12---------------------------------------------------
%-----------------------------------------Folgerung aus Satz 2.11--------------------------------------------

\begin{mylemma} Folgerung aus Satz \ref{pfz-in-hir}

    Sei $0 \neq p(x) \in \K[x]$ ein Polynom vom Grad $n$, dann hat $p(x)$ höchstens $n$ Nullstellen.

    \textit{Beweis:}

    Seien $a_1, \ldots, a_m$ die Nullstellen von $p(x)$, dann sind $x - a_1, x - a_2, \ldots, x - a_m$ irreduzible Teiler von $p$ mit $\deg x - a_i = 1 \quad \forall i = 1, \ldots, m$.
    
    Daher ist auch $\prod\limits_{i = 1}^m x - a_i$ ein Teiler von $p$.
    Es gilt
    \begin{align*}
        \deg \left( \prod_{i = 1}^m x - a_i \right) = m < n = \deg p(x)
    \end{align*}
    und die Behauptung folgt.
\end{mylemma}

%-----------------------------------------------Lemma 2.13---------------------------------------------------
%------------------------------------------------------------------------------------------------------------

\begin{mylemma}\label{lem2.13} \qquad  \par
    Seien $a_1,\ldots,a_n$ teilerfremde Elemente eines Hauptidealringes. Dann existieren $b_1,\ldots,b_n$ mit 
    \begin{align*}
        1 = \sum_{i=1}^n b_ia_i
    \end{align*}

    \textit{Beweis:}

    Sei $I = \sum\limits_{i=1}^n Ra_i$ Dann existiert ein $a$, sodass $I = Ra$. Da $a_i \in I$, gibt es ein $c_i$ mit $a_i = c_i a$, also $a \mid a_i$ für $i=1,\ldots,n$.
    Nach Vorraussetzung ist dann $a$ eine Einheit, also $I=R$. Insbesondere ist $1 \in I$, also gibt es $b_i$ mit der gewünschten Eigenschaft.
\end{mylemma}

%-----------------------------------------------Definition 2.14-----------------------------------------------
%----------------------------------------------------ggT------------------------------------------------------

\begin{mydef} \textit{größter gemeinsamer Teiler}

    Seien $a_1, \ldots, a_n$ Elemente eines Hauptidealringes $R$. Dann ist $I = \sum\limits_{i=1}^n Ra_i$ ein Ideal in $R$.
    Da $R$ Hauptidealring ist, existiert ein $g \in R$ mit $I = Rg$. Dann heißt $g$ der größte gemeinsame Teiler (ggT) von $a_1, \ldots, a_n$
\end{mydef}

%-----------------------------------------------Bemerkung---------------------------------------------------

\textit{Bemerkung:}
\begin{itemize}
    \item Nach Definition von ggT existieren $b_1, \ldots, b_n$ mit $g = b_1 a_1 + \ldots + b_n a_n$
        
        \hfill (Verallgemeinerung von \ref{lem2.13})
    \item $\forall i=1, \ldots, n:$
        \begin{align*}
        Rg \geq Ra_i & \Rightarrow g \mid a_i\\
        Rt \geq Ra_i & \Rightarrow Rt \geq Rg \Rightarrow t\mid g
        \end{align*}
\end{itemize}

%-----------------------------------------------Definition 2.15-----------------------------------------------
%----------------------------------------------------kgV------------------------------------------------------

\begin{mydef} \textit{kleinstes gemeinsames Vielfaches}

    Seien $a_1, \ldots, a_n$ Elemente eines Hauptidealringes $R$. Dann ist $I=\bigcap\limits_{i=1}^n$ ein Ideal in $R$.
    Da $R$ Hauptidealring ist, existiert ein $k \in R$ mit $I = Rk$. Dann heißt $k$ das kleinste gemeinsame Vielfache von $a_1, \ldots, a_n$.
\end{mydef}

%-----------------------------------------------Bemerkung---------------------------------------------------

\textit{Bemerkung:}

    Da $Rk \leq Ra_i$ für jedes $i$ nach \ref{lem2.9}.\ref{lem2.9.3} gilt, ist $k$ gemeinsames Vielfaches der $a_i$.
    Wenn zusätzlich $a_i \mid t$ für jedes $i$, dann ist $Rt \leq Ra_i$, also $Rt \leq Rk$ und damit ist $k$ das kleineste gemeinsame Vielfache.\\

%-----------------------------------------------Bemerkung---------------------------------------------------

\textit{Bemerkung:} \textit{Bestimmung von ggT und kgV}

    Sei $0 \neq a_i$ für $i = 1, \ldots, n$ und sei $a_i = \prod\limits_j p_j^{e_{ij}}$ die Faktorisierung in Primelemente.
    Für jedes $j$ sei $u_j = \min(e_{1j}, \ldots, e_{nj})$ und $v_j = \max(e_{1j}, \ldots, e_{nj})$.
    Dann ist $g = \prod\limits_j p_j^{u_j}$ der größte gemeinsame Teiler (ggT) und $k = \prod\limits_j p_j^{v_j}$ das kleinste gemeinsame Vielfache (kgV) der $a_i$.\\

%-----------------------------------------------Bemerkung---------------------------------------------------

\textit{Bemerkung:} \textit{Euklidischer Algorithmus zur Bestimmung des ggT in $\Z$ und $\K[x]$}

    Dazu seien $a\neq0, b\neq0 \in R$. Nach \ref{divRestK[x]} lässt sich dann $a$ zerlegen in \par
    \begin{center}
        \begin{tabular}{ccc}
            & $\Z$ & $\K[x]$ \\ \hline
            $a=q_1\cdot b + r_1$ & $0\leq r_1 < |b|$ & $\deg(r)<\deg(b)$\\
            $b=q_2\cdot r_1 + r_2$ & $0\leq r_2 < r_1 $ & $\deg(r_2)<\deg(r_1)$ \\
            $r_1=q_3\cdot r_2 + r_3$ & $0\leq r_3 < r_2 $ & $\deg(r_3)<\deg(r_2)$\\
            $\vdots$ & $\vdots$ & $\vdots$ \\
            $r_{n-1}=q_{n+1}\cdot r_n + r_{n+1}$ & $0\leq r_{n+1}<r_n$ & $\deg r_{n+1}<\deg r_n$\\
            $r_{n}=q_{n+2}\cdot r_{n+1} + 0$ & &
        \end{tabular}
    \end{center}
    Der Rest wird nach endlichen vielen Schritten $0$, weil $r_i < r_{i-1}$ eine streng monoton fallende Folge von natürlichen Zahlen ist.

    Aus der letzten Gleichung erhält man dann $r_{n+1} \mid r_{n} \Rightarrow r_{n+1} \mid r_{n-1} \Rightarrow \ldots \Rightarrow r_{n+1} \mid b \Rightarrow r_{n+1} \mid a$.
    Also ist $r_{n+1}$ gemeinsamer Teiler von $a$ und $b$.

    Aus der Annahme $\exists t \in R$ mit $t \mid a \ \wedge \ t \mid b$ erhält man wegen $t \mid r_1\Rightarrow t \mid r_2\Rightarrow \ldots \Rightarrow t \mid r_{n+1}$,
    dass $r_{n+1}$ der größte gemeinsame Teiler ist.

    Ist man an der Darstellung des ggT von $a$ und $b$ durch eben diese Elemente interessiert, so erhält man
    \begin{align*}
        r_{n+1} & = r_{n-1} - q_{n+1} \cdot r_n = r_{n-1} - q_{n+1} \cdot (r_{n-2} - q_n \cdot r_{n-1}) = \ldots = \\
        & = k\cdot a + l\cdot b
    \end{align*}

%-----------------------------------------------Beispiel ---------------------------------------------------
\textit{Beispiel:} \textbf{ggT$(9,88)$}

    Wegen 
    \begin{align*}
        88 & = 9 \cdot 9 + 7\\
        9 & = 1 \cdot 7 + 2\\
        7 & = 3 \cdot 2 + 1\\
        2 & = 2 \cdot 1 + 0
    \end{align*}
    ist ggT$(9,88)=1$.
    
    Zusätzlich ist
    \begin{align*}
        1 & = 7 - 3 \cdot 2\\
        & = 7 - 3 \cdot (9 - 7) = 4 \cdot 7 - 3 \cdot 9 = 4 \cdot (88 - 9 \cdot 9) - 3 \cdot 9\\
        & = 4 \cdot 88 - 39 \cdot 9
    \end{align*}
    die Darstellung des ggT$(8,99)$ durch eben diese Zahlen.\par\medskip

    \textbf{Diophantische Gleichungen}

    Als Diophantische Gleichung bezeichnet man eine Gleichung der Form $ax + by = c$ mit $a,b,c\in \Z$ und der Lösungsmenge $(x,y) \in \Z^2$.

    Ganzzahlige Lösungen existieren genau dann, wenn ggT$(a,b) \mid c$, denn der ggT$(a,b)$ teilt die linke Seite der Gleichung, also muss er auch die rechte Seite teilen.

    Wähle $a',b',c'\in \Z$ so, dass $a'x + b'y = c'$ mit ggT$(a', b') = 1$ gilt.

    Dann ist mittels des euklidischen Algorithmus' $1 = k \cdot a' + l \cdot b'$ und $c' = {(c'k)}_{x_0} \cdot a' + {(c'l)}_{y_0} \cdot b'$ mit
    den allgemeinen Lösungspaaren $x_i=x_0 + i\cdot b$ und $y_i=y_i\cdot a$.

%-----------------------------------------------Satz 2.16--------------------------------------------------
%-----------------------------------------Chinesischer Restsatz--------------------------------------------

\begin{mysatz} \textit{Chinesischer Restsatz}

    Seien $a_1, \ldots, a_n$ paarweise teilerfremde Elemente eines Hauptidealringes $R$ und $b_1, \ldots, b_n$ beliebige Elemente aus dem Hauptidealring.
    Dann existiert ein $c \in R$ mit $c \equiv b_i \mod a_i \ \forall i=1, \ldots, n$

    \textit{Beweis:} Induktion über $n$.

    Der Fall $n=1$ hat die triviale Lösung $c=b_1$.
    
    Sei nun $n>1$ und $c_0 \equiv b_i \mod a_i$ für alle $i=1, \ldots, n-1$.

    Weil $a_n$ teilerfremd zu $d = a_1 \cdot \ldots \cdot a_{n-1}$ ist, existieren $r, s \in R$ mit $1 = ra_n + sd$.
    Dann ist also $sd \equiv 0 \mod a_i$ für $i < n$ und $sd \equiv 1 \mod a_n$.
    Daher gilt
    \begin{align*}
        c & = c_0 + (b_n - c_o) \cdot sd \mod a_n = c_0 + (b_n - c_0) \mod a_n \\
        c & = b_n \mod a_n
    \end{align*}
\end{mysatz}


\section{Normalformen von Matrizen}
In dem ganzen Paragraphen sei $\K$ ein beliebiger Körper und $V$ ein endlich-dimensionaler $\K$-VR der Dimension $\dim V=n$ und $\alpha$ ein Endomorphismus von $V$.\par\medskip

Bezeichne End$(V)$ die Menge aller Endomorphismen von $V$.
Dann bildet End$(V)$ mit der Verknüpfung $\oplus$ wegen $(\alpha \oplus \beta) (v) = \alpha(v) + \beta(v) \quad \forall v\in V$ eine kommutative Gruppe.
Zusammen mit der zweiten Verknüpfung $\odot$ wegen $(k \odot \alpha)(v) = k\cdot \alpha(v) \quad \forall k\in \K, \forall v\in V$ sogar einen Vektorraum über dem zugrunde liegendem Körper $\K$.
Bezüglich $\oplus$ und $\circ$, wobei $(\alpha \circ \beta) = \alpha(\beta(v))$ die Hintereinanderausführung bezeichnet, wird $\left(\End(V),\oplus,\circ\right)$ zu einem nichtkommutativem Ring mit Einselement.\par\medskip

Für jedes $f \in \K[x]$, etwa $f(x) = a_n x^n + \ldots + a_1 x + a_0$, ist $f(\alpha) = a_n \alpha^n + \ldots + a_1 \alpha^1 + a_0 \alpha^0$ ein Endomorphismus von $V$, wobei $\alpha^0 = \id_v$.
Wie man leicht sieht ist 
\begin{align*}
    I = \left\{  f\in \K[x] \mid f(\alpha) = 0 \right\}
\end{align*}
ein Ideal von $\K[x]$. $I$ ist aber nicht das Nullideal, denn End$(V)$ ist isomorph zu dem Körper $\K^{n \times n}$ und daher gilt $\dim\End(V) \cong \dim\K^{n \times n} = n^2$.
Die Menge $\left\{ \alpha^0, \alpha, \alpha^2, \ldots, \alpha^{n^2}\right\}$ besitz aber $n^2+1$ Elemente, die damit linear abhängig sind.
Daher existieren $k_0, \ldots k_{n^2} \in \K$, nicht alle$= 0$, mit $\sum\limits_{i=0}^{n^2} k_i \cdot \alpha^i = 0$.
Damit besitzt $I$ mindestens ein Polynom $f \neq 0$.
Nach dem vorherigen Kapitel existiert also ein Polynom $m \neq 0$ mit $I = m \cdot \K[x]$, welches als normiert angesehen werden kann und durch $\alpha$ eindeutig bestimmt ist.

%---------------------------------------------Definition 3.1------------------------------------------------
%---------------------------------------------Minimalpolynom------------------------------------------------

\begin{mydef}\label{minpoly}\textit{Minimalpolynom}

    Das Polynom $m \neq 0$ kleinsten Grades, welches normiert ist und für das $m(\alpha) = 0$ gilt, heißt Minimalpolynom von $\alpha$.

    Sei $A$ eine quadratische Matrix. Dann heißt das kleinste normierte Polynom $m \neq 0$ mit $m(A) = 0$ das Minimalpolynom von $A$.
\end{mydef}

%------------------------------------------------Bemerkung--------------------------------------------------

\textit{Bemerkung:}
\begin{enumerate}
    \item Minimalpolynom und charakteristisches Polynom von $\alpha$ sind zu unterscheiden.
    \item Wenn $A$ eine Matrix bzgl. einer Basis $B$ des Endomorphismus $\alpha$ ist, dann ist das Minimalpolynom von $\alpha$ gleich dem Minimalpolynom von $A$.
        Insbesondere haben ähnliche Matrizen das gleiche Minimalpolynom.
    \item Nach der Definition ist $\deg m \leq n^2$
    \item Es ist $f(\alpha) = 0$ für ein $f \in \K[x]$ genau dann, wenn $m \mid f$.
\end{enumerate}

%------------------------------------------------Bemerkung--------------------------------------------------

\textit{Bemerkung: }\textit{Bezeichnung}

Im ganzen Paragraphen ist $m$ das Minimalpolynom von $\alpha \in \End(V)$.
Um Klammern zu sparen schreiben wir einfach $fv$ für $f(\alpha(v))$.
Des Weiteren sind $\ker (f) = \ker (f(\alpha))$ und $\IM (f)=\IM (f(\alpha))$.

%---------------------------------------------Definition 3.2------------------------------------------------
%------------------------------------------Invarianter Unterraum--------------------------------------------

\begin{mydef}\textit{Invarianter Unterraum}

    Ein Unterraum $U \leq V$ heißt genau dann invariant unter $\alpha$, wenn $\alpha(U)\subseteq U$.
\end{mydef}

%%------------------------------------------------Bemerkung--------------------------------------------------

\begin{mylemma} \label{lemdef3.2}\

    \begin{enumerate}
        \item $\ker \alpha$ und $\IM \alpha$, sowie $\ker f$ und $\IM f$  sind invariante Unterräume für beliebiges $f \in \K[x]$.

            Des Weiteren sind Summen und Schnitte invarianter Unterräume wieder invariant.
        \item \label{lemdef3.2-2} Für ein beliebiges $v \setminus \left\{ 0 \right\} \in V$ erzeugt $\left\{ v,\alpha(v),\alpha^2(v),... \right\}$ einen invarianten Unterraum.

            Solche invarianten Unterräume heißen zyklische Unterräume.
        \item Wenn $U$ ein invarianter Unterraum ist, dann kann man $\alpha$ auf $U$ einschränken und erhält einen Endomorphismus $\alpha_{/_U}$ von 
            $U$. Dieser hat ebenfalls ein Minimalpolynom, etwa $m_U$.

            Da $m(\alpha_{/_U}) = 0$ ist, folgt aus der Bemerkung zu Definition \ref{minpoly} $m_{/_U} \mid m$.
    \end{enumerate}
\end{mylemma}

%-----------------------------------------------Lemma 3.3---------------------------------------------------
%-----------------------------------------------------------------------------------------------------------

\begin{mylemma}\label{m=f*g}\

    Wenn $m = f\cdot g$ mit normierten Polynomen $f, g \in \K[x]$, dann ist $U := \ker g \geq \IM f$ und $m_{/_U} = g$.

    \textit{Beweis:}

    Es ist $(gf)V = mV = 0$, also $\IM f = fV \leq \ker g=U$. Insbesondere ist $m_{/_U} fV = 0$, also $m \mid f \cdot m_{/_U}$ und daher auch $g \mid m_{/_U}$.
    Andererseits ist $g(\alpha_{/_U})=0$ nach Definition von $U$, also $m_{/_U} \mid g$. Da beide Polynome normiert sind, folgt die Gleichheit.
\end{mylemma}

%----------------------------------------------------------------------------------------------------------
%-----------------------------------------------Satz 3.4---------------------------------------------------
%------------------------------------------zyklische Vektorräume-------------------------------------------

\begin{mysatz} \label{zyklischeVR}\textit{Zyklische Vektorräume}

    Wenn $V$ ein zyklischer Raum ist bezüglich des Endomorphismus $\alpha$, d.h. $\exists\ v$ mit $V = \left\langle v, \alpha (v), \alpha^2 (v), \ldots \right\rangle$, dann gilt:
    \begin{enumerate}
        \item \label{zykVR-1} $n = \dim V = \deg m$ und $\left\{ v,\alpha (v), \ldots, \alpha^{n - 1}(v) \right\}$ ist eine Basis von $V$.
        \item Bezüglich der Basis in \ref{zyklischeVR}.\ref{zykVR-1} hat $\alpha$ die Matrix
            \begin{align*}
                A & =
                \begin{pmatrix}
                    0       & \cdots & \cdots   & 0         & -k_0\\
                    1       & \ddots &          & \vdots    & -k_1\\
                    0       & \ddots & \ddots   & \vdots    & \vdots\\
                    \vdots  & \cdots & 1        & 0         & \\
                    0       & \cdots & 0        & 1         & -k_{n-1}
                \end{pmatrix}
                \intertext{wobei}
                m(x) & = x^n + \sum\limits_{j=0}^{n-1} k_j \cdot x^{j}
            \end{align*}
            das Minimalpolynom ist.
        \item $m$ ist zugleich auch das charakteristische Polynom.
        \item \label{zykVR-4} Wenn $m = f \cdot g$ mit normierten Polynomen $f$ und $g$ ist, dann ist $\ker g = \IM f = U$ ein invarianter Unterrum.
            $U$ ist ebenfalls zyklisch und $\dim U = \deg g$.
        \item \label{zykVR-5} Die Abbildung $g \mapsto \ker g$ ist eine bijektive Abbildung der normierten Teiler von $m$ auf die invarianten Unterräume von $V$.
    \end{enumerate}
    \textit{Beweis:}
    \begin{enumerate}
        \item Sei $r \in \N$ so gewählt, dass $\left\{ v,\alpha(v),\ldots,\alpha^{r-1}(v) \right\}$ linear unabhängig, aber $\left\{ v,\alpha(v),\ldots,\alpha^{r-1}(v),\alpha^r (v) \right\}$  linear abhängig ist.
            Offenbar ist dann $\alpha^r (v)$ eine Linearkombination von $\left\{ v,\alpha(v),\ldots,\alpha^{r-1}(v) \right\}$. Also gibt es $k_i \in \K$ mit
            \begin{align*}
                0 & = \alpha^r(v) + \sum\limits_{i = 0}^{r - 1} k_i \cdot \alpha^i (v) = pv
                \intertext{mit}
                p(x) & = x^r + \sum\limits_{i = 0}^{r - 1} k_i \cdot x^i
            \end{align*}
            und nach Wahl von $r$ ist $p$ das normierte Polynom kleinsten Grades, welches $v$ annulliert.
            Es annulliert dann aber auch alle $\alpha^j (v)$, also auch deren Erzeugnis, d.h. ganz $V$.
            Daher ist $p = m$, insbesondere $\deg m = r$.
            Per Induktion sieht man leicht, dass sich mit $\alpha^r (v)$ auch jedes $\alpha^s (v)$ für $s \geq r$ als Linearkombination von $\left\{ v,\alpha(v),\ldots,\alpha^{r-1}(v)\right\}$ schreiben lässt.
            Diese Menge ist also eine Basis von $V$, insbesondere ist $r = n$.

        \item Das Bild eines der Basisvektoren $\alpha^j (v)$ unter $\alpha$ ist $\alpha^{j+1}(v)$, also der nächste Basisvektor außer im Fall $j = n-1$. Dann ist
            \begin{align*}
                \alpha^n (v) = -\sum\limits_{i = 0}^{n - 1} k_i \cdot \alpha^i(v).
            \end{align*}
            Daher hat $A$ die angegebene Form.

        \item Es ist
            \begin{align*}
                xE-A =
                \begin{pmatrix}
                    x & \cdots & \cdots & 0 & k_0\\
                    -1 & \ddots & & \vdots &\\
                    0 & \ddots & \ddots & \vdots & \vdots\\
                    \vdots & \cdots & -1 & x & k_{n-2}\\
                    0 & \cdots & 0 & -1 & x + k_{n-1}
                \end{pmatrix}
            \end{align*}
            Die Untermatrix, welche durch Streichen der ersten Zeile und ersten Spalte entsteht, hat die gleiche Form, also (per Induktion) die Determinante
            \begin{align*}
                x^{n-1} + \sum\limits_{i=1}^{n-1} k_i \cdot x^{i-1}
            \end{align*}
            Durch Entwickeln nach der ersten Zeile erhält man dann daher
            \begin{align*}
                \det(xE - A) = x \left( x^{n-1} + \sum\limits_{i = 1}^{n - 1} k_i \cdot x^{i-1} \right) + (-1)^{n + 1} k_0 (-1)^{n-1} = m(x)
            \end{align*}

        \item In \ref{m=f*g} stehen schon viele der Aussagen. Wir zeigen, dass $\ker g \leq \IM f$.

            Dazu sei $u \in \ker g$. Weil $V$ zyklisch ist, gibt es ein Polynom $h$ mit $u = hv$, also $u \in \IM h$.
            Dann ist $0 = gu = ghv$, also nach \ref{zyklischeVR}.\ref{zykVR-1}, $m = gf$ ein Teiler von $gh$ und daher $f \mid h$.
            Folglich ist $\IM f \geq \IM h$ und insbesondere $u \in \IM f$. Es ist klar, dass $\left\{ fv, \alpha(fv), \ldots \right\}$ ein Erzeugendensystem von $ fV=U$ ist.
            Also ist $U$ zyklisch. Nach \ref{zyklischeVR}.\ref{zykVR-1} ist dann $\dim U = \deg m_{/_U} = \deg g$.

        \item Wenn $g$ und $h$ normierte Teiler von $m$ sind mit $\ker g = \ker h = U$, dann ist $g = m_{/_U} = h$ nach \ref{m=f*g}.
            Also ist die Abbildung $g \mapsto \ker g$ injektiv.

            Sie ist auch surjektiv: dazu sei $U$ ein beliebiger invarianter Unterraum.
            Setzt man $I = \left\{ h \in \K[x] \mid hv \in U \right\}$, dann ist leicht zu kontrollieren, dass $I$ ein Ideal von $K[x]$ ist, welches $m$ enthält.
            Also ist $I = f \cdot K[x]$ für ein geeignetes normiertes $f \in \K[x]$ mit $f \mid m$, etwa $m = f \cdot g$.
            Weil $V$ zyklisch ist, ist $U = Iv = f \cdot \K[x] v = fV = \IM f = \ker g$, wobei die letzte Gleichheit aus \ref{zyklischeVR}.\ref{zykVR-4} folgt.
    \end{enumerate}
\end{mysatz}

%-----------------------------------------------Lemma 3.5---------------------------------------------------
%-----------------------------------------------------------------------------------------------------------

\begin{mylemma}\label{lem3.5}\

    Seien $\lambda_1, \ldots, \lambda_i \in V^{*}$. Dann gilt:
    \begin{enumerate}
        \item $\dim \left( \bigcap\limits_{i=1}^t \ker \lambda_i \right) \geq n-t$
        \item Wenn $\mu \in V^{*}$ linear abhängig von $\left\{ \lambda_1, \ldots, \lambda_i \right\}$ ist, dann
            \begin{align*}
                \ker \mu \cap \bigcap\limits_{i=1}^t \ker \lambda_i = \bigcap\limits_{i=1}^t \ker \lambda_i
            \end{align*}
    \end{enumerate}
    \textit{Beweis:} Übungsaufgabe 
\end{mylemma}

%----------------------------------------------------------------------------------------------------------
%-----------------------------------------------Satz 3.6---------------------------------------------------
%----------------------------------------------------------------------------------------------------------

\begin{mysatz}\label{Existenz/zyklische/invarianteUR}

    Sei $\alpha \in \End(V)$ mit Minimalpolynom $m = p^r$, wobei $p$ irreduzibel ist.
    Dann gilt:
    \begin{enumerate}
        \item Es gibt einen invarianten zyklischen Unterraum $Z \leq V$ mit $m_{/_Z} = m$.
        \item \label{eziUR-2} Es existiert ein invarianter Unterraum $W$ mit $V = Z \oplus W$.
    \end{enumerate}

    \textit{Beweis:}
    \begin{enumerate}
        \item Wäre zu jedem invarianten zyklischen Unterraum $U$ das Minimalpolynom ein echter Teiler von $m$, d.h $m_{/_U}| p^{r-1}$.
            Aber dann wäre $p^{r-1}(\alpha) = 0$ entgegen der Minimalität von $m$.
        \item Nach \ref{zyklischeVR}.\ref{zykVR-4} und \ref{zyklischeVR}.\ref{zykVR-5} enthält $Z$ genau einen kleinsten invarianten Unterraum $\neq 0$, nämlich $S = p^{r-1} Z$.
            Sei $\lambda \in V^{*}$ so gewählt, dass $\lambda_{/_S} \neq 0$.
            Für jedes $i = 0, 1, \ldots$ ist $\lambda_i = \lambda \alpha^{i} \in V^{*}$.

            Sei
            \begin{align*}
                W = \bigcap\limits_{i = 0}^{\infty} \ker \lambda_i
            \end{align*}
            Dann ist $W$ invariant, denn wenn $w \in W$, dann ist $\lambda_i \alpha(w) = \lambda_{i + 1} (w)$ für alle $i$, also $\alpha(w) \in W$.
            Setzt man $t = \deg m$, so ist $\alpha^t$ und jede höhere Potenz von $\alpha$ eine Linearkombination von $\left\{ \alpha^0, \ldots, \alpha^{t - 1} \right\}$,
            folglich sind $\lambda_t, \lambda_{t+1}, \ldots$ Linearkombinationen von $\left\{ \lambda_0, \ldots, \lambda_{t - 1} \right\}$.

            Aus \ref{lem3.5} folgt jetzt, dass
            \begin{align*}
                W = \bigcap\limits_{i = 0}^{t - 1} \ker \lambda_i
            \end{align*}
            und dass $\dim W \geq n-t$.

            Als Schnitt von invarianten Unterräumen ist $Z \cap W$ invariant.
            Daher ist $Z \cap W = 0$, denn sonst wäre $S \leq Z \cap W \leq W \leq \ker \lambda$ im Widerspruch zur Wahl von $\lambda$.
            Weil $Z$ zyklisch mit Minimalpolynom $m$ ist, gilt $\dim Z = \deg m = t$ nach \ref{zyklischeVR}.\ref{zykVR-1}.
            Folglich ist dann
            \begin{align*}
                n = \dim (V) \geq \dim(Z + W) = \dim Z + \dim W \geq t + (n-t) = n
            \end{align*}
            Es folgt die Behauptung.
    \end{enumerate}
\end{mysatz}

%------------------------------------------------Beispiel---------------------------------------------------

\textit{Beispiel:}

Gegeben sei eine Abbildung $\alpha:\R^4 \mapsto \R^4$ und bezeichne $\left\{ b_1, \ldots, b_4 \right\}$ die Standardbasis.

\begin{minipage}{0.4\linewidth}
    \begin{align*}
        \alpha(b_1) & = b_1\\
        \alpha(b_2) & = b_1 + b_2\\
        \alpha(b_3) & = b_2 + b_3\\
        \alpha(b_4) & = b_4 - b_1
    \end{align*}
\end{minipage}
\begin{minipage}{0.5\linewidth}
    \begin{align*}
        \Rightarrow \qquad A =
        \begin{pmatrix}
            1 & 1 & 0 & -1\\
            0 & 1 & 1 & 0\\
            0 & 0 & 1 & 0\\
            0 & 0 & 0 & 1
        \end{pmatrix}
    \end{align*}
\end{minipage}\par\medskip

Dann sind die Elemente der Menge $\left\{ E, A, A^2 \right\}$ linear unabhängig.

Die nächste Potenz von $A$ lässt sich als Linearkombination darstellen
\begin{align*}
    A^3 =
    \begin{pmatrix}
        1 & 3 & 3 & -3\\
        0 & 1 & 3 & 0\\
        0 & 0 & 1 & 0\\
        0 & 0 & 0 & 1
    \end{pmatrix}
    = 3A^2-3A+E
\end{align*}
Damit kann man das Minimalpolynom ablesen: $m(x) = x^3 - 3x^2 + 3x-1 = (x-1)^3$.\par\medskip

Um den zyklischen invarianten Unterraum $Z$ zu bestimmen überlegt man sich zuerst, dass $\dim Z = 3$ sein muss, da $\deg m = 3$ ist.
Dann ist $Z = \left\langle b_3, \alpha(b_3), \alpha^2(b_3) \right\rangle = \langle \underbrace{b_3}_{e_1},\underbrace{b_2+b_3}_{e_2},\underbrace{b_1+2b_2+b_3}_{e_3} \rangle $.

$\alpha^3(b_3)$ ist linear abhängig von $\left\{ b_3,\alpha(b_3),\alpha^2(b_3) \right\}$:
\begin{align*}
    \alpha_{/_Z}(e_3) & = \alpha(b_1 + 2b_1 + b_3)= b_1 + 2(b_1 + b_2)+(b_2 + b_3)\\
    & = b_3 - 3\cdot(b_2 + b_3) + 3\cdot(b_1 + 2b_1 + b_3)\\
    & = e_1 - 3\cdot e_2 + 3\cdot e_3
\end{align*}
Damit ist
$A' =
\begin{pmatrix}
    0 & 0 & 1\\
    1 & 0 & -3\\
    0 & 1 & 3
\end{pmatrix} $ und $m_{/_Z}(x) = (x-1)^3 = m(x)$  wie im Satz behauptet.\par \medskip
Um den invarianten Unterraum $W$ aus Satz \ref{Existenz/zyklische/invarianteUR}.\ref{eziUR-2} zu bestimmen, muss man den Beweis vollführen.
Der kleinste invariante Unterraum $S$, der in $Z$ enthalten ist, errechnet sich durch $S = p^{r-1}Z$, also in diesem Beispiel
\begin{align*}
    (\alpha-1)^2 (e_1) & = \alpha^2 (e_1) - 2\alpha (e_1) + e_1 = b_1\\
    (\alpha-1)^2 (e_2) & = \alpha^2 (e_2) - 2\alpha (e_2) + e_2 = b_1\\
    (\alpha-1)^2 (e_3) & = \alpha^2 (e_3) - 2\alpha (e_3) + e_3 = b_1\\
    \Rightarrow S & = \left\langle b_1 \right\rangle
\end{align*}
\begin{center}
    \begin{tabular}{cccc}
        $\lambda_0 : V \mapsto \K$ & $\lambda_1 = \lambda \alpha$ & $\lambda_2 = \lambda \alpha^2$ & $\lambda_3 = \lambda \alpha^3$ \\
        $b_1 \mapsto 1$ & $b_1 \mapsto 1$ & $b_1 \mapsto 1$ & $b_1 \mapsto 1 $ \\
        $b_2 \mapsto 0$ & $b_2 \mapsto 1$ & $b_2 \mapsto 2$ & $b_2 \mapsto 3 $ \\
        $b_3 \mapsto 0$ & $b_3 \mapsto 0$ & $b_3 \mapsto 1$ & $b_3 \mapsto 3 $ \\
        $b_4 \mapsto 0$ & $b_4 \mapsto -1$ & $b_4 \mapsto -2$ & $b_4\mapsto -3$ \\
    \end{tabular}
\end{center}
Und damit erhält man dann
\begin{align*}
    \ker \lambda_1 & = \left\langle b_3, b_1 + b_4, b_2 + b_4 \right\rangle \\
    \ker \lambda_2 & = \left\langle b_1 - b_3, b_2 + b_4, 2b_1 - b_2 \right\rangle\\
    \ker \lambda_3 & = \left\langle b_2 + b_4, b_3 + b_4, 3b_1 - b_2 \right\rangle
\end{align*}
Daher folgt $W = \bigcap\limits_{i = 0}^3 \ker \lambda_i = \left\langle b_2 + b_4 \right\rangle$.

Zur Probe kann man noch leicht nachprüfen, dass $V = Z \oplus W$ gilt.

%----------------------------------------------------------------------------------------------------------
%-----------------------------------------------Satz 3.7---------------------------------------------------
%----------------------------------------------------------------------------------------------------------

\begin{mysatz}
    Wenn $m = m_1 \cdot m_2 \cdot \ldots \cdot m_r$ eine Faktorisierung des Minimalpolynoms des Endomorphismus $\alpha$ in paarweise teilerfremde Faktoren ist,
    die alle normiert sind, und $U_i = \ker m_i$ für $i = 1, \ldots, r$ dann gilt:
    \begin{enumerate}
        \item Jedes $U_i$ ist invariant.
        \item $m_i$ ist das Minimalpolynom von $\alpha_{/_U}$.
        \item $V = U_1 \oplus \ldots \oplus U_r$.
    \end{enumerate}

    \textit{Beweis:}
    \begin{enumerate}
        \item Das steht schon in der Bemerkung zu \ref{lemdef3.2}.\ref{lemdef3.2-2}.
        \item Das steht schon in \ref{m=f*g}.
        \item Induktion über $r$. Der Fall $r=1$ ist trivial, denn $\ker m =V$.

            $r=2$ Nach \ref{lem2.13} existieren Polynome $f$ und $g$ mit $f m_1 + g m_2 = 1$.

            Für jedes $v \in V$ gilt daher $v = 1 v = g m_2 v + f m_1 v \in U_1+U_2$, denn $g m_2\in \IM m_2 \leq \ker m_1 = U_1$ nach \ref{m=f*g} und ebenso $fm_1v\in U_2$.

            Wenn $v \in U_1 \cap U_2$, dann ist $v = 1v = g m_2v + f m_1 v = 0$ und damit $V = U_1 \oplus U_2$ gezeigt.

            Für $r>2$ setze $\tilde{m}_1 = m_1 \cdot m_2 \cdot \ldots \cdot m_{r-1}$ und $\tilde{U}_1 = \ker \tilde{m}_1$.

            Verwende den Fall $r = 2$ für $m = \tilde{m}_1 \cdot m_{r}$ sowie Induktion für $\alpha_{/_{\tilde{U}_1}}$.
    \end{enumerate}
\end{mysatz}

%    %----------------------------------------------------------------------------------------------------------
%    %-----------------------------------------------Satz 3.8---------------------------------------------------
%    %---------------------------------------kanonisch rationale Form-------------------------------------------
%
%    \begin{satz}\label{kanonischrationaleForm} \qquad  \par
%        Sei $\alpha$ ein Endomorphismus von $V$, dann existiert eine Basis von $V$ bzgl. der die Abbildungsmatrix die Form
%        \begin{align*}
%            A=\mata A_1&0&\cdots&0 \\ 0&\ddots&&\vdots \\ \vdots&&\ddots&0\\0&\cdots&0&A_t\mate\qquad\mbox{mit }
%            A_i=\mata 0&\cdots&\cdots&0&-k_0^{(i)}\\1&\ddots&&\vdots&\\0&\ddots&\ddots&\vdots&\vdots\\ \vdots&\cdots&1&0&\\0&\cdots&0&1&-k_{n_i-1}^{(i)} \mate
%            \in\K^{n_i\times n_i},
%        \end{align*}
%        wobei
%        \begin{align*}
%            x^{n_i} + \sum_{j=0}^{n_i-1} k_j^{(i)}\cdot x^j = p_i(x)^{s_i}
%        \end{align*}
%        mit $s_i\in \N$ und $p_i\in \K[x]$ irreduzibel ist. Dabei gilt: Wenn $d_i=\deg p_i$, dann ist $n_i=d_i\cdot s_i$ und $\sum_{i=1}^{t}=\dim V$.
%    \end{satz}
%
%    \begin{bew}
%        Faktorisiere das Minimalpolynom in Potenzen von normierten irreduziblen Faktoren, etwa
%        \begin{align*}
%            m=\prod_{j\in I} p_j^{e_j}.
%        \end{align*}
%        Mit $U_j=\ker p_j^{e_j}$ ist dann $V=U_1\oplus \ldots \oplus U_r$, wobei $p_j^{e_j}$ das Minimalpolynom von $\alpha_{|_{U_j}}$ ist. Nach 
%        \ref{Existenz/zyklische/invarianteUR} (und trivialer Induktion über die Dimension) zerfällt jedes $U_j$ in die direkte Summe von zyklischen 
%        Unterräumen. Daher findet man für jeden zyklischen Raum $Z_j$ nach \ref{zyklischeVR}$.1$ eine Basis. Dann ist die Vereinigung der Basen all dieser 
%        zyklischen Unterräume eine Basis von $V$, bzgl. der die Abbildungsmatrix die behauptete Gestalt hat.
%    \end{bew}
%
%    %---------------------------------------------Definition 3.9------------------------------------------------
%    %-----------------------------------------kanonisch rationale Form------------------------------------------
%
%    \begin{mydef} \qquad \par
%        Eine Matrix der Gestalt aus Satz \ref{kanonischrationaleForm} heißt Matrix in kanonisch rationaler Form von $A$.
%    \end{mydef}
%
%    %-----------------------------------------------Lemma 3.9---------------------------------------------------
%    %-----------------------------------------------------------------------------------------------------------
%
%    \begin{lem}\label{lem3.10} \qquad \par
%        Sei A in kanonisch rationaler Form. Dann gelten:
%        \begin{itemize}
%            \item [1.] Für jedes $f\in \K[x]$ ist $f(A)=\mata f(A_1) & & \\ & \ddots & \\ & & f(A_t)\mate$.
%            \item [2.] Wenn $m_i$ das Minimalpolynom von $A_i$ ist, dann ist $m$ das kleinste gemeinsame Vielfacher der $m_i$.
%        \end{itemize}
%    \end{lem}
%
%    \begin{bew}
%        1. Diese Aussage ist trivial.\par\medskip
%        2. Nach 1. ist $f(A)=0$ genau dann, wenn $f(A_i)=$ für jedes $i$ gilt. Nach der Bemerkung zu Definition \ref{minpoly} ist das äquivalent zu der 
%        Aussage, dass $m_i|f$ für jedes $i$. Daraus folgt die Behauptung.
%    \end{bew}
%
%    %----------------------------------------------------------------------------------------------------------
%    %-----------------------------------------------Satz 3.11--------------------------------------------------
%    %--------------------------------------------Cayley-Hamilton-----------------------------------------------
%
%    \begin{satz}\label{CayleyHamilton} \textit{Satz von Cayley-Hamilton} \par
%        \begin{itemize}
%            \item [1.] Das Minimalpolynom ist ein Teiler des charakteristischen Polynoms.
%            \item [2.] $\alpha$ ist Nullstelle des charakteristischen Polynoms.
%            \item [3.] Jeder irreduzible Teiler des charakteristischen Polynom ist auch Teiler des Minimalpolynoms.
%        \end{itemize}
%    \end{satz}
%
%    \begin{bew}
%        Sei A eine Matrix zu $\alpha$ in kanonischer rationaler Form und $m_i$ das Minimalpolynom von $A_i$ wie oben. Dann ist $m$ das kleinste gemeinsame 
%        Vielfache der $m_i$ nach \ref{lem3.10}$.2$. Für jedes $i$ ist $m_i$ zugleich das charakteristische Polynom von $A_i$ nach \ref{zyklischeVR}$.3$. 
%        Daher ist das charakteristische Polynom von $A$ gleich dem Produkt der $m_i$. Daraus folgen alle Behauptungen.
%    \end{bew}
%
%
%    \newpage
%
%
%
%    %----------------------------------------------------------------------------------------------------------
%    %-----------------------------------------------Satz 3.12--------------------------------------------------
%    %----------------------------------------------------------------------------------------------------------
%
%    \begin{satz}\label{anzahlbloeckeKRF} \qquad \par
%        Sei $B$ die Matrix eines Endomorphismus $\alpha$ die ähnlich ist zu einer Matrix der Form 
%        \begin{align*}
%            A=\mata A_1 & & \\ & \ddots & \\ & & A_t\mate.
%        \end{align*} 
%        Dann ist $A$ durch $B$ bis auf die Reihenfolge der $A_i$ eindeutig bestimmt. Sei $p$ ein irreduzibles Polynom und $0<k\in \N$, dann lässt sich die 
%        Anzahl  $z(p,k)$ der Blöcke $A_i$ mit $m_i=p^k$ aus der Formel 
%        \begin{align*}
%            z\cdot \deg p = \RG p^{k-1}(B) + \RG p^{k+1}(B) - 2\RG p^k(B)
%        \end{align*}
%        bestimmen.
%    \end{satz}
%
%    \begin{bew}
%        Die erste Aussage ist nur eine Umformulierung von \ref{kanonischrationaleForm}. Die behauptete Eindeutigkeit von $A$ folgt aus der Formel für $z$. 
%        Es genügt also, diese zu beweisen. Ähnliche Matrizen haben den gleichen Rang. Daher reicht es aus, den Beweis für $A$ zu zeigen. Nach Lemma 
%        \ref{lem3.10} ist für ein $f\in\K[x]$ stets $\RG f(A) = \sum_{i=1}^t \RG f(A_i)$ und daher reicht es aus nur einen Block $A_i$ zu betrachten und 
%        \begin{align*}
%            r_i = \RG p^{k-1}(A_i) + \RG p^{k+1}(A_i)-2\RG p^k(A_i) = \begin{cases} \deg p & m_i=p^k\\0 & \mbox{sonst} \end{cases}
%            \end{align*}
%            zu zeigen.\par
%            Sei also $m_i=q^s$ mit einem irreduziblen Polynom $q$. Wenn $p\neq q$, dann sind $p^k$ und $m_i$ teilerfremd. Nach Lemma \ref{lem2.13} existieren 
%            $f,g\in\K[x]$ mit $1=f\cdot p^k+g\cdot m_i$. Durch Einsetzen von $A_i$ folgt
%            \begin{align*}
%                E=f(A_i)\cdot p^k(A_i) + g(A_i)\cdot m_i(A_i) = f(A_i)\cdot p^k(A_i),
%            \end{align*}
%            denn $m_i(A)=0$. Hierbei ist $E$ die Einheitsmatrix passender Größe. Insbesondere ist $p^k(A_i)$ invertierbar, hat also vollen Rang. Mit derselben 
%            Begründung gilt dies auch für $p^{k-1}(A_i)$ und $p^{k+1}(A_i)$. Da die drei Matrizen den gleichen Rang haben, ist $r_i=0$.\par
%            Sei also $p = q$. Wenn $s<k$, dann ist $s\leq k-1$ und daher $p^{k-1}(A_i)=0$ und erst recht $p^k(A_i)=p^{k+1}(A_i)=0$. Auch in diesem Fall haben 
%            also alle beteiligten Matrizen den Rang (diesmal $0$), und daher ist $r_i=0$.\par
%            Wenn $s=k$, dann ist wieder $p^k(A_i)=p^{k+1}(A_i)=0$. Dagegen ist $\RG p^{k-1}(A_i)=\deg p$ nach \ref{zyklischeVR}. In diesem Fall ist also 
%            $r_i=\deg p$. \par
%            Mit \ref{zyklischeVR} wird auch der letzte Fall diskutiert. Wenn $s>k$, etwa $s=k+t$ dan ist $m_i=p^s=p^k\cdot p^t$ eine Faktorisierung. Demnach ist 
%            also $\RG p^k(A_i)=\deg p^t = t\dot \deg p$. Genauo ist $\RG p^{k+1}(A_i)=\deg p^{t+1} = (t+1)\dot \deg p$ und 
%            $\RG p^{k-1}(A_i)=\deg p^{t-1} = (t-1)\dot \deg p$. Daraus ergibt sich wieder $r_i=0$.
%        \end{bew}
%
%
%
%        \newpage
%
%
%
%        %------------------------------------------------Beispiel---------------------------------------------------
%
%        \begin{bsp} \qquad \par
%            Sei
%            \begin{align*}
%                B=\mata -3&-1&4&-3&-1\\1&1&-1&1&0\\-1&0&2&0&0\\4&1&-4&5&1\\-2&0&2&-2&1 \mate \in \IR^{5 \times 5} 
%            \end{align*}
%            diese Matrix bzgl. der Standardvektoren. Dann ist $\mbox{charpol }B=(x-1)^4\cdot (x-2)$. Dieses besitzt also die beiden irreduziblen Faktoren $p_1(x)=x-1$ und $p_2(x)=x-2$. Dann ist \par\medskip
%            \begin{tabular}{cc}
%                \begin{minipage}{5cm}
%                    \begin{eqnarray*}
%                        \RG (B-E) & = & 3 \\ \RG (B-E)^2 &=&2\\ \RG (B-E)^3 &=& 1 \\ \RG (B-E)^4 &=& 1
%                    \end{eqnarray*}
%                \end{minipage}
%                &
%                \begin{minipage}{5cm}
%                    \begin{tabular}{c|c|c}
%                        $k$  &  $\RG p_1^k(B)$  &  $z\cdot\deg p_1$  \\ \hline
%                        $0$  &       $5$        &         $-$                            \\ 
%                        $1$  &       $3$        &         $1$                            \\ 
%                        $2$  &       $2$        &         $0$                            \\ 
%                        $3$  &       $1$        &         $1$                            \\ 
%                        $4$  &       $1$        &         $0$                            \\ 
%                    \end{tabular}
%                \end{minipage}
%            \end{tabular} \par \medskip 
%
%            Die Einträge in der letzten Spalte wurden mit der Formel aus Satz \ref{anzahlbloeckeKRF} errechnet. Die Ränge von $p_1^k(B)$ wurden so lange 
%            bestimmt, bis sich der Rang für ein bestimmtes $k$ nicht mehr ändert. 
%
%
%            Aus der Tabelle entnimmt man, dass die kanonisch rationale Form der Matrix $B$ einen $1$-er Block zu $m_{A_1}=x-1$ und einen $3$-er Block zu 
%            $m_{A_2}=(x-1)^3=x^3-3x^2+3x-1$ besitzt. Zusätzlich kommt dann ein $1$-er Block zu $m_{A_3}=x-2$ dazu. Daher hat die kanonisch rationale Form der 
%            Matrix $B$ die folgende Gestalt.
%            \begin{align*}
%                A=\mata 1&&&& \\ &0&0&1&\\ &1&0&-3& \\ &0&1&3& \\ &&&&2 \mate
%            \end{align*}
%            Ist man zusätzlich an der Transformationmatrix interessiert ($A$ und $B$ sind ähnlich, d.h. $A=T^{-1}BT$), so muss man die verallgemeinerten 
%            Eigenräume zu den Blöcken betrachten. Es ist (zum $3$-er Block)
%            \begin{align*}
%                \ker (B-E)^2 = \left\langle \mata0\\0\\1\\1\\0\mate,\mata1\\0\\1\\0\\0\mate,\mata0\\1\\0\\0\\-1\mate \right\rangle \qquad \mbox{und} \qquad
%                \ker (B-E)^3 = \left\langle \mata0\\0\\0\\0\\1\mate,\mata0\\0\\1\\1\\0\mate,\mata1\\0\\1\\0\\ 0\mate,\mata0\\1\\0\\0\\0\mate\right\rangle
%            \end{align*}
%            Nun wählt man ein Element $b_2$ aus dem Kern von $(B-E)^3$, welches nicht im Kern von $(B-E)^2$ liegt, also $b_2=(0,0,0,0,1)^T$. Dann erzeugt man 
%            ausgehend von diesem Element den Invarianten Unterraum, also
%            \begin{align*}
%                b_2=\mata0\\0\\0\\0\\1\mate \qquad \alpha(b_2) =b_3= \mata-1\\0\\0\\1\\1\mate \qquad \alpha^2(v)=b_4=\mata-1\\0\\1\\2\\1\mate
%            \end{align*}
%            Zu den beiden $1$-er Blöcken und den dazugehörigen Eigenwerten müssen die entsprechenden Eigenvektoren berechnet werden, also
%            \begin{align*}
%                b_1=\mata1\\0\\1\\0\\0\mate\mbox{ ist EV zum EW }\lambda_1=1 \mbox{ und } b_5=\mata0\\1\\2\\3\\-2\mate \mbox{ ist EV zum EW }\lambda_2=2
%            \end{align*}
%            Damit hat man die Transformationsmatrix $T=(b_1,b_2,b_3,b_4,b_5)$ gefunden, mit der man zur Probe $TA = TB$ durchrechnen kann.
%
%        \end{bsp}
%
%        %------------------------------------------------Beispiel---------------------------------------------------
%
%        \begin{bsp}
%            Sei
%            \begin{align*}
%                B=\mata3&-1&0&0&-2\\1&3&1&0&0\\0&1&3&0&2\\1&0&1&3&1\\0&0&0&0&3\mate
%            \end{align*}
%            wieder bzgl. der Standardbasis. Dann ist $p(x)=(x-3)^5$ das charakteristische Polynom von $B$.\par \medskip
%            \begin{tabular}{cc}
%                \begin{minipage}{5cm}
%                    \begin{eqnarray*}
%                        \RG (B-3E) &=& 3\\ \RG (B-3E)^2=1 \\ \RG (B-3E)^3 &=&0
%                    \end{eqnarray*}             
%                \end{minipage}
%                &
%                \begin{minipage}{5cm}
%                    \begin{tabular}{c|c|c}
%                        $k$   &   $\RG p^k(B)$   &   $z\cdot \deg p$ \\ \hline
%                        $0$   &       $5$        &       $-$                             \\
%                        $1$   &       $3$        &       $0$                             \\
%                        $2$   &       $1$        &       $1$                             \\
%                        $3$   &       $0$        &       $1$                             \\
%                    \end{tabular}
%                \end{minipage}
%            \end{tabular} \par \medskip
%            Aus der Tabelle erkennt man, dass $m(x)=m_1(x)=(x-3)^3$ das Minimalpolynom der Matrix $B$ ist. Also muss es mindestens einen $3$-er Block in der 
%            kanonisch rationalen Form der Matrix $B$ geben. Zusätzlich existiert dann noch ein $2$-er Block zu dessen Minimalpolynom $m_2(x)=(x-3)^2$. Die 
%            kanonisch rationale Form der Matrix hat also die folgende Gestalt.
%            \begin{align*}
%                A=\mata0&0&27&&\\1&0&-27&&\\0&1&9&&\\&&&0&-9\\&&&1&6\mate
%            \end{align*}
%            Sei nun noch die Transformationsmatrix gesucht, dann geht man wieder blockweise vor. Für den Block der Größe $3$ sucht man sich ein Element 
%            $b_1\in\ker(B-3E)^3\backslash\ker(B-3E)^2$ und bildet den dazugehörigen zyklischen Unterraum. Zu dem Block der Größe $2$ sucht man sich ein Element 
%            $b_4\in\ker(B-3E)^2\backslash\ker(B-3E)$ und bildet auch hier den dazugehörigen zyklischen Unterraum. Es ist
%            \begin{eqnarray*}
%                \ker(B-3E)^3&=& \IR^5\qquad\ker(B-3E)^2=\left\langle\mata0\\1\\0\\0\\0\mate,\mata0\\0\\0\\1\\0\mate,\mata0\\0\\0\\0\\1\mate,\mata1\\0\\-1\\0\\0\mate\right\rangle \\
%                \ker(B-3E)  &=& \left\langle \mata1\\0\\-1\\0\\0\mate,\mata0\\0\\0\\1\\0\mate \right\rangle
%            \end{eqnarray*}
%            Daher ist (lineare Unabhängigkeit bei der Auswahl beachten)
%            \begin{eqnarray*}
%                b_1 &=& \mata1\\0\\0\\0\\0\mate \qquad \alpha(b_1)=b_2=\mata3\\1\\0\\1\\0\mate \qquad \alpha^2(b_1)=b_3=\mata8\\6\\1\\6\\0\mate \\
%                b_4 &=& \mata0\\0\\0\\0\\1\mate \qquad \alpha(b_4)=b_5=\mata-2\\0\\2\\1\\3\mate
%            \end{eqnarray*}
%            Und man erhält somit schließlich $T=(b_1,b_2,b_3,b_4,b_5)$ die Transformationsmatrix.
%        \end{bsp}
%
%        %------------------------------------------------Bemerkung--------------------------------------------------
%
%        \begin{bem}\qquad \par
%            Sei $\alpha$ ein Endomorphismus von $V$ mit Minimalpolynom $m$ über $\K$ und $\K$ algebraisch abgeschlossen, d.h. alle Polynome aus $\K[x]$ 
%            zerfallen vollständig in Linearfaktoren.
%        \end{bem}
%
%        %----------------------------------------------------------------------------------------------------------
%        %-----------------------------------------------Satz 3.13--------------------------------------------------
%        %----------------------------------------------------------------------------------------------------------
%
%        \begin{satz}\label{jordanscheNormalform} \textit{Jordansche Normalform}\par
%            Bezüglich einer geeigneten Basis von $V$ hat $A$ die Gestalt
%            \begin{align*}
%                A=\mata A_1 & &0\\ & \ddots & \\ 0& & A_t \mate \qquad \mbox{ mit } A_i=\mata \lambda_i&&&0\\1&\ddots&&\\&\ddots&\ddots&\\0&&1&\lambda_i\mate.
%            \end{align*}
%            Dabei heißt $A$ die Jordansche Normalform und die $A_i$ werden als Jordankästchen bezeichnet. $A$ ist bis auf die Reihenfolge der $A_i$ eindeutig 
%            bestimmt. 
%        \end{satz}
%
%        \begin{bew}
%            Existenz einer geeigneten Basis: Nach dem Beweis von \ref{kanonischrationaleForm} existiert eine Zerlegung \par $V=Z_1\oplus\cdots\oplus Z_t$ mit 
%            zyklischen invarianten Unterräumen, deren Minimalpolynome $m_i$ Potenzen von irreduziblen Teilern (nach \ref{CayleyHamilton}) des charakteristischen 
%            Polynoms von $\alpha$ sind. Nun muss für jedes $Z_i$ eine Basis gefunden werden, sodass die Matrix von $\alpha_{|_{Z_i}}$ die Gestalt eines 
%            Jordan-Kästchens hat. Daher darf man annehmen, dass $V=Z_1$ zyklisch ist und dass $m=(x-\lambda)^n$ für geeignetes $\lambda \in \K$ gilt.
%            Daher gibt es nach \ref{zyklischeVR} eine Basis $v,\alpha(v),\ldots,\alpha^{n-1}(v)$.\par
%            Da $(\alpha-\lambda)^n$ das normierte Polynom kleinsten Grades ist mit $(\alpha-\lambda)^nv = 0$ ist, sind die Vektoren 
%            $b_1=v, b_2=(\alpha-\lambda)(v),\ldots,b_n=(\alpha-\lambda)^{n-1}(v)$ linear unabhängig, bilden also eine Basis. Aus $(\alpha-\lambda)(b_k)=b_{k+1}$ 
%            (für $k<n$) bzw. $(\alpha-\lambda)(b_n)=(\alpha-\lambda)^n(v) = 0$ folgt $\alpha(b_k) = b_{k+1} + \lambda b_k$ (für $k<n$) bzw. 
%            $\alpha(b_n)= \lambda b_n$. Bezüglich dieser Basis hat $\alpha$ also die angegebene Form. \par
%            Die Eindeutigkeit folgt aus der folgenden Formel:
%            \begin{align*}
%                z= \RG (\alpha-\lambda)^{k-1} + \RG (\alpha-\lambda)^{k+1} - 2\cdot \RG (\alpha-\lambda)^{k}
%            \end{align*}
%            denn das ist genau die Formel aus \ref{kanonischrationaleForm} mit $\deg p=1$.
%        \end{bew}
%
%        %------------------------------------------------Bemerkung--------------------------------------------------
%
%        \begin{bem}\qquad \par
%            \begin{itemize}
%                \item [1.]Um die Jordansche Normalform zu berechnen, müssen zu jedem Eigenwert $\lambda$ ide Ränge von $(\alpha-\lambda)^s$ bestimmt werden.
%                \item [2.]Eine Basis zu jedem Block der Größe $k$ erhält man durch
%                    \begin{align*}
%                        v\in \ker(\alpha-\lambda)^k\backslash \ker(\alpha-\lambda)^{k-1} \\ 
%                        \Rightarrow \left\{ v,(\alpha-\lambda)v,\ldots,(\alpha-\lambda)^{k-1},\ldots\right\}
%                    \end{align*}
%                    Insbesondere ist $\ker(\alpha-\lambda)\leq \ker(\alpha-\lambda)^2\leq \ldots \leq \ker(\alpha-\lambda)^k=\ker(\alpha-\lambda)^{k+1}$
%            \end{itemize}
%        \end{bem}
%
%        %--------------------------------------------Definition 3.14-----------------------------------------------
%        %-----------------------------------------------nilpotent--------------------------------------------------
%
%        \begin{mydef}\textit{nilpotent} \par
%            $\alpha$ heißt genau dann nilpotent, wenn $\exists k\in \N$ mit $\alpha^k=0$
%        \end{mydef}
%
%        %-----------------------------------------------Bemerkung--------------------------------------------------
%
%        \begin{bem}\qquad \par
%            \begin{itemize}
%                \item [1.]Der Enomorphismus $\alpha$ ist genau dann nilpotent, wenn $x^n$ das charakteristische Polynom von $\alpha$ ist.
%                \item [2.]Das Minimalpolynom entspricht dann dem charakteristischen Polynom.
%            \end{itemize}
%        \end{bem}
%
%        \newpage


\section{Orthogonale Räume und orthogonale Abbildungen}

%---------------------------------------Definition 4.1--------------------------------------------
%----------------------------------------Bilinearform---------------------------------------------

\begin{mydef} \label{Bilinearform} \textit{Bilinearform}

    Sei $V$ ein $\K$-VR, dann heißt die Abbildung $(\cdot , \cdot ):V \times V \mapsto \R$ Bilinearform genau dann, wenn gilt
    \begin{enumerate}
        \item $(v, w_1 + w_2) = (v, w_1) + (v, w_2) \ \forall v, w_1, w_2 \in V$

            $(v_1+v_2,w)=(v_1,w) + (v_2,w) \ \forall v_1,v_2,w \in V$
        \item $(k v,w) = k(v,w) = (v, k w) \ \forall v, w \in V, \forall k \in \K$

            Ist zusätzlich noch
        \item $(v,w) = (w,v) \ \forall v ,w \in V$
    \end{enumerate}
    erfüllt, dann heißt die Bilinearform symmetrisch.
\end{mydef}

%----------------------------------------Beispiel-------------------------------------------------
\textit{Beispiel:}

Sei $V = \K^n$
\begin{itemize}
    \item $(v,w) \mapsto v^T \cdot w \in \K$
\end{itemize}
Sei $V = \R^3$
\begin{itemize}
    \item $(v,w) \mapsto v_1 w_1 + v_2 w_2 - v_3 w_3$
\end{itemize}
Sei $V = \R[x]$
\begin{itemize}
    \item $(p,q) \mapsto \int\limits_{-1}^1 p(x) \cdot q(x) \ dx$
\end{itemize}
Sei $V=\F_2[x]$
\begin{itemize}
    \item $(p,q) \mapsto \sum\limits_{i \in I} a_i \cdot b_i$, wobei $p = \sum\limits_{i = 0}^r a_i \cdot x^i$ und $p = \sum\limits_{i = 0}^s b_i \cdot x^i$
\end{itemize}

%----------------------------------------Bemerkung-------------------------------------------------

\begin{mydef}\label{gramscheMatrix}

    Sei $\dim V = n$ und $\{ b_1, \ldots, b_n \}$ eine Basis von $V$ und $(\cdot ,\cdot )$ eine Bilinearform auf $V$.
    Desweiteren sei $(b_i, b_j)$ für alle $i, j \in \{1, \ldots, n \}$ gegeben.
    Für beliebige $v, w \in V$ ist dann
    \begin{align*}
        (v,w)   & = \left( \sum\limits_{i = 1}^n v_i b_i, \sum\limits_{j = 1}^n w_j b_j \right) = \sum\limits_{i = 1}^n \left(v_i b_i ,\sum\limits_{j = 1}^n w_j b_j \right) \\
        & = \sum\limits_{i, j} v_i(b_i, b_j) w_j
    \end{align*}
    Man nennt die Matrix
    \begin{align*}
        G =
        \begin{pmatrix}
            (b_1,b_1) & \cdots & (b_1,b_n) \\
            \vdots & & \vdots \\
            (b_n,b_1) & \cdots & (b_n,b_n)
        \end{pmatrix}
    \end{align*}
    die Gramsche Matrix der Bilineaform $(\cdot , \cdot )$ bzgl. der Basis B.
\end{mydef}
\textit{Bemerkung:}
\begin{enumerate}
    \item \label{gramscheMatrix-1} Offensichtlich ist
        \begin{align*}
            (v,w) & = v^T \cdot G \cdot w = (v_1, \ldots, v_n) \cdot G \cdot
            \begin{pmatrix}
                w_1\\ \vdots \\ w_n
            \end{pmatrix}
        \end{align*}
    \item \label{gramscheMatrix-2} Die Bilinearform $(\ \cdot \ , \ \cdot \ )$ ist genau dann symmetrisch, wenn $G$ symmetrisch ist.
    \item \label{gramscheMatrix-3} Sei $A$ eine beliebige Matrix aus $\K^{n \times n}$, dann ist durch $(v,w) := v^T \cdot A \cdot w$ eine Bilinearform auf $V = \K^n$ definiert.
\end{enumerate}

%---------------------------------------Definition 4.2--------------------------------------------
%-----------------------------------orthogonal, ausgeartet----------------------------------------

\begin{mydef} \label{orthogonal,ausgeartet} \textit{orthogonal, ausgeartet}

    Sei $(\cdot , \cdot )$ eine Bilinearform auf $V$.
    Die Vektoren $v,w\in V$ heißen orthogonal zueinander, wenn 
    \begin{align*}
        (v,w) & = 0
    \end{align*}
    gilt.
    Eine Bilinearform heißt genau dann nicht ausgeartet, wenn der Nullvektor der einzige Vektor ist, derzu allen Vektoren orthogonal ist.
    \begin{align*}
        (v,w) & = 0 \ \forall w \Rightarrow v = 0.
    \end{align*}
\end{mydef}

%------------------------------------------Lemma 4.3----------------------------------------------
%--------------------------------------TODO!!!----------------------------------------------------

\begin{mylemma}

    Sei $(\cdot , \cdot )$ eine Bilinearform mit der Gramschen Matrix $G$ bzgl. einer Basis $\{ b_1, \ldots, b_n \}$ von V.
    Dann ist $(\cdot , \cdot)$ genau dann nicht ausgeartet, wenn die Gramsche Matrix regulär ist.\par\medskip

    \textit{Beweis:}
    \begin{itemize}
        \item[,,$\Leftarrow$'']
        \item[,,$\Rightarrow$'']
    \end{itemize}
\end{mylemma}

%---------------------------------------Definition 4.4--------------------------------------------
%-----------------------------------orthogonales Komplement---------------------------------------

\begin{mydef} \label{orthogonalesKomplemen} \textit{orthogonales Komplement}

    Sei $X$ eine beliebige Teilmenge eines VR $V$, auf der die Bilinearform $(\cdot , \cdot)$ definiert ist.
    Man nennt die Menge
    \begin{align*}
        X^{\perp} & = \left\{ v \in V \mid (v,x) = 0 \quad \forall x \in X \right\}
    \end{align*}
    das orthogonale Komplement zu $X$. 
    Analog ist $^{\perp}X$ definiert.
\end{mydef}

%----------------------------------------------------------------------------------------------------------
%-----------------------------------------------Satz 4.5---------------------------------------------------
%----------------------------------------------------------------------------------------------------------

\begin{mysatz}\label{eigenschaftenorthKomplement}

    Sei $V$ ein VR mit der Bilinearform $(\cdot , \cdot)$ und $X \subseteq V$.
    Dann gilt
    \begin{enumerate}
        \item $X^{\perp}$ ist ein Unterraum von $V$.
        \item $X \subseteq Y$, dann ist $Y^{\perp} \subseteq X^{\perp}$
        \item $X^{\perp} = \left\langle X \right\rangle^{\perp}$
        \item \label{EigOrthKompl4} $X \subseteq X^{\perp\perp}$
    \end{enumerate}
\end{mysatz}
\textit{Beweis:} Übungsaufgabe

%------------------------------------------Lemma 4.6----------------------------------------------
%---------------------------------------Dimensionsformel------------------------------------------

\begin{mylemma} \label{dimensionsformelorthKomplement} \textit{Dimensionsformel}

    Sei $V$ ein $n-$dimensionaler VR, $(\cdot , \cdot )$ eine nicht ausgeartete Bilinearform auf $V$, so gilt für alle Unterräume $U$ von $V$
    \begin{align*}
        \dim U^{\perp} & = n - \dim U
    \end{align*}
    \textit{Beweis:}

    Sei $\{ b_1, \ldots, b_m, \ldots, b_n \}$ eine Basis von $V$ und $\{ b_1, \ldots, b_m \}$ eine Basis von $U$.
    Sei $x \in U^{\perp}$.
    Dann gilt für alle $j = 1, \ldots, m$, dass $(x, b_j) = 0$.
    Nun lässt sich $x$ als Linearkombination der Basiselemente von $V$ darstellen, etwa $\displaystyle{x=\sum_{i=1}^n k_i\cdot b_i}$.

    Dann folgt
    \begin{align*}
        0 & = (x, b_j) = \sum\limits_{i = 1}^n k_i \cdot (b_i, b_j) \ \forall j = 1, \ldots, m
    \end{align*}
    Das ist ein homogenes lineares Gleichungssystem mit $m$ Gleichungen und $n$ Unbekannten ($k_1, \ldots, k_n$).
    In der Koeffizientenmatrix $G'$ stehen die ersten $m$ Zeilen der Gramschen Matrix $G$.
    Dann hat $G'$ den Rang $n$, weil $G$ regulär ist.
    Also existiert eine $n-m$ parametrige Lösung des Gleichungssystem und daher ist $\dim U^{\perp} = \dim V - \dim U = n-m$.
\end{mylemma}

%----------------------------------------Bemerkung-------------------------------------------------
\textit{Bemerkung:}

Sei $\dim V = n$ und die Bilinearform $(\cdot , \cdot)$ nicht ausgeartet.
Wenn $U = U^{\perp \perp}$ (möglich nach \ref{eigenschaftenorthKomplement}.\ref{EigOrthKompl4}) dann ist $\dim U = \dim U^{\perp\perp}$

%---------------------------------------Definition 4.7--------------------------------------------
%---------------------------------------Skalarprodukte--------------------------------------------

\begin{mydef}\label{skalarprodukte} \textit{Skalarprodukt}

    Eine Bilinearform $\left\langle \cdot , \cdot \right\rangle$ auf dem Vektorraum $V$ heißt genau dann Skalaprodukt,
    wenn sie symmetrisch und positiv definit ist, d.h.
    \begin{align*}
        \forall v\in V : \ \left\langle v, v \right\rangle \geq 0 \text{ und } \left\langle v, v \right\rangle = 0 \Leftrightarrow v = 0
    \end{align*}
\end{mydef}

%----------------------------------------Bemerkung-------------------------------------------------
\textit{Bemerkung:}

Sei auf $V$ ein Skalarprodukt $\left\langle \cdot , \cdot \right\rangle$  definiert.
Dann gilt:
\begin{enumerate}
    \item $\left\langle \cdot ,  \cdot \right\rangle$ ist nicht ausgeartet.
    \item Sei $U$ ein beliebiger Unterraum von $V$. Dann ist $U \cap U^{\perp} = \{ 0 \} \Rightarrow V = U \oplus U^{\perp}$
\end{enumerate}

%----------------------------------------Beispiel-------------------------------------------------

\textit{Beispiel:}

Sei $V = \R^3$. Als Basis wählen wir die Standardbasis. Sei nun $\left\langle \cdot , \cdot \right\rangle$ definiert durch
\begin{align*}
    \left\langle
    \begin{pmatrix}
        x_1\\y_1\\z_1
    \end{pmatrix},
    \begin{pmatrix}
        x_2\\y_2\\z_2
    \end{pmatrix}
    \right\rangle = 3 x_1 x_2 + 2 y_1 y_2 + 2 z_1 z_2 - x_1 y_2 - y_1 z_2 - z_1 y_2 - x_2 y_1 - y_2 z_1 - x_1 z_2
\end{align*}
Nach \ref{gramscheMatrix} hat die Gramsche Matrix der Bilinearform die Gestalt:
\begin{align*}
    G =
    \begin{pmatrix}
        3 & -1 & -1\\
        -1 & 2 & -1\\
        -1 & -1 & 2
    \end{pmatrix}
\end{align*}
Da $G$ symmetrisch ist, ist auch die Bilinearform symmetrisch (vgl. \ref{gramscheMatrix}.\ref{gramscheMatrix-2}).

$(\cdot , \cdot)$ ist nicht ausgeartet, da $G$ regulär ist.

Bleibt also nur noch die positive Definitheit zu überprüfen.
Dazu betrachtet man $(x, x)$ mit $x = (x, y, z)^T$, was nach \ref{gramscheMatrix}.\ref{gramscheMatrix-1} gegeben ist durch
\begin{align*}
    \left\langle
    \begin{pmatrix}
        x\\y\\z
    \end{pmatrix},
    \begin{pmatrix}
        x\\y\\z
    \end{pmatrix}
    \right\rangle & = (x,y,z) \cdot G \cdot
    \begin{pmatrix}
        x\\y\\z
    \end{pmatrix} = 3x^2+2y^2+2z^2-2xy-2xz-2yz \\
    & = (x - y)^2 + (x - z)^2 + (y - z)^2 + x^2 \geq 0
\end{align*}
Und $\left\langle x, x \right\rangle = 0$ genau dann wenn $x = 0$, was man leicht überprüft.
Daher ist $\left\langle \cdot , \cdot \right\rangle$ ein Skalarprodukt.\\

%----------------------------------------Beispiel-------------------------------------------------
\textit{Beispiel:}

Sei $V = \mathcal{C}[a,b]$, der Raum der stetigen Funktionen auf dem Intervall $[a,b] \subset \R$.
Auf diesem Raum sei $\left\langle \cdot , \cdot \right\rangle$ definiert durch
\begin{align*}
    \left\langle f,g \right\rangle & := \int\limits_a^b f(x) \cdot g(x) \ dx
\end{align*}
Die Eigenschaften des Integrals übertragen sich auf die Eigenschaften von $\left\langle \cdot ,  \cdot \right\rangle$.
Damit ist $\left\langle \cdot , \cdot \right\rangle$ zumindestens schon mal eine symmetrische Bilineaform.
Für die positive Definitheit muss das Integral so gewählt werden, dass $\left\langle f, f \right\rangle > 0$ ist.
Wenn $\left\langle f, f \right\rangle = 0$, so folgt aufgrund der Stetigkeit von $f$, dass die Funktion $f = 0$ ist.

%---------------------------------------Definition 4.9--------------------------------------------
%--------------------------------------------Norm-------------------------------------------------

\begin{mydef} \label{norm} \textit{Norm}

    Sei $V$ ein VR über $\R$. Dann heißt eine Abbildung $\| \cdot \|: V \mapsto \R^{+}$ Norm, falls folgende Eigenschaften erfüllt sind:
    \begin{enumerate}
        \item $\| v \| \geq 0 \quad \Rightarrow \quad \| v \| = 0 \Leftrightarrow v = 0 \qquad \forall v \in V$
        \item $\| \lambda v\| = |\lambda| \cdot \| v \| \qquad \forall \lambda \in \R, \ \forall v \in V$
        \item $\| v + w \| \leq \| v \| + \| w \| \qquad \forall v,w\in V$.
    \end{enumerate}
\end{mydef}

%----------------------------------------Beispiel-------------------------------------------------

\textit{Beispiel:}
\begin{enumerate}
    \item   $V = \R^n$ mit $\| v \|_{\infty} =\max\limits_{i} | x_i |$, wobei $v = (x_1, \ldots, x_n)^T$
        $V = \R^n$ mit $\| v \|_{1} = | x_1| + \ldots + | x_n | = \sum\limits_{i = 1}^n | x_i |$
    \item $V = \R^{n \times n}$ mit $\| A \| = \max\limits_{v \in V \setminus\{ 0 \}} \frac{\| Av \|}{\| v \|} = \max\limits_{\| v\| = 1} \| Av \|$
\end{enumerate}

%----------------------------------------------------------------------------------------------------------
%-----------------------------------------------Satz 4.10--------------------------------------------------
%--------------------------------------------euklidische Norm----------------------------------------------

\begin{mysatz}\label{euklidischeNorm} \textit{Euklidische Norm}

    Durch $\| v \| = \sqrt{\left\langle v, v \right\rangle}$ wird in einem euklidischen Vektorraum $V$ (reellwertiger Vektorraum mit Skalaprodukt) eine Norm definiert.

    \textit{Beweis:}

    Die ersten beiden Normeigenschaften folgen unmittelbar. Die Dreiecksungleichung folgt mit Hilfe des nächsten Satzes.
\end{mysatz}

%----------------------------------------------------------------------------------------------------------
%-----------------------------------------------Satz 4.11--------------------------------------------------
%--------------------------------------Cauchy-Schwarzsche-Ungleichung--------------------------------------

\begin{mysatz} \label{CSU} \textit{Cauchy-Schwarzsche-Ungleichung}

    Sei $V$ ein euklidischer Vektorraum. Dann gilt
    \begin{align*}
        | \left\langle v,w \right\rangle | & \leq \| v \| \cdot \| w \|
        \intertext{oder in äquvalenter Weise}
        \left\langle v,w \right\rangle^2 & \leq \left\langle v,v \right\rangle \cdot \left\langle w,w \right\rangle \qquad \forall\ v,w\in V
    \end{align*}

    \textit{Beweis:}

    Wenn $w=0$ oder $v=0$, dann ist die Aussage trivial.

    Daher sei $v,w \neq 0$. Dann gilt für jedes $k \in \R$:
    \begin{align*}
        0 & \leq \left\langle v - kw, v - kw \right\rangle = \left\langle v,v \right\rangle + k^2 \left\langle w,w \right\rangle - 2 k \left\langle v,w \right\rangle
    \end{align*}
    Also insbesondere auch für $k = \frac{\left\langle v,w \right\rangle}{\left\langle w,w \right\rangle}$. Dann ist nämlich
    \begin{align*}
        0 & \leq \left\langle v,v \right\rangle + \frac{\left\langle v,w \right\rangle^2}{\left\langle w,w \right\rangle^2} \left\langle w,w \right\rangle - 2 \frac{\left\langle v,w \right\rangle}{\left\langle w,w \right\rangle} \left\langle v,w \right\rangle = \left\langle v,v \right\rangle - \frac{\left\langle v,w \right\rangle^2}{\left\langle w,w \right\rangle}
    \end{align*}
    und der Satz ist bewiesen.
\end{mysatz}

%----------------------------------------Bemerkung-------------------------------------------------

\textit{Bemerkung:}

Wegen \ref{CSU} gilt
\begin{align*}
    | \left\langle v,w \right\rangle | & \leq  \| v \| \cdot \| w \|
    \intertext{Umstellen liefert:}
    -1 & \leq \frac{ \left\langle v,w \right\rangle}{\| v \| \cdot \| w \|}  \leq  1
    \intertext{Daher existiert im Fall $v, w \neq 0$ genau ein $\varphi \in [0,\pi]$ mit}
    \cos(\varphi) & = \frac{ \left\langle v,w \right\rangle}{\| v \| \cdot \| w \|}
\end{align*}
Dies entspricht der Definition des Winkels zwischen $2$ Vektoren.\\

%----------------------------------------Beispiel-------------------------------------------------

\textit{Beispiel:}

Im $\R^2,\R^3$ mit dem Standardskalarprodukt entspricht $\| v \|$ der Länge und $\varphi$ dem Winkel zwischen den Vektoren $v,w$ im elementargeometrischen Sinn.

\begin{center}
    \begin{tikzpicture}
        \draw[>=latex,->] (0,0) -- (10,0);% Vektor v
        \draw (7,0) node[anchor=north] {$v$};
        \draw[>=latex,->] (0,0) -- (4,3);% Vektor w
        \draw (2,1.5) node[anchor=north west] {$w$};
        \draw (4,3) -- (6.4,4.8);% Strecke q
        \draw (6.4,4.8) -- (10,0);
        \draw (4,0) -- (4,3);
        \draw (4,0.3) arc (90:180:0.3);
        \draw (3.9,0.1) node {$\cdot$};
        \draw (6.16,4.62) arc(210:305:0.3);
        \draw (6.4,4.6) node {$\cdot$};
        \draw (1.8,0) arc (0:37:1.8);% Winkel alpha
        \draw (1.1,0.3) node {$\sphericalangle (v,w)$};
        \draw [gray,decorate,decoration={brace,amplitude=5pt},yshift=-3pt] (4,0)  -- (0,0) node [black,midway,below=4pt] {$p$};
        \draw [gray,decorate,decoration={brace,amplitude=5pt},yshift=2pt,xshift=-2pt] (0,0)  -- (6.4,4.8) node [black,midway,anchor=south east,yshift=1pt,xshift=-1pt] {$q$};
    \end{tikzpicture}
\end{center}
\begin{align*}
    \left.
    \begin{array}{c}
        \displaystyle{ \cos \sphericalangle (v,w) = \frac{p}{\| w \|} } \\
        \displaystyle{ \cos \sphericalangle (v,w) = \frac{q}{\| v \|} }
    \end{array}
    \right\}
    & \Rightarrow \left\langle v,w \right\rangle = q \cdot \| w \| = p \cdot \| v \| \qquad \text{ und damit }\\
    \cos \sphericalangle (v,w) & = \frac{\left\langle v,w \right\rangle}{\| v \| \cdot \| w \|}
\end{align*}
Das Skalarprodukt von $v$ und $w$ ist die Länge von $v$ mal der Länge der Projektion von $w$ auf $v$.
Desweiteren sind zwei Vektoren orthogonal zueinander, wenn das Skalarprodukt verschwindet.
Dies gilt für $\varphi = \pi/2$, aber auch, wenn einer der beiden Vektoren der Nullvektor ist.

%---------------------------------------Definition 4.12-------------------------------------------
%------------------------------------orthogonal,orthonormal---------------------------------------

\begin{mydef}\label{orthogonal,orthonormal}\textit{orthogonal, orthonormal}

    Eine Menge $\{ v_1, \ldots, v_n \}$ von Vektoren in einem euklidischen Vektorraum $V$ heißt genau dann orthogonal,
    wenn $\left\langle v_i,v_j \right\rangle = 0$ für alle $i = j \in \{ 1, \ldots, n \}$ und orthonormal, wenn $\left\langle v_i, v_j \right\rangle = \delta_{ij}$ ist.
\end{mydef}

%------------------------------------------Lemma 4.13---------------------------------------------
%-------------------------------------------------------------------------------------------------

\begin{mylemma}
    Eine orthogonale Menge, die den Nullvektor nicht enthält, ist linear unabhängig.

    \textit{Beweis:}

    Sei $\{ v_1, \ldots, v_n \}$ die orthogonale Menge.
    Die Vektoren $v_1, \ldots, v_n$ sind genau dann linear unabhängig, wenn aus $\sum\limits_{i = 1}^n k_i \cdot v_i = 0$ folgt, dass die Koeffizienten $k_i$ alle $0$ sind.
    Da die Vektoren orthogonal zueinander sind, folgt
    \begin{align*}
        0 & = \left\langle \sum\limits_{i = 1}^n k_i \cdot v_i , v_j \right\rangle = k_j \cdot \left\langle \underbrace{v_j, v_j}_{ \neq 0} \right\rangle
    \end{align*}
    woraus folgt, dass der Koeffizienten $k_j=0$ ist.
\end{mylemma}

%----------------------------------------Bemerkung-------------------------------------------------

\textit{Bemerkung:}

Orthonormale Mengen sind stets linear unabhängig, solange der Nullvektor nicht in der Menge enthalten sind.

%----------------------------------------------------------------------------------------------------------
%-----------------------------------------------Satz 4.14--------------------------------------------------
%--------------------------------------Orthonormalisierungsverfahren---------------------------------------

\begin{mysatz} \textit{Orthonormalisierungsverfahren nach 
    \textsc{Gram}\footnote{J\o rgen Pedersen Gram (* 27. Juni 1850 in Nustrup; $\dagger$ 29. April 1916 in Kopenhagen), dänischer Mathematiker}
    und
    \textsc{Schmidt}\footnote{Erhard Schmidt (* 13. Januar 1876 in Dorpat (heutiges Tartu, Estland); $\dagger$ 6. Dezember 1959 in Berlin), dt. Mathematiker}}

    Jeder endlich dimensionale euklidische Vektorraum besitzt eine Orthonormalbasis, kurz ON-Basis.\medskip

    \textit{Beweis:} Induktion über die Dimension $n$.

    \begin{itemize}
        \item $n = 1$: Sei $V=\left\langle b \right\rangle$. Dann ist $\left\{ e = \frac{b}{\| b \|} \right\}$ eine ON-Basis.
        \item $n - 1 \rightarrow n$: Sei dazu $\{ e_1, \ldots, e_{n-1} \}$ eine ON-Basis von $U \subset V$.
            Dann kann man mit einem geeigenten $b_n\in V$ die Menge $\{ e_1, \ldots, e_{n-1}, b_n \}$ zu einer Basis von $V$ ergänzen.
            Setze nun
            \begin{align*}
                e_n' & = \left( - \sum\limits_{i = 1}^{n-1} \frac{ \left\langle e_i, b_n \right\rangle }{ \left\langle e_i, e_i \right\rangle } \cdot e_i \right) + b_n
            \end{align*}
            Dann folgt für alle $k = 1, \ldots, n-1$
            \begin{align*}
                \left\langle e_n', e_k \right\rangle & = - \sum_{i = 1}^{n-1} \frac{ \left\langle e_i, b_n \right\rangle }{ \left\langle e_i,e_i \right\rangle } \cdot 
                \left\langle e_i,e_k \right\rangle + \left\langle b_n,e_k \right\rangle \\
                & = - \frac{ \left\langle e_k,b_n \right\rangle }{ \left\langle e_k,e_k \right\rangle } \cdot \left\langle e_k,e_k \right\rangle + \left\langle b_n,e_k \right\rangle = 0
            \end{align*}
            Damit hat man ein Element gefunden, welches senkrecht auf den anderen Basisvektoren steht. Normieren von $e_n'$ liefert dann die ON-Basis von $V$.
    \end{itemize}
\end{mysatz}

%----------------------------------------Beispiel-------------------------------------------------

\textit{Beispiel:}

Sei
\begin{align*}
    \{ b_1,b_2,b_3 \} =
    \left\{ \begin{pmatrix}
        1\\-1\\0\\1
    \end{pmatrix}
    ,
    \begin{pmatrix}
        1\\0\\1\\0
    \end{pmatrix}
    ,
    \begin{pmatrix}
        1\\-1\\0\\0
    \end{pmatrix}
    \right\}
\end{align*}
eine Basis von $U \subset \R^4$.
Dann ist $e_1' = b_1$ und $e_2' = \lambda e_1' + b_2$, daher folgt
\begin{align*}
    0 & = \left\langle e_2', e_1' \right\rangle = \lambda \left\langle e_1', e_1' \right\rangle + \left\langle b_2, e_1' \right\rangle \\
    \lambda & = - \frac{ \left\langle b_2, e_1' \right\rangle}{ \left\langle e_1', e_1' \right\rangle} = - \frac{1}{3}
\end{align*}
Also ist
\begin{align*}
    e_2' & = - \frac{1}{3} \cdot
    \begin{pmatrix}
        1\\-1\\0\\1
    \end{pmatrix}
    +
    \begin{pmatrix}
        1\\0\\1\\0
    \end{pmatrix}
    = \frac{1}{3} \cdot
    \begin{pmatrix}
        2\\1\\3\\-1
    \end{pmatrix}
\end{align*}
Für den dritten Basisvektor muss gelten gilt $e_3' = \lambda_1 e_1 + \lambda_2 e_2 + b_3$.
Daher folgt
\begin{align*}
    0 & = \left\langle e_3', e_1' \right\rangle = \lambda_1 \left\langle e_1', e_1' \right\rangle + \lambda_2 \left\langle e_2', e_1' \right\rangle + \left\langle b_3, e_1' \right\rangle\\
    \lambda_1 & = - \frac{ \left\langle b_3, e_1'\right\rangle }{ \left\langle e_1', e_1' \right\rangle } = - \frac{2}{3}
\end{align*}
Dann muss $\lambda_2 = - \frac{1}{5}$ sein, sodass
\begin{align*}
    e_3' = - \frac{2}{3} \cdot
    \begin{pmatrix}
        1\\-1\\0\\1
    \end{pmatrix}
    - \frac{1}{15} \cdot
    \begin{pmatrix}
        2\\1\\3\\-1
    \end{pmatrix}
    +
    \begin{pmatrix}
        1\\-1\\0\\0
    \end{pmatrix}
    = \frac{1}{5}
    \begin{pmatrix}
        1\\-2\\-1\\-3
    \end{pmatrix}
\end{align*}
Normieren von der $e_i'$ liefert dann die ON-Basis
\begin{align*}
    \left\{ \frac{1}{\sqrt{3}} \cdot
    \begin{pmatrix}
        1\\-1\\0\\1
    \end{pmatrix},
    \frac{1}{\sqrt{15}} \cdot
    \begin{pmatrix}
        2\\1\\3\\-1
    \end{pmatrix},
    \frac{1}{\sqrt{15}} \cdot
    \begin{pmatrix}
        1\\-2\\-1\\3
    \end{pmatrix}
    \right\}
\end{align*}

%%%%%%%%%%%%%%%%%%%%%%%%%%%%%%%%%%%%%%%%%%%%%%%%%%%%%%%%%%%%%%%%%
%  TODO Beispiel nochma durchrechnen, irgendwas stimmt da nicht %
%%%%%%%%%%%%%%%%%%%%%%%%%%%%%%%%%%%%%%%%%%%%%%%%%%%%%%%%%%%%%%%%%

%----------------------------------------Bemerkung-------------------------------------------------

\textit{Bemerkung:}

Sei $\{ e_1, \ldots, e_n \}$ eine ON-Basis von $V$ und $v \in V$, dann sind die Koordinaten $k_i$ aus $v = \sum k_i e_i$ gegeben durch
\begin{align*}
    k_i = \left\langle e_i,v \right\rangle
\end{align*}

%%%%% TODO hier fehlt noch einiges %%%%%

%---------------------------------------Definition 4.15-------------------------------------------
%-----------------------------------isometrische Abbildungen--------------------------------------

\begin{mydef}\textit{Isometrische Abbildungen}\medskip

    Eine lineare Abbildung $f$ eines euklidischen VR $V$ in einen euklidischen Vektorraum $W$ heißt genau dann isometrisch, wenn
    \begin{align*}
        \left\langle v,w \right\rangle_V = \left\langle f(v),f(w) \right\rangle_W \quad \forall\ v,w \in W
    \end{align*}
\end{mydef}

%----------------------------------------Bemerkung-------------------------------------------------

\textit{Bemerkung:}

Seien $V$ und $W$ euklidische Vektorräume.
Die lineare Abbildung $f:V\rightarrow W$ ist genau dann eine isometrische Abbildung, wenn sie invariant bzgl. der Normen der jeweiligen Vektorräume ist.
\begin{align*}
    \| f(v) \|_W = \| v \|_V \quad \forall\ v\in V
\end{align*}

%------------------------------------------Lemma 4.16---------------------------------------------
%-------------------------------------------------------------------------------------------------
\begin{mylemma}
    Isometrische Abbildungen sind injektiv.\medskip

    \textit{Beweis:}

    Sei $v \in \ker f$. Dann ist $fv = 0$ und daher
    \begin{align*}
        \left\langle fv,fv \right\rangle = 0 \Leftrightarrow \left\langle v,v \right\rangle = 0 \Leftrightarrow v = 0
    \end{align*}
    Damit liegt nur der Nullvektor im Kern der Abbildung und das Lemma ist bewiesen.
\end{mylemma}

%------------------------------------------Lemma 4.17---------------------------------------------
%-------------------------------------------------------------------------------------------------

\begin{mylemma}
    Sei $\{ b_1, \ldots, b_n \}$ eine ON-Basis von $V$.

    $f$ ist genau dann isometrisch, wenn $\{ f(b_1), \ldots, f(b_n) \}$ eine ON-Basis ist.\medskip

    \textit{Beweis:}
    \begin{itemize}
        \item[,,$\Rightarrow$''] Wenn $f$ isometrisch ist, dann folgt $\left\langle f(b_i),f(b_j) \right\rangle = \left\langle b_i, b_j \right\rangle = \delta_{ij}$.
        \item[,,$\Leftarrow$''] Zu zeigen ist, dass $f$ isometrisch ist, wenn $\{ f(b_1), \ldots, f(b_n) \}$ eine ON-Basis ist.
            Seien dazu $v_1 = \sum k_i \cdot b_i$ und $v_2 = \sum l_i \cdot b_i$.
            Dann ist
            \begin{align*}
                \left\langle v_1, v_2 \right\rangle = \left\langle \sum\limits_{i = 1}^n k_i f(b_i), \sum\limits_{i = 1}^n l_i f(b_i) \right\rangle = \sum k_i \cdot l_i = \left\langle v_1, v_2 \right\rangle
            \end{align*}
    \end{itemize}
\end{mylemma}

Im folgenden sei stets $V = W$. Dann wird $f$ zu einem isometrischen Endomorphismus (auch Automorphismus genannt).
Diese Abbildungen bekommen einen bestimmten Namen. Sie heißen orthogonale Abbildungen.\medskip

Die Menge aller orthogonalen Abbildungen ist eine Gruppe
\begin{align*}
    \mathcal{O}(n) \leq GL(n,\R) = \{ \text{reguläre Matrizen vom Typ $n\times n$} \}
\end{align*}

%----------------------------------------Bemerkung-------------------------------------------------

\textit{Bemerkung:}\medskip

Eine quadratische Matrix heißt genau dann orthogonal, wenn die Spaltenvektoren ein ON-System bilden.

%----------------------------------------------------------------------------------------------------------
%-----------------------------------------------Satz 4.18--------------------------------------------------
%----------------------------------------------------------------------------------------------------------

\begin{mysatz}\label{orthogonaleMatrix}
    Für Matrizen $A\in \R^{n \times n}$ sind folgende Aussagen äquivalent
    \begin{enumerate}
        \item $A$ ist orthogonal.
        \item Die Abbildung $f_A : \R^n \mapsto \R^n$ mit $f_a(v) = A \cdot v$ ist isometrisch.
        \item \label{oM-3} $A^T \cdot A = A \cdot A^T = E$.
        \item Wenn $A$ regulär, dann ist $A^{-1} = A^T$.
        \item Die Zeilenvektoren bilden ein ON-System.
    \end{enumerate}

    \textit{Beweis:}\medskip

    $1.\Leftrightarrow 2.$
\end{mysatz}

%----------------------------------------Bemerkung-------------------------------------------------

\textit{Bemerkung:}
\begin{enumerate}
    \item Sei $A$ eine orthogonale Matrix. Dann ist $\det A = \pm 1$, denn
        \begin{align*}
            1 = \det E = \det A^T \cdot A = \det A^T \cdot \det A = (\det A)^2.
        \end{align*}
    \item Zwei Matrizen $A, B \in \R^{n \times n}$ heißen orthogonal ähnlich, wenn eine orthogonale Matrx $T$ existiert mit $T^{-1} A T = B$.
\end{enumerate}


%-----------------------------------------Beispiel-------------------------------------------------

\textit{Beispiel:}
\textit{Beschreibung von orthogonalen Abbildungen in $\R^2$}\medskip

Sei $V = \R^2$ mit dem Standardskalarprodukt.
Gesucht ist eine Matrix $A \in \R^{2\times 2}$ die orthogonal ist. Nach \ref{orthogonaleMatrix}.\ref{oM-3} muss für $A$ gelten:
\begin{align*}
    \begin{pmatrix}
        a_{11} & a_{12}\\
        a_{21} & a_{22}
    \end{pmatrix}
    \cdot
    \begin{pmatrix}
        a_{11} & a_{21}\\
        a_{12} & a_{22}
    \end{pmatrix}
    =
    \begin{pmatrix}
        1 & 0\\
        0 & 1
    \end{pmatrix}
\end{align*}
Daher muss $a_{11}^2 + a_{12}^2 = 1 = a_{22}^2 + a_{21}^2$ und $a_{11} a_{21} + a_{12} a_{22} = 0$ gelten.
Gleichzeitig müssen die Spalten der Matrix aber auch ein ON-System bilden.
Daher gilt noch $a_{11}^2 + a_{21}^2 = 1$.
Für diese Gleichungen existiert ein $\varphi \in [0 , 2 \pi]$ mit $a_{11} = \cos(\varphi), a_{21} = \sin(\varphi)$.\medskip

Dann sind
\begin{align*}
    D_{\varphi} =
    \begin{pmatrix}
        \cos(\varphi) & - \sin(\varphi)\\
        \sin(\varphi) & \cos(\varphi)
    \end{pmatrix}
    \intertext{oder}
    S_{\varphi} =
    \begin{pmatrix}
        \cos(\varphi) & \sin(\varphi)\\
        \sin(\varphi) & - \cos(\varphi)
    \end{pmatrix}
\end{align*}
die beiden orthogonalen Matrizen in $\R^2$.
Dabei hat $D_{\varphi}$ die Determinante $1$ und stellt eine Drehung um den Winkel $\varphi$ dar.
Die Eigenwerte von $D_{\varphi}$ sind $\lambda_{1,2} = \cos(\varphi) \pm \sqrt{\cos(\varphi)^2 - 1}$.
Diese sind demnach nur dann reell, wenn $\varphi$ den Wert $0$ oder $\pi$ annimmt.\medskip

Dagegen hat $S_{\varphi}$ die Determinante $-1$ und das charakteristische Polynom:
$\charpol S_{\varphi} = \lambda^2 - \cos(\varphi)^2 - \sin(\varphi)^2 = \lambda^2 - 1$ mit den Eigenwerten $\pm 1$.

$S_{\varphi}$ stellt eine Spiegelung der Richtung des Eigenvektors dar.\\


%----------------------------------------Bemerkung-------------------------------------------------

\textit{Bemerkung:}\medskip

Sei $f$ eine orthogonale Abbildung. Dann nehmen die Eigenwerte von $f$, falls diese existieren, den Wert $\pm 1$ an, denn
\begin{align*}
    \| v \| = \| f(v) \| = \| \lambda \cdot v\| = | \lambda | \cdot \| v \|
\end{align*}

%----------------------------------------------------------------------------------------------------------
%-----------------------------------------------Satz 4.19--------------------------------------------------
%----------------------------------------------------------------------------------------------------------

\begin{mysatz}
    Sei $f$ eine orthogonale Abbildung des $n$-dim. Vektorraums $V$ in sich.
    Dann existiert eine ON-Basis von $V$, bzgl der $f$ die Darstellungsmatrix
    \begin{align*}
        \begin{pmatrix}
            1 & 0 & & & \cdots &\cdots & & & 0 \\
            0 & \ddots & & & & & & & \\
            & & 1 & & & & & & \\
            & & & -1 & & & & & \vdots \\
            \vdots & & & & \ddots & & &  &\vdots \\
            \vdots & & & & & -1 & & & \\
            & & & & & & A_1 & & \\
            & & & & & & & \ddots & 0 \\
            0 & & & \cdots & \cdots & & & 0 & A_1
        \end{pmatrix}
    \end{align*}
    besitzt.\medskip

    \textit{Beweis:}

    Induktion über $n$
\end{mysatz}


\section{Affine und Euklidische (Punkt-)Räume} % (fold)
\label{sec:Affine und Euklidische (Punkt-)Räume}

\begin{mydef}\textit{Affiner Raum}\medskip

    Ein affiner Raum $\mathcal{A}_n$ ist eine Menge, deren Elemente $P,Q,R$ genannt werden, auf dem ein $n$-dimen"-sionaler Vektorraum $V$ operiert,
    d.h. es existiert eine Abbildung $+_{\mathcal{A}}: \mathcal{A} \times V \mapsto \mathcal{A}$ mit $P +_{\mathcal{A}} v = Q$ mit folgenden Eigenschaften:
    \begin{enumerate}[label=(\roman*)]
        \item \label{EigaffRaum1} $\forall\ P \in \mathcal{A}:\ P +_{\mathcal{A}} 0 = P$
        \item \label{EigaffRaum2} $\forall\ P \in \mathcal{A}\ \forall \ v,w \in V:\ P +_{\mathcal{A}} (v + w) = (P +_{\mathcal{A}} v) +_{\mathcal{A}} w$
        \item \label{EigaffRaum3} $\forall\ P,Q \in \mathcal{A}\ \exists! \ v \in V:\ P +_{\mathcal{A}} v  = Q$ \hfill (Bezeichnung: $v = \ora{PQ}$)
    \end{enumerate}
    Die Dimension des affinen Raumes ist die Dimension des zugehörigen Vektorraums.
\end{mydef}

\textit{Beispiel:}
\begin{itemize}
    \item elementargeometrischer Punkt und $V$ der Vektorraum der Verschiebungen
    \item $L(A \mid b)$ falls $Ax = b$ lösbar
    \item jeder Vektorraum ist affiner Raum über sich selbst
\end{itemize}

\textit{Bemerkung:}
\begin{itemize}
    \item $\mathcal{A}_n = \left\{ P +_A V \right\} = \left\{ P +_A v \mid v \in V \right\}$

        Jeder Punkt aus $\mathcal{A}$ ist darstellbar durch die Summe aus $P$ und einem Vektor $v$.
    \item $P + v = P + w \Rightarrow v = w$

        $\left( (P+v)-w = (P+w)-w \Rightarrow P + (v-w) = P \Rightarrow v-w = 0 \right)$
\end{itemize}

\begin{mysatz} \textit{Eigenschaften}\medskip

    Für alle $P,Q,R,S \in \mathcal{A}_n$ und für alle $v,w \in V$ gilt:
    \begin{enumerate}
        \item $\ora{PP} = 0$
        \item $P + v = Q + v \Rightarrow P = Q$
        \item \label{SatzEigaffRaum3} $\ora{PQ} + \ora{QR} = \ora{PR}$
        \item $P + v = Q + w \Rightarrow \ora{PQ} = v - w$
        \item $\ora{PQ} + \ora{QP} = 0 \qquad \left( \ora{PQ} = - \ora{QP}  \right)$
        \item $\ora{PQ} = \ora{RS} \Rightarrow \ora{PR} = \ora{QS}$
    \end{enumerate}
    \textit{Beweis:}
    \begin{enumerate}
        \item $P + \ora{PP} \stackrel{\ref{EigaffRaum3}}= P \quad \wedge \quad P + 0 \stackrel{\ref{EigaffRaum1}}= P \Rightarrow 0 = \ora{PP}$
        \item $P + v = Q + v \Rightarrow (P + v) - v = (Q + v) - v \stackrel{\ref{EigaffRaum2}}\Rightarrow P + 0 = Q + 0 \Rightarrow P + Q$
        \item $\uuline{P} + \left( \ora{PQ} + \ora{QR} \right) = \left( P + \ora{PQ} \right) + \ora{QR} = Q + \ora{QR} = \uuline{R} \quad \Rightarrow \quad \ora{PR} = \ora{PQ} + \ora{QR}$
        \item $P + v = Q + w \Rightarrow \left( P + v \right) - w = \left( Q + w \right) - w \Rightarrow P + (v - w) = Q \Rightarrow \ora{PQ} = v - w$
        \item Klar, siehe \ref{SatzEigaffRaum3}.
        \item \ \\
            \begin{minipage}{0.6\textwidth}
                \begin{align*}
                    0 & = \ora{PQ} + \ora{QS} + \ora{SR} + \ora{RP} = \ora{PQ} + \ora{QS} - \ora{RS} - \ora{PR}\\
                    \ora{PR} - \ora{QS} & = \underbrace{\ora{PQ} - \ora{RS}}_{0} \Rightarrow \ora{PR} = \ora{QS}\\
                \end{align*}
            \end{minipage}
            \begin{minipage}{0.4\textwidth}
                \begin{center}
                    \begin{tikzpicture}
                        \draw (0,0) coordinate (P) node[anchor=north east] {$P$};
                        \draw (1,2) coordinate (Q) node[anchor=south east] {$Q$};
                        \draw (2,0) coordinate (R) node[anchor=north west] {$R$};
                        \draw (3,2) coordinate (S) node[anchor=south west] {$S$};
                        \draw[>=latex,->] (P) -- (Q);
                        \draw[>=latex,->] (R) -- (S);
                        \fill (P) circle (1pt);
                        \fill (Q) circle (1pt);
                        \fill (R) circle (1pt);
                        \fill (S) circle (1pt);
                        \draw[>=latex,->,dotted] (P) -- (R);
                        \draw[>=latex,->,dotted] (Q) -- (S);
                    \end{tikzpicture}
                \end{center}
            \end{minipage}\ \\
    \end{enumerate}
\end{mysatz}

\begin{mydef}\textit{Koordinatensystem}\medskip

    Für jeden Punkt $X$ aus $\mathcal{A}_n$ gilt, dass $X = O + \ora{OX}$ für festes $O \in \mathcal{A}_n$.
    Es ist $\ora{OX} = \sum\limits_{i=1}^{n} x_i b_i$, wobei $V = \left\langle b1, \ldots, b_n \right\rangle$.

    Dann heißt die Menge $K = \{ O; \underbrace{b_1, \ldots, b_n}_{B} \}$ Koordinatensystem des $\mathcal{A}_n$.
    \begin{align*}
        X_{/_K} =
        \begin{pmatrix}
            x_1\\ x_2\\ \vdots\\ x_n
        \end{pmatrix}_{/_K}
        \qquad
        (P + v)_{/_K} = P_{/_K} + v_{/_K}
        \qquad
        \left( \ora{PQ} \right)_{/_B} = Q_{/_K} - P_{/_K}
    \end{align*}
\end{mydef}

\begin{mysatz}\textit{Koordinatentransformation}\medskip

    Seien $K = \left\{ O; b_1, \ldots, b_n \right\}$ und $K' = \left\{ O'; b'_1, \ldots, b'_n \right\}$ Koordinatensysteme des $\mathcal{A}_n$, dann gilt
    \begin{align*}
        P_{/_K} = O'_{/_K} + M \cdot P_{/_K'}
    \end{align*}
    wobei $M$ die Matrix des Basisübergangs von $B$ zu $B'$ ist, d.h.
    \begin{align*}
        b'_i = \sum\limits_{j=1}^{n} m_{ji} b_{j} \qquad \forall\ i = 1, \ldots, n
    \end{align*}
    \textit{Beweis:}
    \begin{align*}
        P & = O + \sum\limits_{i=1}^{n} x_i b_i = O' + \sum\limits_{i=1}^{n} x'_i b'_i\\
        \ora{OO'} & = \sum\limits_{i=1}^{n} c_i b_i \quad \Rightarrow \quad O'_{/_K} =
        \begin{pmatrix}
            c_1\\ \vdots\\ c_n
        \end{pmatrix}\\
        P & = O + \sum\limits_{i = 1}^{n} x_i b_i = O + \ora{OO'} + \sum\limits_{i=1}^{n} x'_i \left( \sum\limits_{j=1}^{n} m_{ji} b_{j} \right)\\
        & = O + \ora{OO'} + \sum\limits_{j=1}^{n} \left( \sum\limits_{i=1}^{n} m_{ji} x'_i \right) b_j =
        \begin{pmatrix}
            c_1\\ \vdots\\ c_n
        \end{pmatrix}
        + M \cdot P_{/_K'} = P_{/_K}
    \end{align*}
\end{mysatz}

\textit{Beispiel:} $\quad \mathcal{A}_2 \qquad V = \R^2$\medskip

\begin{minipage}{0.6\textwidth}
    \begin{align*}
        K & = \left\{ O; b_1, b_2 \right\} \quad K' = \left\{ O'; b'_1, b'_2 \right\}\\
        O'_{/_K} & =
        \begin{pmatrix}
            2\\1
        \end{pmatrix}
        \qquad b'_1 = b_1 + b_2 \qquad b'_2 = b_2 - b_1\\
        P_{/_K} & =
        \begin{pmatrix}
            2\\1
        \end{pmatrix}
        +
        \begin{pmatrix}
            1 & -1\\
            1 & 1
        \end{pmatrix}
        P_{/_K'}\\
        P_{/_K'} & = M^{-1} \left( P_{/_K} -
        \begin{pmatrix}
            2\\1
        \end{pmatrix}
        \right)
        = \frac{1}{2}
        \begin{pmatrix}
            1 & 1\\
            -1 & 1
        \end{pmatrix}
        \begin{pmatrix}
            -3\\-2
        \end{pmatrix}
        =
        \uuline{
        \begin{pmatrix}
            -\frac{5}{2}\\
            \frac{1}{2}
        \end{pmatrix}}
    \end{align*}
\end{minipage}
\begin{minipage}{0.4\textwidth}
    \begin{center}
        \begin{tikzpicture}
            \draw (0,0) coordinate (O) node[anchor=north east] {$O$};
            \draw (-1,-1) coordinate (P) node[anchor=north east] {$P$};
            \draw (2,1) coordinate (O') node[anchor=north] {$O'$};
            \fill (O) circle (2pt);
            \fill (P) circle (1pt);
            \fill (O') circle (2pt);
            \draw[->] (-1.5,0) -- (3.5,0);
            \draw[->] (0,-1.5) -- (0,2.5);
            \draw[>=latex,->,thick] (O) -- (1,0) node[midway,anchor=north] {$b_1$};
            \draw[>=latex,->,thick] (O) -- (0,1) node[midway,anchor=east] {$b_2$};
            \draw[>=latex,->,thick] (O') -- (3,2) node[midway,anchor=north,rotate=45] {\small{$b_1 + b_2$}};
            \draw[>=latex,->,thick] (O') -- (1,2) node[midway,anchor=north,rotate=-45] {\small{$b_1 - b_2$}};
            \draw[dotted] (-1,0) -- (P);
            \draw[dotted] (0,-1) -- (P);
        \end{tikzpicture}
    \end{center}
\end{minipage}

\begin{mydef}\textit{Affiner Teilraum}\medskip

    Sei $\mathcal{A}_n$ ein affiner Raum mit zugehörigem Vektorraum $V$.

    Eine Teilmenge $T \subseteq \mathcal{A}_n$ heißt affiner Teilraum genau dann, wenn ein Unterraum $U \subseteq V$ existiert, sodass $T$ zusammen mit $U$ ein affiner Raum ist.
    \begin{itemize}
        \item $T = \left\{ P + U \right\} = \left\{ P + u \mid u \in U \right\}$
            \begin{itemize}
                \item $\dim T = 1 \qquad\quad T:$ Gerade
                \item $\dim T = 2 \qquad\quad T:$ Ebene
                \item $\dim T = n-1 \quad\, T:$ Hyperebene
            \end{itemize}
        \item Sei $G$ eine Gerade, dann gilt $G = \left\{ X:\ X = P + t u \mid t \in \R \ \wedge \ U = \left\langle u \right\rangle \right\}$
        \item Sei $E$ eine Ebene, dann gilt $E = \left\{ X:\ X = P + t_1 u_1 + t_2 u_2 \mid t_{1,2} \in \R \ \wedge \ U = \left\langle u_1, u_2 \right\rangle \right\}$
        \item Seien $P,Q$ Punkte des $\mathcal{A}_n$ mit $P \neq Q$, dann existiert \textbf{genau} eine Gerade $G$ mit $P,Q \in G$.
            \begin{align*}
                G & = \left\{ X:\ X = P + t \cdot \ora{PQ} \mid t \in \R \right\}\\
                & = \left\{ X:\ X = Q + t \cdot \ora{QP} \mid t \in \R \right\}\\
                G' :\ X & = H + t a \qquad \wedge \qquad P,Q \in G'\\
                P & = H + t_1 a\\
                \ora{HP} & = t_1 a\\
                X & = P + \ora{PH} + t a = P + \left( t - t_1 \right) a\\
                G' :\ X & = P + k a \qquad Q = P + k_1 a\\
                \ora{PQ} & = k_1 a\\
                X & = P + \frac{k}{k_1} \ora{PQ}
            \end{align*}
    \end{itemize}
\end{mydef}

\begin{mysatz}\medskip

    Seien $T_1 = \left\{ P+U \right\}$ und $T_2 = \left\{ Q+U \right\}$ affine Teilräume eines affinen Raumes $\mathcal{A}_n$ mit Vektorraum $V$. Dann gilt:
    \begin{align*}
        T_1 = T_2 \Leftrightarrow U = W \ \wedge \ \ora{PQ} \in U
    \end{align*}
    \textit{Beweis:}
    \begin{itemize}
        \item[,,$\Rightarrow$'']
            \begin{align*}
                \left\{ P + U \right\} = \left\{ Q + W \right\} & \Rightarrow \exists u \in U:\ P + u = Q\\
                & \Rightarrow u = \ora{PQ} \in U & \left( \text{analog } \ora{QP} \in W \right)\\
                w \in W & \Rightarrow \exists u':\ P + u' = Q + w\\
                & \Rightarrow u' = \ora{PQ} + w \Rightarrow w = u' - \ora{PQ} \in U\\
                & \left.
                \begin{matrix}
                    \Rightarrow & W \subseteq U\\
                    \text{analog} & U \subseteq W
                \end{matrix}
                \right\} \Rightarrow U = W
            \end{align*}
        \item[,,$\Leftarrow$'']
            \begin{align*}
                \exists x \in U:\ P + x & = Q\\
                x + U & = U\\
                P + U & = P + \left( x + U \right) = \left( P + x \right) + U = Q + U = Q + W
            \end{align*}
    \end{itemize}
\end{mysatz}

\textit{Bemerkung:}
\begin{itemize}
    \item Ist $T$ ein affiner Teilraum, dann ist $T = \left\{ P + \sum\limits_{i = 1}^{n} k_i u_i \mid k_i \in \K \right\}$ (wobei $\left\{ u_1, \ldots, u_n \right\}$ eine Basis von $U$ ist) 
        die Parameterdarstellung des affinen Teilraumes.
    \item Jeder affine Teilraum lässt sich als lineares Gleichungssystem darstellen. Diese Darstellung heißt parameterfreie Darstellung.
\end{itemize}
\textit{Beispiel:}
\begin{itemize}
    \item[$\mathcal{A}_2:$] $T_1 =
        \left\{ \begin{pmatrix}
            p_1\\ p_2
        \end{pmatrix}
        +
        t \cdot
        \begin{pmatrix}
            u_1\\ u_2
        \end{pmatrix}
        \mid t \in \K
        \right\}$
        \begin{align*}
            x & = p_1 + t u_1 \qquad \text{o.B.d.A. } u_1 \neq 0 \ \Rightarrow \ t = \frac{x - p_1}{u_1}\\
            y & = p_2 + t u_2\\
            \text{Einsetzen ergibt:} \qquad u_1 y & = u_1 p_2 + \left( x - p_1 \right) u_2 \ \Rightarrow \ u_2 x - u_1 y = p_1 u_2 - u_1 p_2
        \end{align*}
    \item[$\mathcal{A}_3:$] $T_1 = \left\{ P + t u \mid t \in \K \right\}$
        \begin{align*}
            T_1 & = \left\{
            \begin{pmatrix}
                1\\ 2\\ 3
            \end{pmatrix}
            + t
            \begin{pmatrix}
                1\\ -1\\ 1
            \end{pmatrix}
            \mid t \in \R
            \right\}\\
            & \left.
            \begin{matrix}
                x = 1 + t & \qquad & \rightarrow t = x - 1\\
                y = 2 - t\\
                z = 3 + t
            \end{matrix}
            \right\}\quad
            \begin{matrix}
                y = 2 - x + 1\\
                z = 3 + x - 1
            \end{matrix}\\
            \Rightarrow & \quad
            \begin{matrix}
                x+y & = & 3\\
                -x+z & = & 2
            \end{matrix}
        \end{align*}
    \item[] $T_2 = \left\{ P + t_1 u_1 + t_2 u_2 \mid t_{1,2} \in \R\right\}$
        \begin{align*}
            T_2 & =
            \left\{
            \begin{pmatrix}
                1\\ 0\\ 1
            \end{pmatrix}
            + t_1
            \begin{pmatrix}
                2\\ 1\\3
            \end{pmatrix}
            + t_2
            \begin{pmatrix}
                1\\ 1\\ 1
            \end{pmatrix}
            \mid t_{1,2} \in \F_5
            \right\}\\
            & \left.
            \begin{matrix}
                x & = & 1 + 2 t_1 + t_2 & \qquad & \rightarrow & t_2 = x + 4 + 3 t_1\\
                y & = & t_1 + t_2 & & & y = 4 t_1 + x + 4\\
                z & = & 1 + 3 t_1 + t_2
            \end{matrix}
            \right\} \quad t_1 = 4 y + x + 4\\
            z & = 1 + 3\left( 4 y + x + 4 \right) + x + 4 + 3 \left( 4 y + x + 4 \right) \qquad \Rightarrow \qquad 3 x + y + z = 4
        \end{align*}
\end{itemize}

\begin{mysatz}\label{Schnitte}\textit{Schnitte}\medskip

    Der Durchschnitt zweier affiner Teilräume $T_1$ und $T_2$ von $\mathcal{A}_n$ ist ein affiner Teilraum oder leer.\medskip

    \textit{Beweis:} $\qquad T_1 = P + U \qquad T_2 = Q + W$\medskip

    Sei $T_1 \cap T_2 \neq \emptyset$, also $T_1 \cap T_2 \ni T$
    \begin{align*}
        \Rightarrow & T_1 = T + U \ \wedge \ T_2 = T + W\\
        & T_1 \cap T_2 = S\\
        & R = T + \left( U \cap W \right) \text{ist affiner Teilraum}\\
        & R \subseteq T_1 \ \wedge \ R \subseteq T_2\\
        \Rightarrow & R \subseteq S
    \end{align*}
    Sei $X \in S \Rightarrow X = T + u = T + w \Rightarrow u = w =: v \in U \cap W \Rightarrow X \in T + U \cap W \Rightarrow S = R$ und damit affiner Teilraum.
\end{mysatz}

\textit{Beispiel:}\medskip

$T_1 = 
\left\{
\begin{pmatrix}
    1\\ 1\\ 1
\end{pmatrix}
+ t_1
\begin{pmatrix}
    0\\ 1\\ 1
\end{pmatrix}
+ t_2
\begin{pmatrix}
    4\\ 1\\ 0
\end{pmatrix}
\mid t_{1,2} \in \F_5
\right\}$ und die Ebene $3 x + y + z = 4$.
\begin{align*}
    3\left( 1 + 4 t_2 \right) + \left( 1 + t_1 + t_2 \right) + \left( 1 + t_1 \right) = 4 \qquad & t_1 = s\\
    2 t_1 + 3 t_2 = 4 \qquad & t_2 = 3 + s\\
    T_1 \cap T_2 =
    \left\{
    \begin{pmatrix}
        1\\ 1\\ 1
    \end{pmatrix}
    + s
    \begin{pmatrix}
        0\\ 1\\ 1
    \end{pmatrix}
    + (3 + s)
    \begin{pmatrix}
        4\\ 1 \\0
    \end{pmatrix}
    \mid
    s \in \F_5
    \right\}
    = & 
    \left\{
    \begin{pmatrix}
        3\\ 4\\1
    \end{pmatrix}
    + s
    \begin{pmatrix}
        4\\ 2\\ 1
    \end{pmatrix}
    \mid s \in \F_5
    \right\}
\end{align*}

\begin{mylemma}\textit{kleinster Teilraum}\medskip

    Seien $P_1, P_2, \ldots, P_n, P_{n+1}$ Punkte eines affinen Raumes $\mathcal{A}_n$, dann ist der kleinste affine Teilraum, der diese Punkte enthält, der Teilraum
    \begin{align*}
        T & = \bigcap\limits_{\substack{\scriptscriptstyle P1, \ldots, P_{n+1} \in T_i\\ \scriptscriptstyle T_i \text{ist Teilraum}}} T_i
    \end{align*}
    $T$ heißt der von $P_1, P_2, \ldots, P_n, P_{n+1}$ aufgespannte Teilraum. $T = P_1 + W$ mit $\left\langle \ora{P_1 P_2}, \ldots, \ora{P_1 P_{n+1}} \right\rangle = W$

    \textit{Beweis:}\medskip

    Nach Satz \ref{Schnitte} ist $T$ ein affiner Teilraum und kleinstmöglich.
    Da für den zugehörigen Unterraum $U$ gilt, dass $\ora{P_i P_j} \ \left( i,j \in \left\{ 1, \ldots, n+1 \right\} \right)$ in $U$ enthalten sind, ist $W$ in $U$.
    Da $T$ minimal ist, folgt $T = P_1 + W$.
\end{mylemma}

\begin{mydef}\textit{Punkte in allgemeiner Lage}\medskip

    Die Punkte $P_1, \ldots, P_n, P_{n+1}$ heißen Punkte in allgemeiner Lage genau dann, wenn $\left\{ \ora{P_1 P_2}, \ldots, \ora{P_n P_{n+1}} \right\}$ linear unabhängig ist.
\end{mydef}

\begin{mydef}\textit{Summe von Teilräumen}\medskip

    Seien $T_1 = P + U$ und $T_2 = Q + W$ affine Teilräume, dann heißt
    \begin{align*}
        T_1 + T_2 = \left\{ P + x \mid x \in U + W + \left\langle \ora{PQ} \right\rangle \right\}
    \end{align*}
    Summe der affinen Teilräume.
\end{mydef}

\begin{mysatz}\textit{Dimensionssatz}\medskip

    Seien $T_1 = P + U$ und $T_2 = Q + W$, dann gilt
    \begin{align*}
        \dim T_1 + T_2 =
        \begin{cases}
            \dim T_1 + \dim T_2 - \dim T_1 \cap T_2 & \text{falls} T_1 \cap T_2 \neq \emptyset\\
            \dim T_1 + \dim T_2 - \dim \left( U \cap W \right) + 1 & \text{falls} T_1 \cap T_2 = \emptyset
        \end{cases}
    \end{align*}
    \textit{Beweis:}
    \begin{itemize}
        \item[1. Fall:] $T_1 \cap T_2 \neq \emptyset$.
            Für die Teilräume gilt dann $T_1 = R + U$ und $T_2 = R + W$ mit einem $R \in T_1 \cap T_2$.
            Für die Summe der Teilräume gilt dann $T_1 + T_2 = \left\{ R + \left( U + W \right) \right\}$ was die Behauptung liefert.
        \item[2. Fall:] $T_1 \cap T_2 = \emptyset$.
            Es ist $\ora{PQ} \in U + W$, denn wäre $\ora{PQ} = u + w$, dann wäre $P + U = Q + W \ \lightning!$ da der Schnitt leer ist.
            Daraus folgt: $\dim T_1 + T_2 = \dim \left( U + W \right) + 1$, was die Behauptung liefert.
    \end{itemize}
\end{mysatz}

\begin{mydef}\textit{Parallelität}\medskip

    Zwei affine Teilräume $T_1 := \left\{ X:\ X = P + U \right\}$ und $T_2 := \left\{ X:\ X = Q + W \right\}$ heißen parallel $\left( T_1 \parallel T_2 \right)$ genau dann, wenn $U \subseteq W$ oder $W \subseteq U$.
\end{mydef}
\textit{Bemerkung:}
\begin{enumerate}
    \item $\parallel$ ist keine Äquivalenzrelation.
    \item Bei Einschränkung auf gleichdimensionale Teilräume ist es jedoch eine Äquivalenzrelation.
    \item $\left( T_1 \cap T_2 \neq \emptyset \ \wedge \ T_1 \parallel T_2 \right) \Rightarrow T_1 \subseteq T_2 \ \vee \ T_2 \subseteq T_1$
    \item $T_1 \parallel T_2 \ \wedge \ \dim T_1 = n - 1 \Rightarrow T_1 \cap T_2 \neq \emptyset$ \hfill $\left( \dim T_2 \geq 1 \right)$
\end{enumerate}

Im folgenden ist $\mathcal{A}_n$ stets über $V = \R^n$.

\begin{mydef}\textit{Strecke, Strahl, Mittelpunkt}
    \begin{enumerate}
        \item $\overline{AB} = \left\{ X:\ X = A + t \cdot \ora{AB} \mid 0 \leq t \leq 1, t \in \R \right\}$
        \item $AB^+ = \left\{ X:\ X = A + t \cdot \ora{AB} \mid t \geq 0, t \in \R \right\}$

            Strahl mit Anfangspunkt $A$, der $B$ enthält.
        \item $AB^- = \left\{ X:\ X = A + t \cdot \ora{AB} \mid t \leq 0, t \in \R \right\}$

            Strahl mit Anfangspunkt $A$, der $B$ nicht enthält.
        \item $M = A + \frac{1}{2} \ora{AB}$ heißt Mittelpunkt der Strecke.
    \end{enumerate}
\end{mydef}

\begin{mydef}\textit{Teilverhältnis}\medskip

    Gegeben seien 3 kollineare Punkte $A \neq B \neq X$.

    Dann heißt die eindeutig bestimmte reelle Zahl $\mu$ mit $\ora{AX} = \mu \cdot \ora{BX}$ das Teilverhältnis $\TV(A,B;X)$.
    \begin{enumerate}
        \item $X$ zwischen $A$ und $B$.

            \begin{minipage}{0.6\textwidth}
                \begin{tabular}{rl}
                    Allgemein: & $\TV(A,B;X) < 0$\\
                    Sonderfall: & $\ora{AX} = - \ora{BX}$ wenn $X = M$.
                \end{tabular}
            \end{minipage}
            \begin{minipage}{0.4\textwidth}
                \begin{center}
                    \begin{tikzpicture}
                        \fill (0,0) coordinate (A) circle (1pt);
                        \fill (1,0.5) coordinate (X) circle (1pt);
                        \fill (2,1) coordinate (B) circle (1pt);
                        \draw (A) node[anchor=north west] {$A$};
                        \draw (X) node[anchor=south east] {$X$};
                        \draw (B) node[anchor=north west] {$B$};
                        \draw (-0.5,-0.25) -- (2.5,1.25);
                    \end{tikzpicture}
                \end{center}
            \end{minipage}
        \item $A$ zwischen $X$ und $B$.

            \begin{minipage}{0.6\textwidth}
                $0 \leq \TV(A,B;X) < 1$\\
            \end{minipage}
            \begin{minipage}{0.4\textwidth}
                \begin{center}
                    \begin{tikzpicture}
                        \fill (0,0) coordinate (X) circle (1pt);
                        \fill (0.8,0.4) coordinate (A) circle (1pt);
                        \fill (2,1) coordinate (B) circle (1pt);
                        \draw (A) node[anchor=north west] {$A$};
                        \draw (X) node[anchor=south east] {$X$};
                        \draw (B) node[anchor=north west] {$B$};
                        \draw (-0.5,-0.25) -- (2.5,1.25);
                    \end{tikzpicture}
                \end{center}
            \end{minipage}
        \item $B$ zwischen $A$ und $X$.

            \begin{minipage}{0.6\textwidth}
                $\TV(A,B;X) > 1$\\
            \end{minipage}
            \begin{minipage}{0.4\textwidth}
                \begin{center}
                    \begin{tikzpicture}
                        \fill (0,0) coordinate (A) circle (1pt);
                        \fill (1.2,0.6) coordinate (B) circle (1pt);
                        \fill (2,1) coordinate (X) circle (1pt);
                        \draw (A) node[anchor=north west] {$A$};
                        \draw (X) node[anchor=south east] {$X$};
                        \draw (B) node[anchor=north west] {$B$};
                        \draw (-0.5,-0.25) -- (2.5,1.25);
                    \end{tikzpicture}
                \end{center}
            \end{minipage}
    \end{enumerate}
    \begin{align*}
        X & = A + t_X \ora{AB}\\
        \ora{BX} & = \ora{AX} - \ora{AB} = t_X \ora{AB} - \ora{AB}\\
        \ora{AB} & = \frac{1}{t_X - 1} \cdot \ora{BX} \qquad \ora{AX} = \frac{t_X}{t_X - 1} \ora{BX} \qquad \mu = \frac{t_X}{t_X - 1} \qquad t_X = \frac{\mu}{\mu - 1}
    \end{align*}
\end{mydef}

\begin{mysatz}\textit{Seitenhalbierende}\medskip

    In einem Dreieck $ABC$ schneiden sich die Seitenhalbierenden in einem Punkt $S$ und es gilt
    \begin{align*}
        \TV(M_c,C;S) = \TV(M_a,A;S) = \TV(M_b,B;S) = - \frac{1}{2}
    \end{align*}
    \begin{flushright}
        ($M_a, M_b, M_c$ seien die jeweiligen Seitenschnitte)
    \end{flushright}
    \begin{minipage}{0.6\textwidth}
        \begin{align*}
            \textcolor{blue}{S_c}:\ X & = C + t_1 \ora{CM_c} = C + t_1 \left( \ora{CA} + \frac{1}{2} \ora{AB} \right)\\
            \textcolor{blue}{S_b}:\ X & = B + t_2 \ora{BM_b} = B + t_2 \left( \ora{BA} + \frac{1}{2} \ora{AC} \right)\\
            \textcolor{blue}{S_a}:\ X & = A + t_3 \ora{AM_b} = A + t_3 \left( \ora{AB} + \frac{1}{2} \ora{BC} \right)
        \end{align*}
    \end{minipage}
    \begin{minipage}{0.4\textwidth}
        \begin{center}
        \begin{tikzpicture}
            \draw (0,0) coordinate (A) node[anchor=north east] {$A$};
            \draw (2,2) coordinate (B) node[anchor=south] {$B$};
            \draw (5,0) coordinate (C) node[anchor=west] {$C$};
            \draw (1,1) coordinate (Mc) node[anchor=south, above=5pt] {$M_c$};
            \draw (2.5,0) coordinate (Mb) node[anchor=north west] {$M_b$};
            \draw (3.5,1) coordinate (Ma) node[anchor=south, above=5pt] {$M_a$};
            \draw[color=blue] (7/3,2/3) coordinate (S) node[anchor=south west] {$S$};
            \fill (A) circle (1pt);
            \fill (B) circle (1pt);
            \fill (C) circle (1pt);
            \fill (Ma) circle (1pt);
            \fill (Mb) circle (1pt);
            \fill (Mc) circle (1pt);
            \fill[color=blue] (S) circle (1pt);
            \draw (A) -- (B) -- (C) -- (A);
            \draw[color=blue] (A) -- (Ma) -- (4.5,9/7) node[anchor=south] {$S_a$};
            \draw[color=blue] (C) -- (Mc) -- (0,5/4) node[anchor=south] {$S_c$};
            \draw[color=blue] (B) -- (Mb) -- (2.625,-0.5) node[anchor=east] {$S_b$};;
        \end{tikzpicture}
        \end{center}
    \end{minipage}
    \begin{align*}
        S_c & \cap S_b\\
        C + t_1 \left( \ora{CA} + \frac{1}{2} \ora{AB} \right) & = \ \ \, \underbrace{B} + t_2 \left( \ora{BA} + \frac{1}{2} \ora{AC} \right)\\
        & = \overbrace{C + \ora{CB}} + t_2 \left( \ora{BA} + \frac{1}{2} \ora{AC} \right)\\
        t_1 \ora{CA} + t_1 \cdot \frac{1}{2} \ora{AB} & = t_2 \ora{BA} + t_2 \cdot \frac{1}{2} \ora{AC} + \ora{CB}\\
        0 & = \left( \frac{t_1}{2} + t_2 \right) \ora{AB} + \left( t_1 + \frac{t_2}{2} \right) \ora{CA} \underbrace{- \ora{CB}}_{-\left( \ora{CA} + \ora{AB} \right)}\\
        0 & = \left( \frac{t_1}{2} + t_2 - 1 \right) \ora{AB} + \left( t_1 + \frac{t_2}{2} - 1 \right) \ora{CA}\\
        \frac{t_1}{2} + t_2 = 1 \qquad t_1 + \frac{t_2}{2} & = 1 \qquad \Rightarrow \quad t_1 = t_2 = \frac{2}{3}\\
        S_c \cap S_b & = S_1 = C + \frac{2}{3} \left( \ora{CA} + \frac{1}{2} \ora{AB} \right)\\
        S_c \cap S_a & = S_2 = A + \frac{2}{3} \left( \ora{AB} + \frac{1}{2} \ora{BC} \right)\\
        \Rightarrow S_1 & = S_2\\
        S_2 & = A + \frac{2}{3} \ora{AB} + \frac{1}{3} \ora{BC} = C + \ora{CA} + \frac{1}{3} \left( \ora{AB} + \ora{BC} \right) + \frac{1}{3} \ora{AB}\\
        & = C + \ora{CA} + \frac{1}{3} \ora{AC} + \frac{1}{3} \ora{AB} = C + \frac{2}{3} \ora{CA} + \frac{1}{3} AB = S_1 = S\\
        \Rightarrow \ora{CS} & = \frac{2}{3} \ora{CA} + \frac{1}{3} \ora{AB}\\
        \ora{AS} & = \frac{2}{3} \ora{AB} + \frac{1}{3} \ora{BC}\\
        \ora{SM_c} & = \ora{SA} + \frac{1}{2} \ora{AB} = - \frac{2}{3} \ora{AB} - \frac{1}{3} \ora{BC} + \frac{1}{2} \ora{AB} = - \frac{1}{6} \ora{AB} - \frac{1}{3} \ora{BC}\\
        & = - \frac{1}{6} \ora{AB} - \frac{1}{3} \left( \ora{BA} + \ora{AC} \right) = \frac{1}{6} \ora{AB} + \frac{1}{3} \ora{CA} = \frac{1}{2} \ora{CS}
    \end{align*}
\end{mysatz}

\begin{mysatz}\textit{Strahlensatz} (\textsc{Thales}\footnote{Thales von Milet (* um 624 v. Chr. in Milet, Kleinasien; $\dagger$ um 546 v. Chr.), griechischer Naturphilosoph, Staatsmann, Mathematiker, Astronom und Ingenieur.})

    Schneiden sich 2 Geraden $G_1 \neq G_2$ in einem Punkt $P$ und seien $R_1, Q_1 \in G_1$ und $R_2, Q_2 \in G_2$, dann gilt
    \begin{align*}
        G(R_1 R_2) \parallel G(Q_1 Q_2) \quad \Leftrightarrow \quad \TV\left( R_1, Q_1; P \right) = \TV\left( R_2, Q_2; P \right)
    \end{align*}
    \begin{minipage}{0.6\textwidth}
        \begin{itemize}
            \item[,,$\Rightarrow$'']
                \begin{align*}
                    \ora{R_1 R_2} & = \lambda \ora{Q_1 Q_2}\\
                    \mu_1 & = \TV\left( R_1, Q_1; P \right)\\
                    \mu_2 & = \TV\left( R_2, Q_2; P \right)\\
                    \ora{R_1 P} & = \mu_1 \ora{Q_1 P}\\
                    \ora{R_2 P} & = \mu_2 \ora{Q_2 P}
                \end{align*}
        \end{itemize}
    \end{minipage}
    \begin{minipage}{0.4\textwidth}
        \begin{center}
        \begin{tikzpicture}
            \draw (0,1) coordinate (Q1) node[anchor=south east] {$Q_1$};
            \draw (1,2) coordinate (R1) node[anchor=south east] {$R_1$};
            \draw (2,3) coordinate (P) node[anchor=west] {$P$};
            \draw (2,0) coordinate (Q2) node[anchor=west] {$Q_2$};
            \draw (2,1.5) coordinate (R2) node[anchor=west] {$R_2$};
            \fill (Q1) circle (1pt);
            \fill (R1) circle (1pt);
            \fill (P) circle (1pt);
            \fill (Q2) circle (1pt);
            \fill (R2) circle (1pt);
            \draw (-0.5,0.5) -- (2.3,3.3);
            \draw (2,3.5) -- (2,-0.5);
            \draw[color=blue,thick] (R1) -- (R2);
            \draw[color=blue,thick] (Q1) -- (Q2);
        \end{tikzpicture}
        \end{center}
    \end{minipage}
    \begin{align*}
        \ora{PQ_2} & = \ora{PQ_1} + \ora{Q_1 Q_2}\\
        \ora{PR_2} & = \ora{PR_1} + \ora{R_1 R_2} = \ora{PR_1} + \lambda \ora{Q_1 Q_2}\\
        \lambda \ora{PQ_2} - \ora{PR_2} = \lambda \ora{PQ_1} - \ora{PR_1}\\
        \lambda \ora{PQ_2} + \mu_2 \ora{Q_2 P} & = \lambda \ora{PQ_1} + \mu_1 \ora{Q_1 P}\\
        \left( la - \mu_2 \right) \ora{PQ_2} + \left( \lambda - \mu_1 \right) \ora{PQ_1} & = 0\\
        \Rightarrow \quad \lambda - \mu_2 = \lambda - \mu_1 = 0 \quad & \Rightarrow \quad \lambda = \mu_1 = \mu_2\\
    \end{align*}
    \begin{itemize}
        \item[,,$\Leftarrow$'']
            \begin{align*}
                \ora{R_1 P} = \mu \ora{Q_1 P} \quad & \wedge \quad \ora{R_2 P} = \mu \ora{Q_2 P}\\
                \ora{R_1 R_2} = \ora{R_1 P} + \ora{PR_2} \quad & = \quad \mu \ora{Q_1 P} - \mu \ora{Q_2 P} = \mu \ora{Q_1 Q_2}\\
                \Rightarrow \quad G(R_1 R_2) \ & \parallel \ G(Q_1 Q_2)
            \end{align*}
    \end{itemize}
\end{mysatz}

%\newpage

\begin{mysatz}\textit{Satz von} \textsc{Menelaos}\footnote{* um 70 in Alexandria; $\dagger$ um 140 vermutlich in Rom}

    \begin{minipage}{0.6\textwidth}
        Gegeben sei ein Dreieck $ABC$ und eine Gerade $G$, die nicht durch einen Eckpunkt geht.
        Seien $A'$ der Schnittpunkt von $G$ mit $G(BC)$, $B'$ der Schnittpunkt von $G$ mit $G(AC)$ und $C'$ der Schnittpunkt von $G$ mit $G(AB)$, dann gilt
    \end{minipage}
    \begin{minipage}{0.4\textwidth}
        \begin{center}
            \begin{tikzpicture}
                \draw (0,0) coordinate(as) node[anchor=east] {$A'$};
                \draw (1,1) coordinate(c) node[anchor=south east] {$C$};
                \draw (3,3) coordinate(b) node[anchor=south] {$B$};
                \draw (4,1) coordinate(a) node[anchor=north west] {$A$};
                \draw (2.5,1) coordinate(bs) node[anchor=north] {$B'$};
                \draw (3.74,1.5) coordinate(cs) node[anchor=south west] {$C'$};
                \fill (as) circle (1pt);
                \fill (c) circle (1pt);
                \fill (b) circle (1pt);
                \fill (a) circle (1pt);
                \fill (bs) circle (1pt);
                \fill (cs) circle (1pt);
                \draw (a) -- (b) -- (c) -- (a);
                \draw[dashed] (as) -- (c);
                \draw[color=blue] (as) -- (bs) -- (cs);
            \end{tikzpicture}
        \end{center}
    \end{minipage}
    \begin{align*}
        \underbrace{\TV(A,B;C')}_{v}
        \cdot
        \underbrace{\TV(B,C;A')}_{w}
        \cdot
        \underbrace{\TV(C,A;B')}_{u}
        = 1
    \end{align*}
    \textit{Beweis:}
    \begin{align*}
        \ora{AC'} & = v \cdot \ora{BC'} = \frac{v}{v-1} \ora{AB} \qquad \ora{BA'} = w \cdot \ora{CA'} = \frac{w}{w-1} \ora{BC} \qquad \ora{CB'} = u \cdot \ora{AB'} = \frac{u}{u-1} \ora{CA}
        \intertext{Nun definieren wir:}
        \ora{AB} & =\ : c \qquad \ora{AC} =\ : b
        \intertext{Daraus folgt:}
        \ora{AC'} & = \frac{v}{v-1} c \qquad \ora{AB'} = \frac{-1}{u-1} b \qquad \ora{AA'} = \ora{AB} + \ora{BA'} = c + \frac{w}{w-1} \ora{BC}\\
        \frac{w}{w-1} b & - \frac{1}{w-1} c = c + \frac{w}{w-1} \left( b-c \right)\\
        \ora{A'B'} & = \ora{A'A} + \ora{AB'} = \frac{-w}{w-1} b + \frac{1}{w-1} c - \frac{1}{u-1} b \qquad \ora{A'C'} = \ora{A'A} + \ora{AC'} = \frac{-w}{w-1} b + \frac{1}{w-1} c + \frac{v}{v-1} c
    \end{align*}
    weil $A',B',C' \in G \Rightarrow \exists !\ \lambda :\ \lambda \ora{A'B'} = \ora{A'C'}$
    \begin{align*}
        \lambda \left( \frac{1}{w-1}c - \frac{w}{w-1}b - \frac{1}{u-1}b \right) & = \frac{1}{w-1}c - \frac{w}{w-1}b + \frac{v}{v-1}c\\
        \left( -\frac{w \lambda}{w-1} - \frac{\lambda}{u-1} + \frac{w}{w-1} \right)b & + \left( \frac{\lambda}{w-1} - \frac{1}{w-1} - \frac{v}{v-1} \right)c = 0
    \end{align*}
    da $b,c$ linear unabhängig sind:
    \begin{align*}
        \frac{\lambda}{w-1} - \frac{1}{w-1} - \frac{v}{v-1} = 0 \Rightarrow \lambda = \left( \frac{v}{v-1} + \frac{1}{w-1} \right)(w-1) = \frac{v(w-1)}{v-1} +1
    \end{align*}
    Da auch der andere Koeffizient 0 ist, muss gelten:
    \begin{align*}
        0 & = - \frac{w \left( \frac{v(w-1)}{v-1} + 1 \right)}{w-1} - \frac{\frac{v(w-1)}{v-1}+1}{u-1} + \frac{w}{w-1}\\
        0 & = - \frac{w(v(w-1))+ v-1}{w-1} - \frac{v(w-1)+v-1}{u-1} + \frac{w(v-1)}{w-1}\\
        0 & = - \frac{w(vw-1)}{w-1} - \frac{vw-1}{u-1} + \frac{vw-w}{w-1}\\
        0 & = -w(vw-1)(u-1) - (vw-1)(w-1) + (vw-w)(u-1)\\
        0 & = -uvw^2+vw^2+uw-w-vw^2+vw+w-1+uvw-vw-uw+w\\
        0 & = uvw(1-w)+(w-1)\\
        1 & = uvw
    \end{align*}
\end{mysatz}

\begin{mysatz}\textit{Satz von} \textsc{Ceva}\footnote{Giovanni Ceva, * 7. Dezember 1647 in Mailand; $\dagger$ 15. Juni 1734 in Mantua}

    \begin{minipage}{0.6\textwidth}
        Schneiden sich die 3 Ecktransversalen eines Dreiecks $ABC$ in einem Punkt $P$ und seien $A',B',C'$ wie oben, dann gilt
    \end{minipage}
    \begin{minipage}{0.4\textwidth}
        \begin{center}
            \begin{tikzpicture}
                %\draw[help lines] (0,0) grid (3,3);
                \draw (0,0) coordinate(c) node[anchor=south east] {$C$};
                \draw (2,2) coordinate(b) node[anchor=south] {$B$};
                \draw (3,0) coordinate(a) node[anchor=north west] {$A$};
                \draw (2.5,1) coordinate(cs) node[anchor=south west] {$C'$};
                \draw (2,0.8) coordinate(p) node[anchor=south west] {$P$};
                \draw (4/3,4/3) coordinate(as) node[anchor=east] {$A'$};
                \draw (2,0) coordinate(bs) node[anchor=north] {$B'$};
                \fill (c) circle (1pt);
                \fill (b) circle (1pt);
                \fill (a) circle (1pt);
                \fill (cs) circle (1pt);
                \fill (p) circle (1pt);
                \fill (as) circle (1pt);
                \fill (bs) circle (1pt);
                \draw (a) -- (b) -- (c) -- (a);
                \draw[color=blue] (c) -- (cs);
                \draw[color=blue] (a) -- (as);
                \draw[color=blue] (b) -- (bs);
            \end{tikzpicture}
        \end{center}
    \end{minipage}
    \begin{align*}
        \TV(A,B;C') \cdot \TV(B,C;A') \cdot \TV(C,A;B') = 1
    \end{align*}
    \textit{Beweis:} Übungsaufgabe
\end{mysatz}
\begin{mysatz}\textit{Satz von} \textsc{Pappos}\footnote{um 300, Alexandria}

    \begin{minipage}{0.6\textwidth}
        $G,G'$ verschiedene Geraden in einer affinen Ebene und $P_1,P_2,P_3 \in G$ und $P'_1,P'_2,P'_3 \in G'$, dann gilt:
    \end{minipage}
    \begin{minipage}{0.4\textwidth}
        \begin{center}
            \begin{tikzpicture}
                %\draw[help lines] (0,0) grid (4,3);
                \draw (0,0) coordinate(p3s) node[anchor=north] {$P'_3$};
                \draw (0,1) coordinate(p1) node[anchor=south] {$P_1$};
                \draw (1.5,0) coordinate(p2s) node[anchor=north] {$P'_2$};
                \draw (3.5,0) coordinate(p1s) node[anchor=north] {$P'_1$};
                \draw (3.5,1.7) coordinate(p3) node[anchor=south] {$P_3$};
                \draw (20/13,17/13) coordinate(p2) node[anchor=south] {$P_2$};
                \fill (p3s) circle (1pt);
                \fill (p1) circle (1pt);
                \fill (p2s) circle (1pt);
                \fill (p1s) circle (1pt);
                \fill (p3) circle (1pt);
                \fill (p2) circle (1pt);
                \draw (-0.3,0) -- (p3s) -- (p2s) -- (p1s) -- (3.8,0);
                \draw (-0.3,0.94) -- (p1) -- (p3) -- (3.8,1.76);
                \draw[dotted] (p1) -- (p3s);
                \draw[dotted] (p3) -- (p1s);
                \draw[dotted] (p1) -- (p2s);
                \draw[dotted] (p2) -- (p1s);
                \draw[color=blue,thick] (p2) -- (p3s);
                \draw[color=blue,thick] (p2s) -- (p3);
            \end{tikzpicture}
        \end{center}
    \end{minipage}
    \begin{align*}
        G(P_1 P'_3) \parallel G(P_3 P'_1) \ \wedge \ G(P_1 P'_2) \parallel G(P_2 P'_1) \quad \Rightarrow \quad G(P_2 P'_3) \parallel G(P_3 P'_2)
    \end{align*}
    \textit{Beweis:}
    \begin{enumerate}
        \item[1. Fall:] $G \cap G' = \emptyset$
            \begin{align*}
                \TV(P1,P3;O) \quad & = \quad \TV(P'_3,P'_1;O) & \text{{(Strahlensatz)}}\\
                \ora{P_1 O} = k \cdot \ora{P_3 O} \quad & \wedge \quad \ora{P'_3 O} = k \cdot \ora{P'_1 O}\\
                \TV(P1,P2;O) \quad & = \quad \TV(P'_2,P'_1;O)\\
                \ora{P_1 O} = l \cdot \ora{P_2 O} \quad & \wedge \quad \ora{P'_2 O} = l \cdot \ora{P'_1 O}
            \end{align*}
            \begin{align*}
                \ora{P_3 O} = \frac{l}{k}  \cdot \ora{P_2 O} \quad & \phantom{\wedge} \quad \ora{P'_2 O} = \frac{l}{k} \cdot \ora{P'_3 O}\\
                \TV(P_3,P_2;O) \quad & = \quad \TV(P'_2,P'_3;O) & \Rightarrow \text{Behauptung}
            \end{align*}
        \item[2. Fall:] $G \parallel G'$
            \begin{align*}
                \ora{P_1P_3} = \lambda \ora{P'_3 P'_1}  \quad & \wedge \quad \ora{P_1 P'_3} = \mu \ora{P_3 P'_1}\\
                & \Rightarrow \quad \lambda = \mu = 1 & \left( \ora{P_1P_3} = \ora{P_1P'_3} + \ora{P'_3 P'_1} + \ora{P'_1 P_3} \right)
                \intertext{analog gilt:}
                \ora{P_1P_2} = \ora{P'_2 P'_1} \quad & \wedge \quad \ora{P_1 P'_2} = \ora{P_2 P_1}\\
                \Rightarrow \quad \uwave{\ora{P_2 P_3}} = \ora{P_2 P_1} + \ora{P_1 P_3} & = \ora{P'_1 P'_2} + \ora{P_3 P'_1} = \uwave{\ora{P'_2 P_3}}
            \end{align*}
    \end{enumerate}
\end{mysatz}

\begin{mydef}\textit{Euklidischer Punktraum}\medskip

    Ein affiner Raum heißt euklidisch, wenn der zugehörige Vektorraum euklidisch ist.
\end{mydef}

\begin{mydef}\textit{Abstand}\medskip

    Seien $P$ und $Q$ Punkte des euklidischen Punktraumes $\mathcal{A}_n$, dann heißt
    \begin{align*}
        d(P,Q) := \sqrt{\left\langle \ora{PQ}, \ora{PQ} \right\rangle}
    \end{align*}
    Abstand der Punkte $P,Q$. Seien $M_1$ und $M_2$ Punktmengen, dann ist
    \begin{align*}
        d(M_1,M_2) := \inf\limits_{x,y} \big\{ d(x,y) \mid x \in M_1, y \in M_2 \big\}
    \end{align*}
    (Falls $M_1,M_2$ affine Teilräume sind, definiert sich der Abstand durch das Minimum)
\end{mydef}

\begin{mysatz}\textit{Für den Abstand von Punkten gelten folgende Eigenschaften:}
    \begin{itemize}
        \item[(i)] $d(P,Q) = d(Q,P)$
        \item[(ii)] $d(P,Q) \geq 0$
        \item[(iii)] $d(P,Q) = 0 \Leftrightarrow P = Q$
        \item[(iv)] $d(P,Q) \leq d(P,R) + d(R,Q)$
    \end{itemize}
    \textit{Beweis:} siehe Norm.
\end{mysatz}

\begin{mylemma}\textsc{Hesse}\textit{sche Form}\footnote{Ludwig Otto Hesse, * 22. April 1811 in Königsberg (Preußen); $\dagger$ 4. August 1874 in München}

    Sei $G$ eine Gerade mit $G = \left\{ X : X = P+t \cdot a \right\}$ in der Ebene $\mathcal{A}_2$ und $n$ der Stellungsvektor der Geraden, d.h. $\left\langle n,a \right\rangle=0$, dann gilt
    \begin{align*}
        \left\langle \ora{PX} , n \right\rangle = 0 \Leftrightarrow X \in G
    \end{align*}
    \textit{Beweis:}
    \begin{align*}
        \left\langle n \right\rangle^{\bot} = \left\langle a \right\rangle \Rightarrow \left\langle \ora{PX} , n \right\rangle = 0 \Leftrightarrow \ora{PX} = \lambda a \Rightarrow X = P + \lambda a \in G
    \end{align*}
\end{mylemma}
\textit{Bemerkung:} \textsc{Hesse}sche Normalform: $n_0 = \frac{n}{\| n \|}$ und $\left\langle \ora{PX} , n_0 \right\rangle = 0$

\textit{Beispiel:}
\begin{align*}
    X =
    \begin{pmatrix}
        1\\2
    \end{pmatrix}
    + t
    \begin{pmatrix}
        3\\4
    \end{pmatrix}
    \qquad n =
    \begin{pmatrix}
        -4\\3
    \end{pmatrix}\\
    \left.
    \begin{matrix}
        \left\langle
        \begin{pmatrix}
            x-1\\y-2
        \end{pmatrix}
        ,
        \begin{pmatrix}
            -4\\3
        \end{pmatrix}
        \right\rangle
        & =0\\
        -4x+3y-2 & =0
    \end{matrix}
    \right\}
    \text{HF}\\
    -\frac{4}{5} x + \frac{3}{5} y - \frac{2}{5} = 0 \Rightarrow \text{HNF}
\end{align*}
\begin{mylemma}\textit{Abstand: Punkt-Gerade} $\big( d(Q,G) \big)$\\

    \begin{minipage}{0.6\textwidth}
        Sei $G = \left\{ X : X = P+t \cdot a \right\}$ eine Gerade und $Q$ ein Punkt, dann ist
        \begin{align*}
            d(Q,G) = \left| \left\langle \ora{PQ} , n_0 \right\rangle \right|
        \end{align*}
        wobei $n_0$ der normierte Stellungsvektor von $G$ ist.
    \end{minipage}
    \begin{minipage}{0.4\textwidth}
        \begin{center}
            \begin{tikzpicture}
                %\draw[help lines] (0,0) grid (4,3);
                \draw (0,0) coordinate(p) node[anchor=north west] {$P$};
                \draw (1.5,1) coordinate(qs) node[anchor=north west] {$Q^*$};
                \draw (3,2) coordinate(g) node[anchor=west] {$G$};
                \draw (0.5,2.5) coordinate(q) node[anchor=south] {$Q$};
                \fill (p) circle (1pt);
                \fill (qs) circle (1pt);
                \fill (q) circle (1pt);
                \draw (-0.5,-1/3) -- (p) -- (qs) -- (3,6/3);
                \draw[color=blue,thick] (q) -- (qs);
            \end{tikzpicture}
        \end{center}
    \end{minipage}

    \textit{Beweis:}
    \begin{align*}
        \ora{QQ^*} & \bot a\\
        \ora{Q^*Q} & = \lambda \cdot n_0 \qquad \Rightarrow \qquad \left| \ora{QQ^*} \right| = \left| \lambda \right|\\
        \ora{PQ} & = \ora{PQ^*}  + \ora{Q^*Q}\\
        \left\langle \ora{PQ^*} , n_0 \right\rangle & = 0\\
        \left\langle \ora{PQ} , n_0 \right\rangle & = \underbrace{\left\langle \ora{PQ^*} , n_0 \right\rangle}_{0} + \left\langle \lambda \cdot n_0, n_0 \right\rangle = \lambda\\
        & \Rightarrow \left\langle \ora{PQ} , n_0 \right\rangle = \lambda \quad \text{und} \quad |\lambda| = d(Q,G)
    \end{align*}
\end{mylemma}
\textit{Beispiel:}
\begin{align*}
    G & : -\frac{4x}{5} + \frac{3y}{5} - \frac{2}{5} = 0\\
    d & \left(
    \begin{pmatrix}
        2\\3
    \end{pmatrix}
    ,G
    \right)
    =
    \left| -\frac{8}{5} + \frac{9}{5} - \frac{2}{5}  \right| = \frac{1}{5}
\end{align*}

\begin{mylemma}

\end{mylemma}

% section Affine und Euklidische (Punkt-)Räume (end)


\end{document}
