\documentclass[%
a4paper,
%empty,		% keine Seitenzahlen
%a5paper,	% alle weiteren Papierformat einstellbar
11pt,		% Schriftgröße (12pt, 11pt (Standard))
%leqno,		% Nummerierung von Gleichungen links
%fleqn,		% Ausgabe von Gleichungen linksbündig
twoside,
]
{scrartcl}

%% Deutsche Anpassungen 
\usepackage[ngerman]{babel}
\usepackage[T1]{fontenc}
\usepackage[sc]{mathpazo}
\linespread{1.05}
\usepackage[utf8]{inputenc}

%obligatorischer Mathekram:
\usepackage{amssymb,amstext,dsfont,trsym,pifont}
\usepackage[sumlimits]{amsmath}
\usepackage{eulervm}

%nützlicher Mathekram:
\newcommand{\R}{\mathbb{R}}
\newcommand{\C}{\mathbb{C}}
\newcommand{\K}{\mathbb{K}}
\newcommand{\off}{\text{off}\,}
\newcommand{\rg}{\text{rg}\,}
\newcommand{\Sp}{\text{Sp}\,}
\newcommand{\charpol}{\text{charpol}\,}
\newcommand{\id}{\text{id}\,}
\newcommand{\TV}{\text{TV}}
\newcommand{\ora}{\overrightarrow}
%Borners Kommandos:
\newcommand{\mata}{\left(\begin{matrix}}
\newcommand{\mate}{\end{matrix}\right)}
\newcommand{\la}{\lambda_1}
\newcommand{\lb}{\lambda_2}

%nützliche Extras:
\usepackage{array,
hhline,
longtable,
tabularx,
enumerate,
%hyperref,
color,
setspace,
booktabs,
%cite,
caption,
%lineno,
%lastpage,
%algorithmic,
%algorithm,
ulem,
}

%schreibt "Algorithmus" statt "Algorithm":
%\floatname{algorithm}{Algorithmus}

%\usepackage{arydshln}

\usepackage[amsmath,thmmarks,thref]{ntheorem}

% meine Theoremdefinitionen:
% +------------------------+

% Definitionen:
\theoremstyle{plain}
\theoremheaderfont{\bfseries}
\theorembodyfont{}
\theoremseparator{\ }
%\theoremprework{\hfill \rule{0.5\textwidth}{1pt} \hspace*{0.25\textwidth} }
%\theorempostwork{\rule}
%\theoremindent2ex
\newtheorem{mydef}{Definition}[section]

%Sätze:
\theoremstyle{plain}
\theoremheaderfont{\bfseries}
\theorembodyfont{}
\theoremsymbol{$\Box$}
\newtheorem{mysatz}[mydef]{Satz}

%Lemmata:
\theoremstyle{plain}
\theoremheaderfont{\bfseries}
\theorembodyfont{}
\theoremsymbol{$\Box$}
\newtheorem{mylemma}[mydef]{Lemma}

%Bemerkungen:
\theoremstyle{plain}
\theoremheaderfont{\itshape}
\theorembodyfont{}
\theoremsymbol{}
\newtheorem{mybem}[mydef]{Bemerkung}

%Beispiele:
\theoremstyle{plain}
\theoremheaderfont{\itshape}
\theorembodyfont{}
\theoremsymbol{}
\newtheorem*{mybsp}[mydef]{Beispiel}

% Randverwaltung (entweder geometry oder =fullpage=)
%\usepackage[left=1cm,right=1cm,top=1cm,bottom=1cm,includeheadfoot]{geometry}
\usepackage[%cm,
%headings,
]{fullpage}

% die fancy-Header:
%\usepackage{fancyhdr}
%\pagestyle{fancy}
%\fancyhf{}
%
%\fancyhead{}
%\fancyfoot{}
%\fancyhead[L]{}
%\fancyhead[C]{}
%\fancyhead[R]{\nouppercase \leftmark}
%\fancyfoot[L]{}
%\fancyfoot[C]{\thepage}
%\fancyfoot[OR]{\thepage}
%\fancyfoot[LE]{\thepage}
%Linie oben/unten
%\renewcommand{\headrulewidth}{0.0pt}
%\renewcommand{\footrulewidth}{0.0pt}

%kein Einrücken der Paragraphen
\parindent 0pt

%% Packages für Grafiken & Abbildungen
%\usepackage{graphicx}
%\usepackage{subfig}    %%Teilabbildungen in einer Abbildung
\usepackage{tikz}      %%TeX ist kein Zeichenprogramm
%\usepackage[all]{xy}
%\usepackage{pst-all}


\begin{document}

%\setcounter{section}{6}

%\section{Eigenwerte, Eigenvektoren} % (fold)
%\label{sec:Eigenwerte, Eigenvektoren}
%
%Zum Anfang eine ganz kurze Wiederholung zu linearen Abbildungen:
%
%$\alpha: V \mapsto V$ \hfill $\alpha$ ist eine lineare Abbildung eines $\K$-Vektorraumes $V$ in sich (Endomorphismus).
%\begin{align*}
%    \alpha(v_1 + v_2) & = \alpha(v_1) + \alpha(v_2) & \forall\, v_1,v_2 \in V\\
%    \alpha(\lambda \cdot v) & = \lambda \cdot \alpha(v) & \forall\, \lambda \in \K, v \in V
%\end{align*}
%
%\begin{mydef} Eigenwert, Eigenvektor
%
%    Sei $\alpha$ ein Endomorphismus des $\K$-VR $V$ und $\lambda \in \K$. $\lambda$ heißt \textbf{Eigenwert} von $\alpha$ genau dann, wenn ein Vektor $v$ aus $V \setminus \left\{ 0 \right\}$ existiert mit $\alpha(v) = \lambda v$. $v$ heißt dann \textbf{Eigenvektor} zu $\lambda$.
%\end{mydef}
%
%\textit{Beispiel:} $V = \R^2$
%\begin{enumerate}
%    \item $\qquad \alpha:\ (x,y) \mapsto (x+y,x+y)$
%        \begin{align*}
%            \alpha(x,y) & = \lambda(x,y)\\
%            (x+y,x+y) & = (\lambda x, \lambda y)\\
%            x+y & = \lambda x\\
%            x+y & = \lambda y\\
%            \uline{\lambda = 0} \Rightarrow y & = -x & v = (1,-1)\\
%            \alpha(1,-1) & = 0 \cdot (1,-1)\\
%            v & = t(1,-1)\\
%            \uline{\lambda = 2} \Rightarrow x & = y & v = (1,1)
%        \end{align*}
%
%        Man betrachte das homogene Gleichungssystem:
%        \begin{align*}
%            \begin{pmatrix}
%                1-\lambda & 1\\
%                1 & 1-\lambda
%            \end{pmatrix}
%        \end{align*}
%        Mittels der Berechnung der Determinante lassen sich alle Lösungen ermitteln.
%
%        Man betrachte nun die Abbildungsmatrix $A$ von $\alpha$:
%        \begin{align*}
%            A = 
%            \begin{pmatrix}
%                1 & 1\\
%                1 & 1
%            \end{pmatrix}
%        \end{align*}
%        \textit{Bonusfrage:} wie sieht $A^{100}$ aus?
%
%        Dazu betrachte man $A$ bzgl. der Eigenvektorbasis $\langle (1,1),(1,-1) \rangle$.
%        \begin{align*}
%            A' & =
%            \begin{pmatrix}
%                2 & 0\\
%                0 & 0
%            \end{pmatrix}\\
%            A' & = T^{-1} A T \qquad (T = \text{Transformationsmatrix})\\
%            \langle (1,0),(0,1) \rangle & \overset{T}{\to} \langle (1,1),(1,-1) \rangle\\
%            T & = 
%            \begin{pmatrix}
%                1 & 1\\
%                1 & -1
%            \end{pmatrix}\\
%        \end{align*}
%        \begin{align*}
%            A & = T A' T^{-1}\\
%            \uline{A^n} & = (T A' T^{-1})(T A' T^{-1}) \ldots (T A' T^{-1}) = \uuline{T {A'}^n T^{-1}}\\
%            & = -\frac{1}{2}
%            \begin{pmatrix}
%                1 & 1\\
%                1 & -1
%            \end{pmatrix}
%            \begin{pmatrix}
%                2^n & 0\\
%                0 & 0
%            \end{pmatrix}
%            \begin{pmatrix}
%                -1 & -1\\
%                -1 & 1
%            \end{pmatrix}
%            =
%            -\frac{1}{2}
%            \begin{pmatrix}
%                2^n & 0\\
%                2^n & 0
%            \end{pmatrix}
%            \begin{pmatrix}
%                -1 & -1\\
%                -1 & 1
%            \end{pmatrix}\\
%            & =
%            -\frac{1}{2}
%            \begin{pmatrix}
%                -2^n & -2^n\\
%                -2^n & -2^n
%            \end{pmatrix}
%            =
%            \uuline{
%            \begin{pmatrix}
%                2^{n-1} & 2^{n-1}\\
%                2^{n-1} & 2^{n-1}
%            \end{pmatrix}
%            }
%        \end{align*}
%    \item $\qquad\alpha(x,y) = (x+y,x-y)$
%        \begin{align*}
%            x+y & = \lambda x\\
%            x-y & = \lambda y
%        \end{align*}
%        \begin{align*}
%            \Rightarrow\ &
%            A = 
%            \underbrace{
%            \begin{pmatrix}
%                1-\lambda & 1\\
%                1 & -1-\lambda 
%            \end{pmatrix}}_{\text{Wann wird der Rang 1?}}
%        \end{align*}
%        Es existieren genau dann nichttriviale Lösungen, wenn $\rg(A)<2$ und dies ist genau dann der Fall, wenn $\det A=0$.
%        \begin{align*}
%            \Rightarrow \det A & = -(1-\lambda)(1+\lambda)-1 = -2+\lambda^2\\
%            \lambda^2 & = 2 \Rightarrow \lambda_{1,2} = \pm \sqrt{2}
%        \end{align*}
%\end{enumerate}
%\subsection{Bestimmung von Eigenwerten und Eigenvektoren} % (fold)
%\label{sub:Bestimmung von Eigenwerten und Eigenvektoren}
%$\alpha: V \to V$ und $\dim V = n$
%\begin{align*}
%    \alpha(v) = \lambda v \qquad & \Leftrightarrow \qquad \underbrace{A}_{\substack{\text{Darstellungsmatrix}\\ \text{von $\alpha$ bzgl. $B$}}} \cdot  v_{'B} = \lambda v_{'B}\\
%    & \Leftrightarrow 0 = \lambda \underbrace{E_n v_{'B}}_{v_{'B}} - Av_{'B} = (\lambda E_n - A)v_{'B} \tag{$\star$} \label{eqn:LambdaEMinusA}
%\end{align*}
%
%Da $v_{'B} \in V\setminus\left\{0\right\}$ sind nichttriviale Lösungen des homogenen Gleichungssystems (\ref{eqn:LambdaEMinusA}) gesucht. Diese existieren genau dann, wenn $\rg(\lambda E - A) < n$, d.h. genau dann, wenn $\det(\lambda E-A)=0$.
%
%\begin{mysatz}
%    Sei $\alpha$ ein Endomorphismus des $\K$-VR $V$, dann sind folgende Aussagen äquivalent:
%    \begin{enumerate}
%        \item $\lambda$ ist ein Eigenwert von $\alpha$
%        \item $\det(\lambda E - A) = 0$
%        \item $\rg(\lambda E - A) < n$
%        \item $\ker(\lambda E - A) \neq \left\{ 0 \right\}$
%    \end{enumerate}
%    \begin{align*}
%        \det &
%        \begin{pmatrix}
%            \lambda - a_{11} & -a_{12} & \dots & -a_{1n}\\
%            -a_{21} & \lambda - a_{22} & \dots & -a_{2n}\\
%            \vdots & & \ddots & \vdots\\
%            -a_{n1} & \dots & \dots & \lambda - a_{nn}
%        \end{pmatrix}
%        =
%        (\lambda - a_{11}) \ldots (\lambda - a_{nn}) + \ldots - \ldots\\
%        & = (-1)^n \lambda^n + (-1)^{n-1} (\Sp A)\lambda^{n-1} + \ldots + k_0\\
%        \Big( k_0 & = \det(0 \cdot E-A) = \det(-A) = (-1)^n \det A \Big)\\
%        & = \lambda^n - (\Sp A)\lambda^{n-1} + \ldots + (-1)^n \det A
%    \end{align*}
%\end{mysatz}
%
%\begin{mydef}
%    \textit{charakteristisches Polynom} $(\charpol)$
%    \begin{enumerate}
%        \item Sei $A$ eine $n \times n$-Matrix. Das $\charpol$ von $A$ ist das Polynom $\det(\lambda E-A) \in \K\left[ \lambda \right]$
%        \item Sei $\alpha$ ein Endomorphismus von $V$, dann ist das $\charpol \alpha$ das Polynom $\det(\lambda E-A)$, wobei $A$ die Darstellungsmatrix von $\alpha$ bzgl. Basis $B$ ist.
%    \end{enumerate}
%\end{mydef}
%
%\textit{Bemerkung:} Ähnliche Matrizen haben dasselbe $\charpol$.
%\begin{align*}
%    A' & = T^{-1}AT\\
%    \charpol A' & = \det(\lambda E - T^{-1}AT) = \det(T^{-1}\lambda ET - T^{-1}AT)\\
%    & = \det( T^{-1}(\lambda E - A) T )\\
%    & = \det( T^{-1}) \cdot \det(\lambda E - A) \cdot  \det(T)\\
%    & = \underbrace{\det( T^{-1}) \cdot \det(T)}_{=1} \cdot \det(\lambda E - A) = \charpol A
%\end{align*}
%Die Eigenwerte von $\alpha$ sind die Nullstellen des charakteristischen Polynoms.\\
%
%\textit{Beispiel:} $\qquad V=\R^2 \qquad A = \begin{pmatrix}
%    1 & -1\\
%    1 & 1
%\end{pmatrix}$
%\begin{align*}
%    \charpol A = \det
%    \begin{pmatrix}
%        \lambda - 1 & -1\\
%        -1 & \lambda - 1
%    \end{pmatrix}
%    & = \lambda^2 -2 \lambda + 2\\
%    \lambda_{1,2} & = 1 \pm \sqrt{1-2} \notin \R\\
%    & \Rightarrow \text{keine Eigenwerte}
%\end{align*}
%
%\textit{Anmerkung:}
%Für die Fälle $n=2$ und $n=3$ existieren Formeln für das charakteristische Polynom:
%\begin{align*}
%    p(\lambda) & = \lambda^2 - \Sp A\, \lambda  + \det A & (n=2)\\
%    p(\lambda) & = -\lambda^3 + \Sp A\, \lambda^2 - k_1 + \det A & (n=3)\\
%    k_1 \text{ ist für } A & = 
%    \begin{pmatrix}
%        a & b & c\\
%        d & e & f\\
%        g & h & j
%    \end{pmatrix}
%    \text{ definiert als }
%    k_1 = 
%    \begin{vmatrix}
%        a & b\\
%        d & e
%    \end{vmatrix}
%    +
%    \begin{vmatrix}
%        a & c\\
%        g & j
%    \end{vmatrix}
%    +
%    \begin{vmatrix}
%        e & f\\
%        h & j
%    \end{vmatrix}
%\end{align*}
%
%%\newpage
%
%\textit{Anmerkung:}
%\textsc{Horner}-Schema:
%\begin{itemize}
%    \item[Vor.:] Polynom $f(x) = a_n x^n + a_{n-1} x^{n-1} + \ldots + a_1 x + a_0$
%    \item[Bsp.:] $f(x) = 2x^4 + x^3 -3x -34$
%\end{itemize}
%\begin{itemize}
%    \item \textit{Berechne} $f(-1)$!
%
%        Für das \textsc{Horner}-Schema benötigt man eine kleine Tabelle mit 3 Zeilen und $n+2$ Spalten, wobei $n$ der Grad des Polynoms ist.
%        \begin{enumerate}
%            \item In die Kopfzeile schreibt man alle Koeffizienten des Polynoms; in das 1. Feld der 2. Zeile den zu berechnenden Wert:
%                \begin{center}
%                    \begin{tabular}{c|ccccc}
%                        & $2$   & $1$   & $0$   & $-3$  & $-34$ \\\hline
%                        $-1$    &       &       &       &       &       \\\hline
%                        \phantom{$-1$}
%                    \end{tabular}
%                \end{center}
%            \item Das 2. Feld der 2. Zeile bleibt leer. Darunter jedoch (2. Feld, 3. Zeile) schreibt man den Wert, der in der 1. Zeile steht (also $2$).
%                \begin{center}
%                    \begin{tabular}{c|ccccc}
%                        & $2$   & $1$   & $0$   & $-3$  & $-34$ \\\hline
%                        $-1$    &       &       &       &       &       \\\hline
%                        & $2$
%                    \end{tabular}
%                \end{center}
%            \item Nun multipliziert man den eben eingetragenen Wert mit der Zahl ganz links (also dem zu berechnenden Argument) und trägt ihn das Feld darüber rechts ein.
%                \begin{center}
%                    \begin{tabular}{c|ccccc}
%                        & $2$   & $1$   & $0$   & $-3$  & $-34$ \\\hline
%                        $-1$    &       & $-2$  &       &       &       \\\hline
%                        & $2$
%                    \end{tabular}
%                \end{center}
%            \item Jetzt addiert man die Werte der 3. Spalte und trägt das Ergebnis in das unterste Feld ein.
%                \begin{center}
%                    \begin{tabular}{c|ccccc}
%                        & $2$   & $1$   & $0$   & $-3$  & $-34$ \\\hline
%                        $-1$    &       & $-2$  &       &       &       \\\hline
%                        & $2$   & $-1$
%                    \end{tabular}
%                \end{center}
%            \item Dieses Prinzip führt man bis zum Ende durch (mit dem Wert ganz links multiplizieren; in das Feld rechts darüber eintragen; summieren).
%                \begin{center}
%                    \begin{tabular}{c|ccccc}
%                        & $2$   & $1$   & $0$   & $-3$  & $-34$ \\\hline
%                        $-1$    &       & $-2$  & $1$   &       &       \\\hline
%                        & $2$   & $-1$
%                    \end{tabular}
%                    $\Rightarrow \ldots \Rightarrow$
%                    \begin{tabular}{c|ccccc}
%                        & $2$   & $1$   & $0$   & $-3$  & $-34$ \\\hline
%                        $-1$    &       & $-2$  & $1$   & $-1$  & $4$   \\\hline
%                        & $2$   & $-1$  & $1$   & $-4$  & \framebox{$-30$}
%                    \end{tabular}
%                \end{center}
%            \item[$\Rightarrow$] Der eingerahmte Wert ist der gesuchte Funktionswert.
%        \end{enumerate}
%    \item \textit{Polynomdivision}
%
%        Angenommen man hat durch Ausprobieren herausgefunden, dass $f(2)=0$ gilt und möchte jetzt das Polynom zerlegen.
%        \begin{enumerate}
%            \item Dazu schaut man sich das ausgefüllte \textsc{Horner}-Schema an:
%                \begin{center}
%                    \begin{tabular}{c|ccccc}
%                        & $2$   & $1$   & $0$   & $-3$  & $-34$ \\\hline
%                        $2$     &       & $4 $  & $10$  & $20$  & $34$   \\\hline
%                        & $2$   & $5$   & $10$  & $17$  & $0$
%                    \end{tabular}
%                \end{center}
%            \item Die Einträge der letzten Zeile bilden die Koeffizienten des Restpolynoms.
%            \item[$\Rightarrow$] Insgesamt gilt also:
%                \begin{align*}
%                    f(x) = (x-2)(2x^3+5x^2+10x+17)
%                \end{align*}
%        \end{enumerate}
%\end{itemize}
%
%%\newpage
%
%\begin{mydef} \textit{Eigenraum}
%
%    Sei $\alpha: V \to V$ ein Endomorphismus und $\lambda \in \K$ ein Eigenwert von $\alpha$, dann heißt
%
%    $E(\lambda) = \left\{ v \in V\ |\ \alpha(v) = \lambda v \right\}$ \textbf{Eigenraum} zum Eigenwert $\lambda$.
%\end{mydef}
%
%\textit{Bemerkung:}
%\begin{itemize}
%    \item $E(\lambda)$ ist die Menge aller Eigenvektoren von $\lambda$ und $0$.
%    \item $E(\lambda) = \ker(\lambda \id - \alpha)$ \hfill da $ \alpha(v) = \lambda v \Leftrightarrow 0 = \lambda \id v - \alpha(v) = (\lambda \id - \alpha) v$
%\end{itemize}
%
%\begin{mydef} \textit{Diagonalmatrix, diagonalisierbar}
%
%    Eine $n \times n$ Matrix heißt Diagonalmatrix genau dann, wenn $a_{ij}=0$ für alle $i \neq j; i,j \in \left\{ 1,\dots,n \right\}$. Man nennt $A$ diagonalisierbar genau dann, wenn $A$ zu einer Diagonalmatrix ähnlich ist.
%    Ein Endomorphismus $\alpha$ heißt diagonalisierbar, genau dann, wenn eine Basis $B$ von $V$ existiert, sodass die Abbildungsmatrix eine Diagonalmatrix ist.
%\end{mydef}
%
%\textit{Bemerkung:} $\alpha$ ist diagonalisierbar genau dann, wenn eine Basis aus Eigenvektoren existiert.
%
%% subsection Bestimmung von Eigenwerten und Eigenvektoren (end)
%
%% section Eigenwerte, Eigenvektoren (end)

%\setcounter{section}{3}
%
%\section{Orthogonale Räume und orthogonale Abbildungen} % (fold)
%
%\begin{mydef}
%    Bilinearform
%\end{mydef}
%
%\begin{mydef}
%    orthogonal, ausgeartet
%\end{mydef}
%
%\begin{mylemma}
%    Sei $(\cdot,\cdot)$ eine Bilinearform \dots 
%\end{mylemma}
%
%\begin{mysatz}
%    Euklidische Norm
%
%    Durch $\| v \| := \sqrt{\langle v,v \rangle}$ wird in einem euklidischen Vektorraum $V$ über $\R$ eine Norm definiert.
%
%    \textit{Beweis:} 1.,2.,3. folgen unmittelbar aus der Definition. 4. folgt aus dem folgenden Satz:
%\end{mysatz}
%
%\begin{mysatz}
%    Cauchy-Schwarzsche-Ungleichung
%
%    $V$ sei ein euklidischer Vektorraum, dann gilt
%    \begin{align*}
%        \forall v,w \in V: \quad \left| \langle v,w \rangle \right| \leq \| v \| \cdot \| w \| \qquad \left( \text{ bzw. } {\left\langle v,w \right\rangle}^2 \leq \left\langle v,v \right\rangle \cdot \left\langle w,w \right\rangle \right)
%    \end{align*}
%    \textit{Beweis:}
%    \begin{itemize}
%        \item $w=0 \Rightarrow$ Ungleichung gilt.
%        \item $w \neq 0 \Rightarrow \text{sei } k := \frac{\langle v,w \rangle}{\langle w,w \rangle}$
%            \begin{align*}
%                0 & \leq \left\langle v-kw, v-kw \right\rangle = \left\langle v,v \right\rangle -2k \left\langle v,w \right\rangle +k^2 \left\langle w,w \right\rangle\\
%                & = \left\langle v,v \right\rangle -2 \frac{ \left\langle v,w \right\rangle^2 }{\left\langle w,w \right\rangle } + \frac{ \left\langle v,w \right\rangle^2 }{\left\langle w,w \right\rangle^2} \cdot \left\langle w,w \right\rangle\\
%                & = \left\langle v,v \right\rangle - \frac{ \left\langle v,w \right\rangle^2 }{ \left\langle w,w \right\rangle }\\
%                \Rightarrow & \left\langle v,w \right\rangle^2 \leq \left\langle v,v \right\rangle \cdot \left\langle w,w \right\rangle 
%            \end{align*}
%    \end{itemize}
%\end{mysatz}
%Damit kann die 4. Normeigenschaft bewiesen werden:
%\begin{align*}
%    \left( \| v \| + \| w \| \right)^2 & = \| v \|^2 +2 \| v \| \cdot \| w \| + \| w \|^2 \geq \| v \|^2 + 2 \left| \left\langle v,w \right\rangle  \right| + \| w \|^2\\
%    & \geq \| v \|^2 + 2 \left\langle v,w \right\rangle + \| w \|^2 = \left\langle v+w,v+w \right\rangle = \| v+w \|^2
%\end{align*}
%Darüber hinaus ergibt sich $\forall v,w \in V\setminus\left\{ 0 \right\}:$
%\begin{align*}
%    \frac{| \left\langle v,w \right\rangle |}{\|v\| \cdot \|w\|} \leq 1 \Rightarrow -1 \leq \frac{| \left\langle v,w \right\rangle |}{\|v\| \cdot \|w\|} \leq 1 \Rightarrow \exists!\ \phi \in \left[ 0, \pi \right] \quad \text{mit } \cos \phi = \frac{\left\langle v,w \right\rangle}{\|v\| \cdot \|w\|}
%\end{align*}
%$\Rightarrow$ Definition des Winkels zwischen Vektoren $v,w \in V\setminus\left\{ 0 \right\}$
%
%Der Winkel $\phi$ zwischen zwei Vektoren $v,w \in V\setminus\left\{ 0 \right\}$ ist derjenige eindeutig bestimmte Wert $\phi$ aus dem Intervall $\left[ 0,\pi \right]$ für den gilt
%\begin{align*}
%    \cos \phi = \frac{\left\langle v,w \right\rangle}{\|v\| \cdot \|w\|}
%\end{align*}
%
%\newpage
%
%Im $\R^2,\R^3$ mit Standardskalarprodukt ist $\| v \|$ und $\sphericalangle (v,w)$ Länge und Winkel im elementargeometrischen Sinne
%\begin{center}
%    \begin{tikzpicture}%[>=angle 90]
%        \draw[help lines] (0,0) grid (4,3);
%        \draw[dotted] (0,0) -- (3,3);
%        \draw[>=latex,->] (0,0) -- (2,2);
%        \draw[>=latex,->] (0,0) -- (4.5,0);
%        \draw (1,0) arc (0:45:1cm);
%        \draw (0.6,0.2) node[scale=0.5] {$\sphericalangle(v,w)$};
%        \draw (2+0.5,-0.2) node {$v$};
%        \draw (1-0.2,1+0.2) node {$w$};
%        \draw (2,2) -- (2,0);
%        \draw (4.5,0) -- (2.25,2.25);
%        \draw (2.2,0) arc (0:90:0.2cm);
%        \draw (2.075,0.075) node {$\cdot$};
%        \draw (2.1,2.1) arc (215:315:0.2cm);
%    \end{tikzpicture}
%\end{center}

\section{Eigenwerte und Eigenvektoren}

Sei $V$ ein $n$-dimensionaler Vektorraum über dem Körper $\K$.

%---------------------------------------------Definition 1.3------------------------------------------------
%----------------------------------------Eigenwert, Eigenvektor---------------------------------------------
\begin{mydef}\textit{Eigenwert und Eigenvektor}\medskip

    Sei $\alpha$ ein Endomorphismus des $\K$-VR V und $\lambda\in \K$. $\lambda$ heißt genau dann Eigenwert von $\alpha$, wenn ein Vektor $v\in 
    V\backslash\lbrace0\rbrace$ existiert mit
    \begin{align*}
        \alpha(v) = \lambda v
    \end{align*}
    Der Vektor $v$ heißt dann Eigenvektor zum Eigenwert $\lambda$.
\end{mydef}

%-----------------------------------------------Beispiel---------------------------------------------------
\textit{Beispiel:}\medskip

Sei $V=\R^2$ und der Endomorphismus $\alpha:(x,y)\mapsto (x+y,x+y)$.
\begin{align*}
    \Rightarrow \quad & \alpha(x,y)=(x+y,x+y) = \lambda(x,y)=(\lambda x,\lambda y)\\
    & \begin{array}{cccrccc}
        \lambda=0 & \Rightarrow & y= & -x & \Rightarrow & v= &\langle \begin{pmatrix}1\\-1\end{pmatrix} \rangle \\
        \lambda=2 & \Rightarrow & y= &  x & \Rightarrow & v= &\langle \begin{pmatrix}1\\ 1\end{pmatrix} \rangle
    \end{array}
\end{align*}
Bezüglich der Standardbasis des $\R^2$ hat $\alpha$ die Abbildungsmatrix $A=\begin{pmatrix}1&1\\1&1\end{pmatrix}$, bzgl. der Basis $\langle(1,1),(1,-1)\rangle$ die
Abbildungsmatrix $D=\begin{pmatrix}2&0\\0&0\end{pmatrix}$. Dann nennt man $A$ ähnlich zu $D$, denn $A'=T^{-1}AT$, wobei $T$ die Matrix des Basiswechsels ist.\\

Bonusfrage: wie sieht $A^{n}$ aus?

Dazu betrachte man $A$ bzgl. der Eigenvektorbasis $\left\langle (1,1),(1,-1) \right\rangle$:
\begin{align*}
    A' & =
    \begin{pmatrix}
        2 & 0\\
        0 & 0
    \end{pmatrix}\\
    A' & = T^{-1} A T\\
    \langle (1,0),(0,1) \rangle & \overset{T}{\to} \langle (1,1),(1,-1) \rangle\\
    T & = 
    \begin{pmatrix}
        1 & 1\\
        1 & -1
    \end{pmatrix}
    \intertext{Und nun wird die $n$-te Potenz berechnet:}
    A & = T A' T^{-1}\\
    \uline{A^n} & = (T A' T^{-1})(T A' T^{-1}) \ldots (T A' T^{-1}) = \uuline{T {A'}^n T^{-1}}\\
    & = -\frac{1}{2}
    \begin{pmatrix}
        1 & 1\\
        1 & -1
    \end{pmatrix}
    \begin{pmatrix}
        2^n & 0\\
        0 & 0
    \end{pmatrix}
    \begin{pmatrix}
        -1 & -1\\
        -1 & 1
    \end{pmatrix}
    =
    -\frac{1}{2}
    \begin{pmatrix}
        2^n & 0\\
        2^n & 0
    \end{pmatrix}
    \begin{pmatrix}
        -1 & -1\\
        -1 & 1
    \end{pmatrix}\\
    & =
    -\frac{1}{2}
    \begin{pmatrix}
        -2^n & -2^n\\
        -2^n & -2^n
    \end{pmatrix}
    =
    \uuline{
    \begin{pmatrix}
        2^{n-1} & 2^{n-1}\\
        2^{n-1} & 2^{n-1}
    \end{pmatrix}
    }
\end{align*}

%\newpage

%----------------------------------------------------------------------------------------------------------
%-----------------------------------------------Satz 1.2---------------------------------------------------
%----------------------------------------------------------------------------------------------------------
\begin{mysatz}\ \\\medskip
    Sei $\alpha$ ein Endomorphismus von $V$, dann sind folgende Aussagen äqiuvalent
    \begin{enumerate}
        \item $\lambda$ ist Eigenwert von $\alpha$
        \item $\det(\lambda \id_V-\alpha)=0$
        \item rg$(\lambda \id_V-\alpha)<n$
        \item $\ker(\lambda \id_V-\alpha) \neq \left\{ 0 \right\}$
    \end{enumerate}
\end{mysatz}

%---------------------------------------------Definition 1.3------------------------------------------------
%----------------------------------------charakteristisches Polynom-----------------------------------------
\begin{mydef}\label{charpol}\textit{charakteristisches Polynom}
    \begin{enumerate}
        \item Sei $A$ eine $n\times n$-Matrix. Das charakteristische Polynom von $A$ (charpol $A$) ist das Polynom
            \begin{align*}
                \det(\lambda\cdot E-A)\in \K[\lambda].
            \end{align*}
        \item Sei $\alpha$ ein Endomorphismus von $V$. Das charakteristische Polynom von $\alpha$ ist charpol $\alpha=\charpol A$, wobei $A$ die Darstellungsmatrix von $\alpha$ bzgl. einer Basis von $V$ ist.
    \end{enumerate}
\end{mydef}

%-----------------------------------------------Bemerkung---------------------------------------------------
\textit{Bemerkung:}
\begin{enumerate}
    \item Die Eigenwerte von $\alpha$ sind die Nullstellen des charakteristischen Polynoms.
    \item Ähnliche Matrizen haben dasselbe charpol
        \begin{eqnarray*}
            \charpol A' &=& \det(\lambda E-T^{-1}AT) = \det(T^{-1}\lambda ET-T^{-1}AT)=\det(T^{-1}(\lambda E-A)T)\\
            &=& \det(T^{-1}) \cdot \det(\lambda E-A) \cdot \det(T) = \det(\lambda E-A)\\
            &=& \charpol A
        \end{eqnarray*}
\end{enumerate}

\textit{Beispiel:} $V = \R^2$
\begin{align*}
    A & =
    \begin{pmatrix}
        1 & -1\\
        1 & 1
    \end{pmatrix}\\
    \charpol A & = \det
    \begin{pmatrix}
        \lambda - 1 & -1\\
        -1 & \lambda - 1
    \end{pmatrix}
    = \lambda^2 -2 \lambda + 2\\
    \lambda_{1,2} & = 1 \pm \sqrt{1-2} \notin \R\\
    & \Rightarrow \text{keine Eigenwerte}
\end{align*}

\textit{Anmerkung:}\medskip

Für die Fälle $n=2$ und $n=3$ existieren Formeln für das charakteristische Polynom:
\begin{align*}
    p(\lambda) & = \lambda^2 - \left( \Sp A \right) \lambda  + \det A & (n=2)\\
    p(\lambda) & = -\lambda^3 + \left( \Sp A \right) \lambda^2 - k_1 + \det A & (n=3)\\
    k_1 \text{ ist für } A & = 
    \begin{pmatrix}
        a & b & c\\
        d & e & f\\
        g & h & j
    \end{pmatrix}
    \text{ definiert als }
    k_1 = 
    \begin{vmatrix}
        a & b\\
        d & e
    \end{vmatrix}
    +
    \begin{vmatrix}
        a & c\\
        g & j
    \end{vmatrix}
    +
    \begin{vmatrix}
        e & f\\
        h & j
    \end{vmatrix}
\end{align*}

%\newpage

\textit{Anmerkung:}
\textsc{Horner}-Schema\footnote{William George Horner, * 1786 in Bristol; $\dagger$ 22. September 1837 in Bath, englischer Mathematiker}
\begin{itemize}
    \item[Vor.:] Polynom $f(x) = a_n x^n + a_{n-1} x^{n-1} + \ldots + a_1 x + a_0$
    \item[Bsp.:] $f(x) = 2x^4 + x^3 -3x -34$
\end{itemize}
\begin{itemize}
    \item \textit{Berechne} $f(-1)$!

        Für das \textsc{Horner}-Schema benötigt man eine kleine Tabelle mit 3 Zeilen und $n+2$ Spalten, wobei $n$ der Grad des Polynoms ist.
        \begin{enumerate}
            \item In die Kopfzeile schreibt man alle Koeffizienten des Polynoms; in das 1. Feld der 2. Zeile den zu berechnenden Wert:

                \hspace{3cm}
                    \begin{tabular}{c|ccccc}
                        & $2$   & $1$   & $0$   & $-3$  & $-34$ \\\hline
                        $-1$    &       &       &       &       &       \\\hline
                        \phantom{$-1$}
                    \end{tabular}
            \item Das 2. Feld der 2. Zeile bleibt leer. Darunter jedoch (2. Feld, 3. Zeile) schreibt man den Wert, der in der 1. Zeile steht (also $2$).

                \hspace{3cm}
                    \begin{tabular}{c|ccccc}
                        & $2$   & $1$   & $0$   & $-3$  & $-34$ \\\hline
                        $-1$    &       &       &       &       &       \\\hline
                        & $2$
                    \end{tabular}
                \hspace{2cm}
                $\downarrow$
            \item Nun multipliziert man den eben eingetragenen Wert mit der Zahl ganz links (also dem zu berechnenden Argument) und trägt ihn das Feld rechts darüber ein.

                \hspace{3cm}
                    \begin{tabular}{c|ccccc}
                        & $2$   & $1$   & $0$   & $-3$  & $-34$ \\\hline
                        $-1$    &       & $-2$  &       &       &       \\\hline
                        & $2$
                    \end{tabular}
                \hspace{1.3cm}
                $\cdot (-1) \quad \nearrow$
            \item Jetzt addiert man die Werte der 3. Spalte und trägt das Ergebnis in das unterste Feld ein.

                \hspace{3cm}
                    \begin{tabular}{c|ccccc}
                        & $2$   & $1$   & $0$   & $-3$  & $-34$ \\\hline
                        $-1$    &       & $-2$  &       &       &       \\\hline
                        & $2$   & $-1$
                    \end{tabular}
                \hspace{1.5cm}
                $+ \quad \downarrow$
            \item Dieses Prinzip führt man bis zum Ende durch (mit dem Wert ganz links multiplizieren; in das Feld rechts darüber eintragen; summieren).
                \begin{center}
                    \begin{tabular}{c|ccccc}
                        & $2$   & $1$   & $0$   & $-3$  & $-34$ \\\hline
                        $-1$    &       & $-2$  & $1$   &       &       \\\hline
                        & $2$   & $-1$
                    \end{tabular}
                    $\Rightarrow \ldots \Rightarrow$
                    \begin{tabular}{c|ccccc}
                        & $2$   & $1$   & $0$   & $-3$  & $-34$ \\\hline
                        $-1$    &       & $-2$  & $1$   & $-1$  & $4$   \\\hline
                        & $2$   & $-1$  & $1$   & $-4$  & \framebox{$-30$}
                    \end{tabular}
                \end{center}
            \item[$\Rightarrow$] Der eingerahmte Wert ist der gesuchte Funktionswert.
        \end{enumerate}
    \item \textit{Polynomdivision}

        Angenommen man hat durch Ausprobieren herausgefunden, dass $f(2)=0$ gilt und möchte jetzt das Polynom zerlegen.
        \begin{enumerate}
            \item Dazu schaut man sich das ausgefüllte \textsc{Horner}-Schema an:
                \begin{center}
                    \begin{tabular}{c|ccccc}
                        & $2$   & $1$   & $0$   & $-3$  & $-34$ \\\hline
                        $2$     &       & $4 $  & $10$  & $20$  & $34$   \\\hline
                        & $2$   & $5$   & $10$  & $17$  & $0$
                    \end{tabular}
                \end{center}
            \item Die Einträge der letzten Zeile bilden die Koeffizienten des Restpolynoms.
            \item[$\Rightarrow$] Insgesamt gilt also:
                \begin{align*}
                    f(x) = (x-2)(2x^3+5x^2+10x+17)
                \end{align*}
        \end{enumerate}
\end{itemize}


%---------------------------------------------Definition 1.4------------------------------------------------
%------------------------------------------------Eigenraum--------------------------------------------------
\begin{mydef}\label{eigenraum}\textit{Eigenraum}\medskip

    Sei $\alpha:V\to V$ ein Endomorphismus und $\lambda \in \K$ ein Eigenwert von $\alpha$, dann heißt
    \begin{align*}
        E(\lambda) = \left\{ v\in V \mid \alpha(v)=\lambda v \right\}
    \end{align*}
    der Eigenraum zum Eigenwert $\lambda$.
\end{mydef}

%-----------------------------------------------Bemerkung---------------------------------------------------
\textit{Bemerkung:}
\begin{enumerate}
    \item $E(\lambda)$ ist die Menge aller EV von $\lambda$ und $0$
    \item $E(\lambda) = \ker(\lambda\id_V-\alpha) )$
    \item $E(\lambda)\leq V$
\end{enumerate}

%---------------------------------------------Definition 1.5------------------------------------------------
%------------------------------------Diagonalmatrix, diagonalisierbar---------------------------------------
\begin{mydef} \textit{Diagonalmatrix, diagonalisierbar}

    Eine $n\times n$-Matrix $A$ heißt Diagonalmatrix, wenn $a_{ij}=0$ für $i\neq j; \ i,j=1,\ldots,n$ gilt.

    Man nennt die Matrix $A$ genau dann diagonalisierbar, wenn $A$ zu einer Diagonalmatrix ähnlich ist.

    Ein Endomorphismus $\alpha$ heißt genau dann diagonalisierbar, wenn eine Basis $B$ von $V$ existiert, bzgl. der die Abbildungsmatrix eine Diagonalmatrix ist.
\end{mydef}

%-----------------------------------------------Bemerkung---------------------------------------------------
\textit{Bemerkung:}\medskip

Der Endomorphismus $\alpha$ ist genau dann diagonalisierbar, wenn es eine Basis von $V$ gibt, welche aus Eigenvektoren von $\alpha$ besteht. Die verschiedenen Einträge auf der Diagonalen sind dann genau di Eigenwerte von $\alpha$.\\

%-----------------------------------------------Beispiel---------------------------------------------------
\textit{Beispiel:}\medskip

Gegeben sei $V=\R^3$ und ein Endomorphismus in $V$, welcher bzgl. der Standardbasis die folgende Abbildungsmatrix besitzt:
\begin{align*}
    A =
    \begin{pmatrix}
        4 & 0 & 2\\
        -6 & 1 & -4\\
        -6 & 0 & -3
    \end{pmatrix}
\end{align*}
Nach \ref{charpol} ist dann $\charpol A = \det(\lambda E-A)=(\lambda-1)^2\cdot \lambda$. Demnach sind nach \ref{eigenraum}
\begin{eqnarray*}
    E(\la) & = & \ker(\la E-A) = \ker
    \begin{pmatrix}
        -3 & 0 & -2\\
        6 & 0 & 4\\
        6 & 0 & 4
    \end{pmatrix}
    = \langle
    \begin{pmatrix}
        0\\1\\0
    \end{pmatrix}
    ,
    \begin{pmatrix}
        -2\\0\\3
    \end{pmatrix}
    \rangle\\
    E(\lb) & = & \ker (\lb E-A) = \ker
    \begin{pmatrix}
        -4 & 0 & -2\\
        6 & -1 & 4\\
        6 & 0 & 3
    \end{pmatrix}
    = \langle
    \begin{pmatrix}
        1\\-2\\-2
    \end{pmatrix}
    \rangle
\end{eqnarray*}
Die Eigenräume zu den Eigenwerten $\la=1$ und $\lb=0$.

Demnach ist die Matrix $A$ ähnlich zu der Matrix
\begin{align*}
    A' =
    \begin{pmatrix}
        1 & 0 & 0\\
        0 & 1 & 0\\
        0 & 0 & 0
    \end{pmatrix}
    \qquad \mbox{ mit } \qquad 
    A' = T^{-1} \cdot A \cdot T \qquad \mbox{ und } \qquad
    T =
    \begin{pmatrix}
        0 & -2 & 1\\
        1 & 0 & -2\\
        0 & 3 & -2
    \end{pmatrix}
\end{align*}
In der Transformationsmatrix $T$ stehen spaltenweise die Eigenvektoren.\\

%-----------------------------------------------Bemerkung---------------------------------------------------
\textit{Bemerkung:}\medskip

Die Summe der Eigenwerte einer Matrix (mit der Vielfachheit genommen) entspricht der Spur der Matrix. Sei $A\in \R^{n \times n}$ ähnlich zu der 
Matrix $D=T^{-1}AT$ dann ist $D$ eine Diagonalmatrix und es gilt
\begin{align*}
    \Sp D = \Sp (T^{-1}(AT)) = \Sp (T^{-1}TA) = \Sp (A)
\end{align*}
%\end{bem}
%-----------------------------------------------Lemma 1.6---------------------------------------------------
%-----------------------------------------------------------------------------------------------------------
\begin{mylemma}\label{lemdiag}\ \medskip

    Seien $v_1, \ldots, v_s$ Eigenvektoren zu verschiedenen Eigenwerten $\lambda_1, \ldots, \lambda_s$ eines Endomorphimus $\alpha$ von $V$. Dann ist
    \begin{align*}
        \sum_{i=1}^s v_i \neq 0
    \end{align*}

    \textit{Beweis:}
    Induktion über $s$.

    Klar für $s = 1$, da per Definition $0$ kein Eigenvektor ist.

    Sei $s > 1$ und $w_i = (\lambda_s - \lambda_i) \cdot v_i$ für $i = 1, \ldots, s-1$. Dann ist $w_i$ ebenfalls ein Eigenvektor zum Eigenwert $\lambda_i$, denn
    \begin{align*}
        \alpha(w_i) = \alpha((\lambda_s - \lambda_i) \cdot v_i)=(\lambda_s - \lambda_i) \cdot \alpha(v_i)= \lambda_i \cdot (\lambda_s - \lambda_i) \cdot v_i
    \end{align*}
    Nun folgt für den Induktionsschritt
    \begin{align*}
        (\lambda_s\id_V - \alpha)\cdot \sum_{i=1}^s v_i = \sum_{i=1}^s (\lambda_s\id_v - \alpha)\cdot v_i
        =\sum_{i=1}^{s-1} (\lambda_s-\lambda_i)\cdot v_i = \sum_{i=1}^{s-1} w_i \neq 0
    \end{align*}
\end{mylemma}

%----------------------------------------------------------------------------------------------------------
%-----------------------------------------------Satz 1.7---------------------------------------------------
%----------------------------------------------------------------------------------------------------------
\begin{mysatz}\ \medskip

    Sei $\alpha$ ein Endomorphismus von $V$ und $\lambda_1,\ldots,\lambda_s$ verschiedene Eigenwerte von $\alpha$ in $\K$. Für $j=1,\ldots,s$ sei $d_j=\dim E(\lambda_j)$. Dann gilt:
    \begin{enumerate}
        \item $\sum\limits_{j=1}^s d_j \leq n$
        \item $\sum\limits_{j=1}^s d_j = n \Leftrightarrow \alpha$ ist diagonalisierbar.
    \end{enumerate}

    \textit{Beweis:}
    \begin{enumerate}
        \item Sei für $j=1,\ldots,s: \left\{ b_{ji} \mid i=1,\ldots,d_j \right\}$ eine Basis von $E(\lambda_j)$. Dann ist die Vereinigung aller Basen $\left\{ b_{ji}|=1,\ldots,s; i=1,\ldots,d_j \right\}$ linear unabhängig, denn wenn
            \begin{align*}
                0 = \sum_{j,i} k_{ji}b_{ji} = \sum_{j=1}^s v_j, \quad \mbox{wobei} \quad v_j = \sum_j k_{ji} b_{ji}
            \end{align*}
            und (für festes $j$) nicht alle $k_{ji}=0$, dann ist $v_j \neq 0$, also ein Eigenvektor zu $\lambda_j$, was aber \ref{lemdiag} widerspricht. Dann folgt auch schon die Behauptung, denn
            \begin{align*}
                n & \geq \left|\left\{ b_{ji} \mid j = 1,\ldots,s;\ i=1,\ldots,d_j \right\} \right| = \sum_{j=1}^s d_j
            \end{align*}
        \item In diesem Fall existiert eine Basis $\left\{ b_{ji} \mid =1,\ldots,s;\ i=1,\ldots,d_j \right\}$, die nur Eigenvektoren enthält. Damit ist $\alpha$ diagonalisierbar. Sei umgekehrt $\alpha$ diagonalisierbar. Dann ist $\left\{ b_{ji} \mid i=1,\ldots,d_j \right\}$ eine Basis von $E(\lambda_j)$. Damit folgt die Behauptung.
    \end{enumerate}
\end{mysatz}

%-----------------------------------------------Bemerkung---------------------------------------------------
\textit{Bemerkung:}\medskip

Sind die Eigenwerte eines Endomorphismus $\alpha$ alle verschieden, dann ist $\alpha$ diagonalisierbar.
\begin{center}
Die Umkehrung gilt im Allgemeinen nicht!
\end{center}
$\alpha$ ist diagonalisierbar, wenn das charpol vollständig in Linearfaktoren zerfällt und $\dim E(\lambda_i)=$ Vielfachheit der Nullstelle 
$\lambda_i$ in charpol $\alpha$ ist.

Falls $\alpha$ diagonalisierbar ist, so ist $V=\sum\limits_{i=1}^s E(\lambda_i) \quad \Rightarrow \quad E\left( \lambda_i \right) \cap E \left( \lambda_j \right) = \left\{ 0 \right\} \quad i \neq j$

%---------------------------------------------Definition 1.8------------------------------------------------
%----------------------------------------direkte Summe, Projektion------------------------------------------
\begin{mydef}\textit{Direkte Summe, Projektion}\medskip

    Sei $V$ ein Vektorraum über $\K$ und $U_1,\ldots,U_s$ Unterräume von $V$. Man nennt $V$ die direkte Summe der $U_i$ genau dann, wenn jeder Vektor $v$ aus $V$ sich eindeutig als Summe
    \begin{align*}
        v=\sum_{i=1}^s u_i \qquad \mbox{ mit } u_i\in U_i
    \end{align*}
    darstellen lässt (Bezeichnung: $V = U_1 \oplus \ldots \oplus U_s$).

    Wenn $V = \sum\limits_{i=1}^s \oplus U_i$, dann sei $\pi_i:V\rightarrow U_i$ definiert durch
    \begin{align*}
        \pi_i \left( \sum_{j=1}^s u_j \right) = u_i
    \end{align*}
    Man nennt $\pi_i$ die Projektion von $V$ auf $U_i$.
\end{mydef}

%-----------------------------------------------Bemerkung---------------------------------------------------
\textit{Bemerkung:}
\begin{enumerate}
    \item Der Durchschnitt zweier Unterräume einer direkten Summe ist stets trivial:
        \begin{align*}
            U_i \cap U_j = \left\{  0 \right\} \mbox{ für } i \neq j
        \end{align*}
    \item $\dim V=\sum\limits_{i=1}^s\dim U_i$
    \item Eine Summe aus Unterräumen heiß \textit{direkt}, wenn sich die Unterräume paarweise trivial schneiden.
    \item Die Summe von Eigenräumen zu verschiedenen Eigenwerten ist direkt.
    \item $\alpha$ ist genau dann diagonalisierbar, wenn $V$ die direkte Summe der Eigenräume $E(\lambda_i)$ zu den verschiedenen Eigenwerten $\lambda_i$ ist.
    \item $\pi_i$ ist ein Epimorphismus (surjektive lineare Abbildung).
\end{enumerate}

%----------------------------------------------------------------------------------------------------------
%-----------------------------------------------Satz 1.9---------------------------------------------------
%----------------------------------------------------------------------------------------------------------
\begin{mysatz}\ \medskip

    Seien $\alpha$ und $\beta$ diagonalisierbare Endomorphismen, welche miteinander kommutieren (d.h. $\alpha\beta = \beta\alpha$).
    Dann sind $\alpha$ und $\beta$ gemeinsam diagonalisierbar, d.h. es gibt eine Basis aus Eigenvektoren von $\alpha$ und $\beta$.

    \textit{Beweis:}
    $\alpha$ ist diagonalisierbar $\Rightarrow \lambda_1,\ldots,\lambda_s$ seien verschiedene Eigenwerte und $U_1,\ldots,U_s$ die dazugehörigen Eigenräume mit $V = U_1 \oplus \ldots \oplus U_s$.
    Sei $u_i\in U_i$, dann ist auch $\beta(u_i)=u_i'\in U_i$, weil
    \begin{align*}
        \alpha(u_i')= \alpha(\beta(u_i)) = \beta\alpha(u_i) = \beta \lambda_i u_i = \lambda_i u_i'
    \end{align*}
    Da $\beta$ diagonalisierbar ist, existiert eine Basis $\left\{ b_1,\ldots,b_n \right\}$ von $V$ aus Eigenvektoren von $\beta$, d.h. $\beta(b_j)=\mu_j\cdot b_j$. Jedes $b_j$ lässt sich eindeutig schreiben als $b_j=\sum\limits_{i=1}^s u_{ij}$ mit $u_{ij}\in U_i.$ Dann ist
    \begin{align*}
        \sum_{i=1}^s \beta(u_{ij}) = \beta(b_j)=\mu_j\cdot b_j = \sum_{i=1}^s\mu_j u_{ij}
    \end{align*}
    Da $u_{ij}' = \beta(u_{ij}) \in U_i$ für jeden $i$ und $\mu_ju_{ij} \in U_i$, folgt die Eindeutigkeit $\beta(u_{ij}) = \mu_j u_{ij}$. Demnach ist, wenn $u_{ij}\neq0$ gilt $u_{ij}$ ein Eigenwert zu $\alpha$ mit $\alpha(u_{ij}) = \lambda_i \cdot u_{ij}$, aber auch ein Eigenwert zu $\beta$ mit $\beta(u_{ij})=\mu_j\cdot u_{ij}$ und daher ist die Menge $\left\{ u_{ij} \mid i = 1,\ldots,s;\ j = 1,\ldots,n \right\}$ ein Erzeugendensystem, aus dem man eine Basis wählen kann.
\end{mysatz}

%-----------------------------------------------Beispiel---------------------------------------------------
\textit{Beispiel:}\medskip

Gegeben Sei $V=\R^2$ und die Endomorphismen $\alpha,\beta$, die bzgl. der Standardbasis die folgenen Abbildungsmatrizen besitzen.
\begin{align*}
    \alpha:A=\begin{pmatrix}1&0\\1&2\end{pmatrix} \qquad \mbox{und} \qquad \beta:B=\begin{pmatrix}3&0\\-4&-1\end{pmatrix} \qquad \mbox{mit} \qquad AB=BA=\begin{pmatrix}3&0\\-5&-2\end{pmatrix}
\end{align*}
Dann ist 
\begin{align*}
    E_{\alpha}(1) = \langle \begin{pmatrix}1\\-1\end{pmatrix} \rangle = E_{\beta}(3) \mbox{ und }	E_{\alpha}(2) = \langle \begin{pmatrix}0\\ 1\end{pmatrix} \rangle = E_{\beta}(-1)
\end{align*}


\section{Affine und Euklidische (Punkt-)Räume} % (fold)
\label{sec:Affine und Euklidische (Punkt-)Räume}

\begin{mydef}
    Affiner Raum
\end{mydef}

\begin{mybem}
    \begin{itemize}
        \item $\mathcal{A}_n = \left\{ P +_A V \right\} = \left\{ P +_A v \mid v \in V \right\}$
    \end{itemize}
\end{mybem}

\begin{mysatz}
    Eigenschaften

    Für alle $P,Q,R,S \in \mathcal{A}_n$ und  für alle $v,w \in V$ gilt:
\end{mysatz}

\begin{mysatz}
    Koordinatentransformation
\end{mysatz}

\begin{mydef}
    Affiner Teilraum
\end{mydef}

\begin{mysatz}
    \ \\
    Seien $T_1 = \left\{ P+U \right\}$ und $T_2 = \left\{ Q+U \right\}$ affine Teilräume eines affinen Raumes $\mathcal{A}_n$ mit Vektorraum $V$.
\end{mysatz}

\begin{mysatz}
    Schnitte
\end{mysatz}

\begin{mylemma}
    kleinster Teilraum
\end{mylemma}

\begin{mydef}
    Punkte in allgemeiner Lage
\end{mydef}

\begin{mydef}
    Summe von Teilräumen
\end{mydef}

\begin{mysatz}
    Dimensionssatz
\end{mysatz}

\begin{mydef}
    Parallelität
\end{mydef}

\begin{mydef}
    Strecke, Strahl, Mittelpunkt
\end{mydef}

\begin{mydef}
    Teilverhältnis
\end{mydef}

\begin{mysatz}
    Seitenhalbierende
\end{mysatz}

\begin{mysatz}
    Strahlensatz (\textsc{Thales})
\end{mysatz}

\newpage

\begin{mysatz}
    Satz von \textsc{Menelaos} (70? - 140?)

    \begin{minipage}{0.45\textwidth}
        Gegeben sei ein Dreieck $ABC$ und eine Gerade $G$, die nicht durch einen Eckpunkt geht. Seien $A'$ der Schnittpunkt von $G$ mit $G(BC)$, $B'$ der Schnittpunkt von $G$ mit $G(AC)$ und $C'$ der Schnittpunkt von $G$ mit $G(AB)$, dann gilt
    \end{minipage}
    \begin{minipage}{0.55\textwidth}
        \begin{center}
            \begin{tikzpicture}
                \draw (0,0) coordinate(as) node[anchor=east] {$A'$};
                \draw (1,1) coordinate(c) node[anchor=south east] {$C$};
                \draw (3,3) coordinate(b) node[anchor=south] {$B$};
                \draw (4,1) coordinate(a) node[anchor=north west] {$A$};
                \draw (2.5,1) coordinate(bs) node[anchor=north] {$B'$};
                \draw (3.74,1.5) coordinate(cs) node[anchor=south west] {$C'$};
                \fill (as) circle (1pt);
                \fill (c) circle (1pt);
                \fill (b) circle (1pt);
                \fill (a) circle (1pt);
                \fill (bs) circle (1pt);
                \fill (cs) circle (1pt);
                \draw (a) -- (b) -- (c) -- (a);
                \draw[dashed] (as) -- (c);
                \draw (as) -- (bs) -- (cs);
            \end{tikzpicture}
        \end{center}
    \end{minipage}
    \begin{align*}
        \underbrace{\TV(A,B;C')}_{v}
        \cdot
        \underbrace{\TV(B,C;A')}_{w}
        \cdot
        \underbrace{\TV(C,A;B')}_{u}
        = 1
    \end{align*}
    \textit{Beweis:}
    \begin{align*}
        \ora{AC'} & = v \cdot \ora{BC'} = \frac{v}{v-1} \ora{AB}\\
        \ora{BA'} & = w \cdot \ora{CA'} = \frac{w}{w-1} \ora{BC}\\
        \ora{CB'} & = u \cdot \ora{AB'} = \frac{u}{u-1} \ora{CA}\\
        \ora{AB} & =\ : c\\
        \ora{AC} & =\ : b\\
        \ora{AC'} & = \frac{v}{v-1} c\\
        \ora{AB'} & = \frac{-1}{u-1} b\\
        \ora{AA'} & = \ora{AB} + \ora{BA'} = c + \frac{w}{w-1} \ora{BC}\\
        \frac{w}{w-1} b & - \frac{1}{w-1} c = c + \frac{w}{w-1} \left( b-c \right)\\
        \ora{A'B'} & = \ora{A'A} + \ora{AB'} = \frac{-w}{w-1} b + \frac{1}{w-1} c - \frac{1}{u-1} b\\
        \ora{A'C'} & = \ora{A'A} + \ora{AC'} = \frac{-w}{w-1} b + \frac{1}{w-1} c + \frac{v}{v-1} c
    \end{align*}
    weil $A',B',C' \in G \Rightarrow \exists !\ \lambda :\ \lambda \ora{A'B'} = \ora{A'C'}$
    \begin{align*}
        \lambda \left( \frac{1}{w-1}c - \frac{w}{w-1}b - \frac{1}{u-1}b \right) & = \frac{1}{w-1}c - \frac{w}{w-1}b + \frac{v}{v-1}c\\
        \left( -\frac{w \lambda}{w-1} - \frac{\lambda}{u-1} + \frac{w}{w-1} \right)b & + \left( \frac{\lambda}{w-1} - \frac{1}{w-1} - \frac{v}{v-1} \right)c = 0
    \end{align*}
    da $b,c$ linear unabhängig sind:
    \begin{align*}
        \frac{\lambda}{w-1} - \frac{1}{w-1} - \frac{v}{v-1} = 0 \Rightarrow \lambda = \left( \frac{v}{v-1} + \frac{1}{w-1} \right)(w-1) = \frac{v(w-1)}{v-1} +1
    \end{align*}
    Da auch der andere Koeffizient 0 ist, muss gelten:
    \begin{align*}
        0 & = - \frac{w \left( \frac{v(w-1)}{v-1} + 1 \right)}{w-1} - \frac{\frac{v(w-1)}{v-1}+1}{u-1} + \frac{w}{w-1}\\
        0 & = - \frac{w(v(w-1))+ v-1}{w-1} - \frac{v(w-1)+v-1}{u-1} + \frac{w(v-1)}{w-1}\\
        0 & = - \frac{w(vw-1)}{w-1} - \frac{vw-1}{u-1} + \frac{vw-w}{w-1}\\
        0 & = -w(vw-1)(u-1) - (vw-1)(w-1) + (vw-w)(u-1)\\
        0 & = -uvw^2+vw^2+uw-w-vw^2+vw+w-1+uvw-vw-uw+w\\
        0 & = uvw(1-w)+(w-1)\\
        1 & = uvw
    \end{align*}
\end{mysatz}

\begin{mysatz}
    Satz von \textsc{Ceva}\footnote{Giovanni Ceva, * 7. Dezember 1647 in Mailand; $\dagger$ 15. Juni 1734 in Mantua}

    \begin{minipage}{0.45\textwidth}
        Schneiden sich die 3 Ecktransversalen eines Dreiecks $ABC$ in einem Punkt $P$ und seien $A',B',C'$ wie oben, dann gilt
    \end{minipage}
    \begin{minipage}{0.55\textwidth}
        \begin{center}
            \begin{tikzpicture}
                %\draw[help lines] (0,0) grid (3,3);
                \draw (0,0) coordinate(c) node[anchor=south east] {$C$};
                \draw (2,2) coordinate(b) node[anchor=south] {$B$};
                \draw (3,0) coordinate(a) node[anchor=north west] {$A$};
                \draw (2.5,1) coordinate(cs) node[anchor=south west] {$C'$};
                \draw (2,0.8) coordinate(p) node[anchor=south west] {$P$};
                \draw (4/3,4/3) coordinate(as) node[anchor=east] {$A'$};
                \draw (2,0) coordinate(bs) node[anchor=north] {$B'$};
                \fill (c) circle (1pt);
                \fill (b) circle (1pt);
                \fill (a) circle (1pt);
                \fill (cs) circle (1pt);
                \fill (p) circle (1pt);
                \fill (as) circle (1pt);
                \fill (bs) circle (1pt);
                \draw (a) -- (b) -- (c) -- (a);
                \draw (c) -- (cs);
                \draw (a) -- (as);
                \draw (b) -- (bs);
            \end{tikzpicture}
        \end{center}
    \end{minipage}
    \begin{align*}
        \TV(A,B;C') \cdot \TV(B,C;A') \cdot \TV(C,A;B') = 1
    \end{align*}
    \textit{Beweis:} Übungsaufgabe
\end{mysatz}
\begin{mysatz}
    Satz von \textsc{Pappos} (um 300, Alexandria)

    \begin{minipage}{0.45\textwidth}
        $G,G'$ verschiedene Geraden in einer affinen Ebene und $P_1,P_2,P_3 \in G$ und $P'_1,P'_2,P'_3 \in G'$, dann gilt:
    \end{minipage}
    \begin{minipage}{0.55\textwidth}
        \begin{center}
            \begin{tikzpicture}
                %\draw[help lines] (0,0) grid (4,3);
                \draw (0,0) coordinate(p3s) node[anchor=north] {$P'_3$};
                \draw (0,1) coordinate(p1) node[anchor=south] {$P_1$};
                \draw (1.5,0) coordinate(p2s) node[anchor=north] {$P'_2$};
                \draw (3.5,0) coordinate(p1s) node[anchor=north] {$P'_1$};
                \draw (3.5,1.7) coordinate(p3) node[anchor=south] {$P_3$};
                \draw (20/13,17/13) coordinate(p2) node[anchor=south] {$P_2$};
                \fill (p3s) circle (1pt);
                \fill (p1) circle (1pt);
                \fill (p2s) circle (1pt);
                \fill (p1s) circle (1pt);
                \fill (p3) circle (1pt);
                \fill (p2) circle (1pt);
                \draw (-0.3,0) -- (p3s) -- (p2s) -- (p1s) -- (3.8,0);
                \draw (-0.3,0.94) -- (p1) -- (p3) -- (3.8,1.76);
                \draw[dotted] (p1) -- (p3s);
                \draw[dotted] (p3) -- (p1s);
                \draw[dotted] (p1) -- (p2s);
                \draw[dotted] (p2) -- (p1s);
                \draw[color=blue,thick] (p2) -- (p3s);
                \draw[color=blue,thick] (p2s) -- (p3);
            \end{tikzpicture}
        \end{center}
    \end{minipage}
    \begin{align*}
        G(P_1 P'_3) \parallel G(P_3 P'_1) \ \wedge \ G(P_1 P'_2) \parallel G(P_2 P'_1) \quad \Rightarrow \quad G(P_2 P'_3) \parallel G(P_3 P'_2)
    \end{align*}
    \textit{Beweis:}
    \begin{enumerate}
        \item[1. Fall:] $G \cap G' = \emptyset$
            \begin{align*}
                \TV(P1,P3;O) \quad & = \quad \TV(P'_3,P'_1;O) & \text{{(Strahlensatz)}}\\
                \ora{P_1 O} = k \cdot \ora{P_3 O} \quad & \wedge \quad \ora{P'_3 O} = k \cdot \ora{P'_1 O}\\
                \TV(P1,P2;O) \quad & = \quad \TV(P'_2,P'_1;O)\\
                \ora{P_1 O} = l \cdot \ora{P_2 O} \quad & \wedge \quad \ora{P'_2 O} = l \cdot \ora{P'_1 O}
            \end{align*}
            \begin{align*}
                \ora{P_3 O} = \frac{l}{k}  \cdot \ora{P_2 O} \quad & \phantom{\wedge} \quad \ora{P'_2 O} = \frac{l}{k} \cdot \ora{P'_3 O}\\
                \TV(P_3,P_2;O) \quad & = \quad \TV(P'_2,P'_3;O) & \Rightarrow \text{Behauptung}
            \end{align*}
        \item[2. Fall:] $G \parallel G'$
            \begin{align*}
                \ora{P_1P_3} = \lambda \ora{P'_3 P'_1}  \quad & \wedge \quad \ora{P_1 P'_3} = \mu \ora{P_3 P'_1}\\
                & \Rightarrow \quad \lambda = \mu = 1 & \left( \ora{P_1P_3} = \ora{P_1P'_3} + \ora{P'_3 P'_1} + \ora{P'_1 P_3} \right)
                \intertext{analog gilt:}
                \ora{P_1P_2} = \ora{P'_2 P'_1} \quad & \wedge \quad \ora{P_1 P'_2} = \ora{P_2 P_1}\\
                \Rightarrow \quad \uwave{\ora{P_2 P_3}} = \ora{P_2 P_1} + \ora{P_1 P_3} & = \ora{P'_1 P'_2} + \ora{P_3 P'_1} = \uwave{\ora{P'_2 P_3}}
            \end{align*}
    \end{enumerate}
\end{mysatz}

\begin{mydef}
    Euklidischer Punktraum

    Ein affiner Raum heißt euklidisch, wenn der zugehörige Vektorraum euklidisch ist.
\end{mydef}

\begin{mydef}
    Abstand

    Seien $P$ und $Q$ Punkte des euklidischen Punktraumes $\mathcal{A}_n$, dann heißt
    \begin{align*}
        d(P,Q) := \sqrt{\left\langle \ora{PQ}, \ora{PQ} \right\rangle}
    \end{align*}
    Abstand der Punkte $P,Q$. Seien $M_1$ und $M_2$ Punktmengen, dann ist
    \begin{align*}
        d(M_1,M_2) := \inf\limits_{x,y} \big\{ d(x,y) \mid x \in M_1, y \in M_2 \big\}
    \end{align*}
    (Falls $M_1,M_2$ affine Teilräume sind, definiert sich der Abstand durch das Minimum)
\end{mydef}

\begin{mysatz}
    Für den Abstand von Punkten gelten folgende Eigenschaften:
    \begin{itemize}
        \item[(i)] $d(P,Q) = d(Q,P)$
        \item[(ii)] $d(P,Q) \geq 0$
        \item[(iii)] $d(P,Q) = 0 \Leftrightarrow P = Q$
        \item[(iv)] $d(P,Q) \leq d(P,R) + d(R,Q)$
    \end{itemize}
    \textit{Beweis:} siehe Norm.
\end{mysatz}

\begin{mylemma}
    \textsc{Hesse}sche Form (1811 - 1874)

    Sei $G$ eine Gerade mit $G = \left\{ X : X = P+t \cdot a \right\}$ in der Ebene $\mathcal{A}_2$ und $n$ der Stellungsvektor der Geraden, d.h. $\left\langle n,a \right\rangle=0$, dann gilt
    \begin{align*}
        \left\langle \ora{PX} , n \right\rangle = 0 \Leftrightarrow X \in G
    \end{align*}
    \textit{Beweis:}
    \begin{align*}
        \left\langle n \right\rangle^{\bot} = \left\langle a \right\rangle \Rightarrow \left\langle \ora{PX} , n \right\rangle = 0 \Leftrightarrow \ora{PX} = \lambda a \Rightarrow X = P + \lambda a \in G
    \end{align*}
\end{mylemma}
\textit{Bemerkung:} \textsc{Hesse}sche Normalform: $n_0 = \frac{n}{\| n \|}$ und $\left\langle \ora{PX} , n_0 \right\rangle = 0$

\textit{Beispiel:}
\begin{align*}
    X =
    \begin{pmatrix}
        1\\2
    \end{pmatrix}
    + t
    \begin{pmatrix}
        3\\4
    \end{pmatrix}
    \qquad n =
    \begin{pmatrix}
        -4\\3
    \end{pmatrix}\\
    \left.
    \begin{matrix}
        \left\langle
        \begin{pmatrix}
            x-1\\y-2
        \end{pmatrix}
        ,
        \begin{pmatrix}
            -4\\3
        \end{pmatrix}
        \right\rangle
        & =0\\
        -4x+3y-2 & =0
    \end{matrix}
    \right\}
    \text{HF}\\
    -\frac{4}{5} x + \frac{3}{5} y - \frac{2}{5} = 0 \Rightarrow \text{HNF}
\end{align*}
\begin{mylemma}
    Abstand: Punkt-Gerade $\big( d(Q,G) \big)$\\

    \begin{minipage}{0.6\textwidth}
        Sei $G = \left\{ X : X = P+t \cdot a \right\}$ eine Gerade und $Q$ ein Punkt, dann ist
        \begin{align*}
            d(Q,G) = \left| \left\langle \ora{PQ} , n_0 \right\rangle \right|
        \end{align*}
        wobei $n_0$ der normierte Stellungsvektor von $G$ ist.
    \end{minipage}
    \begin{minipage}{0.4\textwidth}
        \begin{center}
            \begin{tikzpicture}
                %\draw[help lines] (0,0) grid (4,3);
                \draw (0,0) coordinate(p) node[anchor=north west] {$P$};
                \draw (1.5,1) coordinate(qs) node[anchor=north west] {$Q^*$};
                \draw (3,2) coordinate(g) node[anchor=west] {$G$};
                \draw (0.5,2.5) coordinate(q) node[anchor=south] {$Q$};
                \fill (p) circle (1pt);
                \fill (qs) circle (1pt);
                \fill (q) circle (1pt);
                \draw (-0.5,-1/3) -- (p) -- (qs) -- (3,6/3);
                \draw[color=blue,thick] (q) -- (qs);
            \end{tikzpicture}
        \end{center}
    \end{minipage}

    \textit{Beweis:}
    \begin{align*}
        \ora{QQ^*} & \bot a\\
        \ora{Q^*Q} & = \lambda \cdot n_0 \qquad \Rightarrow \qquad \left| \ora{QQ^*} \right| = \left| \lambda \right|\\
        \ora{PQ} & = \ora{PQ^*}  + \ora{Q^*Q}\\
        \left\langle \ora{PQ^*} , n_0 \right\rangle & = 0\\
        \left\langle \ora{PQ} , n_0 \right\rangle & = \underbrace{\left\langle \ora{PQ^*} , n_0 \right\rangle}_{0} + \left\langle \lambda \cdot n_0, n_0 \right\rangle = \lambda\\
        & \Rightarrow \left\langle \ora{PQ} , n_0 \right\rangle = \lambda \quad \text{und} \quad |\lambda| = d(Q,G)
    \end{align*}
\end{mylemma}
\textit{Beispiel:}
\begin{align*}
    G & : -\frac{4x}{5} + \frac{3y}{5} - \frac{2}{5} = 0\\
    d & \left(
    \begin{pmatrix}
        2\\3
    \end{pmatrix}
    ,G
    \right)
    =
    \left| -\frac{8}{5} + \frac{9}{5} - \frac{2}{5}  \right| = \frac{1}{5}
\end{align*}

\begin{mylemma}
    
\end{mylemma}

% section Affine und Euklidische (Punkt-)Räume (end)


\end{document}
